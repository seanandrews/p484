%\documentclass[referee]{aa}
\documentclass{aa}
%\documentclass{mn2e}
\usepackage[usenames,dvipsnames]{xcolor}
\usepackage{natbib}
\usepackage{xspace}
\usepackage{amssymb}
\usepackage{amsmath}
\usepackage{graphicx}
\usepackage{hyperref}
\def\remark#1{{{\bf remark:} \bf #1}}
\def\action#1{{\bf #1}}
\def\putin#1{{\it #1}}
\def\revised#1{{\bf #1}}
\newcommand{\todo}[1]{\textcolor{red}{\bf #1}}
%
\def\aap{A\&A}
\def\apss{A\&SS}
\def\pasp{Proc.Astr.Soc.Pacific}
\def\pasj{Publ.Astr.Soc.Japan}
\def\araa{AnnRevA\& A}
\def\mnras{MNRAS}
\def\icarus{Icarus}
\def\nat{Nature}
\def\apj{ApJ}
\def\aj{AJ}
\def\apjl{ApJL}
\def\apjs{ApJS}
\def\jqsrt{JQSRT}
\def\max{\mathrm{max}}
\def\erg{\hbox{erg}}
\def\sec{\hbox{s}}
\def\Hz{\hbox{Hz}}
\def\gram{\hbox{g}}
\def\cm{\hbox{cm}}
\def\km{\hbox{km}}
\def\kel{\hbox{K}}
\def\KeV{\hbox{KeV}}
\def\MeV{\hbox{MeV}}
\def\eV{\hbox{eV}}
\def\yr{\hbox{yr}}
\def\AU{\hbox{AU}}
\def\ster{\hbox{ster}}
%\def\in{\mathrm{in}}
\def\out{\mathrm{out}}
\def\eff{\mathrm{eff}}
\def\rad{\mathrm{rad}}
\def\vrt{\mathrm{vert}}
\def\vset{\ensuremath{v_\mathrm{sett}}\xspace}
\def\fact{\ensuremath{\xi}\xspace}
\def\cs{\ensuremath{c_\mathrm{s}}\xspace}
\def\siggas{\ensuremath{\sigma_\mathrm{g}}\xspace}
\def\sigcoag{\ensuremath{\sigma_\mathrm{c}}\xspace}
\def\Tgas{\ensuremath{T_\mathrm{gas}}\xspace}
%\def\rhogas{\ensuremath{\rho_\mathrm{gas}}\xspace}
\def\rhogas{\ensuremath{\rho}\xspace}
\def\rhodust{\ensuremath{\rho_\mathrm{d}}\xspace}
\def\mugas{\ensuremath{\mu_\mathrm{gas}}\xspace}
\def\mp{\ensuremath{m_\mathrm{p}}\xspace}
\def\Hp{\ensuremath{H_\mathrm{p}}\xspace}
\def\Hs{\ensuremath{H_\mathrm{s}}\xspace}
\def\Omegak{\ensuremath{\Omega_\mathrm{K}}\xspace}
\def\Mstar{\ensuremath{M_\star}\xspace}
\def\Sc{\ensuremath{\mathrm{Sc}}\xspace}
\def\St{\ensuremath{\mathrm{St}}\xspace}
\def\Pt{\ensuremath{\mathrm{Pt}}\xspace}
\def\comma{\,,}
\def\fullstop{\,.}
%
\def\submitted{submitted\xspace}
\def\inprep{in prep.\xspace}
%
\begin{document}
%\thesaurus{02.01.2,08.03.4,08.06.2,08.16.5,13.09.6}
\title{Dust trapping in thin-ringed protoplanetary disks}  \titlerunning{Thin-ringed disks}
\authorrunning{Dullemond, Birnstiel, Perez, Andrews, Isella, Huang, Zhu, Guzman, Benisty, Carpenter}
\author{C.P.~Dullemond$^{1}$, T.~Birnstiel$^{2}$, L.~Perez$^{3}$, S.~Andrews$^{4}$, A.~Isella$^{5}$, J.~Huang$^{4}$, Z.~Zhu$^{6}$, V.~Guzm\'an$^{7}$, M.~Benisty$^{8}$, J.~Carpenter$^{9}$}
\institute{
  (1) Zentrum f\"ur Astronomie, Heidelberg University, Albert Ueberle Str.~2, 69120 Heidelberg, Germany\\
  (2) {\bf XXXX}
  (3) {\bf XXXX}
  (4) {\bf XXXX}
  (5) {\bf XXXX}
  (6) {\bf XXXX}
} \date{\today}

\abstract{A large fraction of the protoplanetary disks observed with the ALMA
  Large Programme {\bf XXXX} display multiple well-defined and nearly perfectly
  circular rings in the continuum, in many cases with substantial peak-to-valley
  contrast. Several of these rings are very narrow in radial extent.
  In this paper we analyze these results using the assumption that
  these dust rings are caused by dust trapping in radial pressure bumps. We
  model this process in a 1-D axisymmetric way, initially with a simple analytic
  model, then with a more detailed numerical model. We find that .....
%  including time-dependent viscous disk evolution coupled to time-dependent
%  radial dust drift. We perform model calculations without and with dust growth,
%  the latter with a simplified dust growth recipe. We test several scenarios
%  for the origin of the pressure bumps, including planetary gaps and dead zones.
{\bf CONCLUSIONS}
}

\maketitle



\begin{keywords}
accretion, accretion disks -- circumstellar matter 
-- stars: formation, pre-main-sequence -- infrared: stars 
\end{keywords}

\section{Introduction}
The concept of dust trapping in local pressure maxima has become a central theme
in studies of planet formation and protoplanetary disk evolution, as it may
provide an elegant solution to several problems in these fields of
study. Theories of planet formation are plagued by the ``radial drift barrier'':
the problem that as dust aggregates grow by coagulation, they tend to radially
drift toward the star before they reach planetsimal size
\citep[e.g.][]{2010A&A...513A..79B}. A natural solution to this problem could be
the trapping of dust particles in local pressure maxima
\citep{1972fpp..conf..211W, 2007ApJ...664L..55K, 1995A&A...295L...1B,
  1997Icar..128..213K}. From an observational perspective, the radial drift
problem manifests itself by the presence of large grains in the outer regions of
protoplanetary disks \citep{2003A&A...403..323T, 2009ApJ...700.1502A}, which
appears to be in conflict with theoretical predictions
\citep{2007A&A...469.1169B}. One possible solution to this observational
conundrum could be that the disks are much more massive in the gas than
previously suspected, leading to a higher gas friction for millimeter grains and
thus longer drift times \citep{2017ApJ...840...93P}. But another explanation is
to invoke, again, pressure traps. The most striking observational evidence for
dust trapping seems to come from large transitional disks, which feature giant
dust rings, sometimes lopsided, in which large quantities of dust appears to be
concentrated \citep{2013Natur.493..191C, 2013Sci...340.1199V}. These observations
appear to be well explained by the dust trapping scenario
\citet{2012A&A...545A..81P}. But these transitional disks seem to be rather
violent environments, possibly with strong warps \citep{2015ApJ...798L..44M,
  2017A&A...597A..42B} and companion-induced spirals
\citep{2016ApJ...816L..12D}. For more ``normal'' protoplanetary disk the dust
traps would have to be more subtle. \cite{2012A&A...538A.114P} explored the
possibility that the disk features many axisymmetric local pressure maxima, and
calculated how the dust drift and growth would behave under such conditions. It
was found that, if the pressure bumps are strong enough, the dust trapping can
keep a sufficient fraction of the dust mass at large distances from the star to
explain the observed dust millimeter flux. It would leave, however, a detectable
pattern of rings that should be discernable with ALMA observations.  Since the
multi-ringed disk observation of HL Tau \citep{2015ApJ...808L...3A} a number of
such multi-ringed disks have been detected \citep{2016ApJ...820L..40A,
  2016PhRvL.117y1101I, 2017ApJ...851L..23C, 2017A&A...600A..72F,
  2018A&A...610A..24F}. It is therefore very tempting to see also these
multi-ringed disks as evidence for dust trapping, and as an explanation for
the retention of dust in the outer regions of protoplanetary disks.

The data from the ALMA Large Programme 2016.1.00484.L offers an exciting new
opportunity to put this concept to the test, and to put constraints on the
physics of dust trapping in axisymmetric pressure maxima. This is an opportunity
which we explore in this paper.

If we assume that the rings seen in our data are caused by dust trapping, the
question arises: what constraints do the data impose on the physics of dust
trapping? To explore this we start by studying the rings individually, assuming
that the dust does not escape from the ring. This makes it possible to look for
a steady-state dust trapping model in which the radial drift forces (that push
the dust to the pressure peak) are balanced by turbulent mixing (that tends to
smear out the dust away from the pressure peak). In Section
\ref{sec-steady-state-analytic-trap-model} we will construct a very simplified
analytic dust trapping model, and confront this with the most well-isolated
rings from our sample. {\bf Describe the next sections.} But before we do the
modeling, we review, in Section \ref{sec-data}, the key aspects of the data from
our ALMA Large Programme that we compare our models to.



\section{The high-contrast rings of AS 209, Elias 24, GW Lup, HD 143006 and HD 163296}
\label{sec-data}
%
In this paper we discuss in particular those sources of the ALMA Large Programme
that show high-contrast and radially thin rings that are separated by deep
valleys, and that are sufficiently face-on to not have to worry about 3-D
lie-of-sight aspects. These are AS 209, Elias 24, GW Lup, HD 143006 and HD
163296. The high-contrast rings of these sources provide ``clean laboratories''
for testing the theory of dust trapping, both in a ring-by-ring manner, as well
as by comparing to global dust drift models. The most striking case is AS 209,
which features two high-contrast radially thin rings which are isolated and
large ({\bf see Guzm\'an et al.~(2018) for the detailed description of this
  dataset}). Fig.\ref{fig-obs-as209} shows the radial profile (deprojected for
inclination) of the thermal emission of the dust of the disk around AS 209. The
brightness is expressed as linear brightness temperature
($T_{\mathrm{bright,linear}}=(c^2/2k_B)I_\nu/\nu^2$), where $I_\nu$ is the
intensity, $c$ the light speed, $k_B$ the Boltzmann constant, and $\nu$ the
frequency. It should be noted that the full (non-linear) brightness temperature
(defined such that $B_\nu(T_{\mathrm{bright,full}})=I_\nu$, with $B_\nu(T)$ the
Planck function) can have very different values, in particular at low
brightness. As a result $T_{\mathrm{bright,full}}(r)$ tends to vastly
underestimate the contrast between the rings and the inter-ring
regions. Therefore, in this paper we plot the brightness using the linear
brightness temperature.

\begin{figure}
\centerline{\includegraphics[width=0.5\textwidth]{as209_gauss_modelfit.pdf}}
\caption{\label{fig-obs-as209}The linear brightness temperature in band 6 of the
  disk around AS 209, as a function of deprojected distance to the star (see
  {\bf Guzm\'an et al.~(2018)} for a detailed description and analysis of these data). The
  vertical axis is logarithmic to better show the contrast. The two most
  prominent rings are fitted by Gaussian rings. Ring 1 is fitted by
  $T_{\mathrm{bright,linear}}(r)=
  3.5\,\exp(-0.5(r-77\,\mathrm{au})^2/(4.0\,\mathrm{au})^2)$, while ring 2 is fitted by
  $T_{\mathrm{bright,linear}}(r)=
  2.8\,\exp(-0.5(r-124.7\,\mathrm{au})^2/(4.4\,\mathrm{au})^2)$.}
\end{figure}

The two most prominent rings (seen as bumps in Fig.\ref{fig-obs-as209}) are
fitted by Gaussian profiles with a standard deviation width of
$4.0\,\mathrm{au}$ and $4.4\,\mathrm{au}$ respectively. This corresponds to a
full-width-at-half-maximum (FWHM) of $9.3\,\mathrm{au}$ and $10.3\,\mathrm{au}$.
The ALMA beam FWHM for these observations is roughly $4.4\,\mathrm{au}$, meaning
that the rings are 2.1 and 2.3 beams wide.

These Gaussian profiles fit the observed data remarkably well, even in the wings
of the profile (except the outer wing of the outer ring). This is somewhat
surprising, given that the rings are resolved. The Gauss profiles therefore do
not represent the beam profile, but tell something about the actual dust
distribution within the ring.

The source Elias 24 also shows a prominent high-contrast isolated ring,
albeit flanked on the outside by a plateau of emission (see {\bf Andrews
  et al.~(2018) for a detailed description of this dataset}). 

\begin{figure}
\centerline{\includegraphics[width=0.5\textwidth]{elias24_gauss_modelfit.pdf}}
\caption{\label{fig-obs-elias24}As Fig.~\ref{fig-obs-as209}, but now for the
  source Elias 24 (see {\bf Andrews et al.~(2018)} for a detailed description
  and analysis of these data). The most prominent ring is fitted by a Gaussian
  profile $T_{\mathrm{bright,linear}}(r)=
  5.3\,\exp(-0.5(r-79.9\,\mathrm{au})^2/(5.0\,\mathrm{au})^2)$.}
\end{figure}

The ring can be reasonably well fitted with a Gaussian profile with standard
deviation width of $5.0\,\mathrm{au}$ corresponding to a FWHM of
$11.7\,\mathrm{au}$. Given a beam size in band 6 of about $4.9\,\mathrm{au}$,
the ring of Elias 24 is also resolved by about 2.4 beam widths. The Gaussian fit
is, however, less convincing in the case of Elias 24 than in the case of AS 209,
so that it cannot be clearly argued that the ring has a Gaussian shape in the
case of Elias 24.

{\bf [TODO: GW Lup, HD 143006 and HD 163296]}

\section{Steady-state dust distribution in a ringlike trap}
\label{sec-steady-state-analytic-trap-model}
%
Given the nicely separated narrow rings in the above sources, we can analyze
these rings on a one-by-one basis. This is what we will do in this Section. We
will assume that the dust trapping in these rings is absolute: no dust escapes.
But increased turbulent mixing can spread the dust over a broader ring. Before
we start constructing a model, we will discuss an important feature of the
observed rings: that they appear to be radially resolved, Gaussian in shape, yet
sub-Planckian in brightness. We will present a simple analytic dust trapping
model, and show which radial brightness profiles it predicts, depending on the
dust size distribution, the turbulent strength and the gas pressure bump width.
This, when compared to the data, will allow us to draw some conclusions about
the size distribution {\bf [and perhaps more?]}.

\subsection{Resolved ring width and the brightness temperature}
If our finding is correct that the rings in our sample are all spatially
resolved in radial direction, then this poses an interesting question. In many
cases the brightness of these rings is arguably sub-Planckian. This is, for
instance, the case for the high-contrast rings of AS 209, as well as {\bf
  XXXXXXX}. For the outer two rings of AS 209 (at 77 au and 125 au,
respectively, the non-linear brightness temperatures are $3.5\,\mathrm{K}$ and
$2.8\,\mathrm{K}$, respectively. If these rings would have been {\em
  un}resolved, then this sub-Planckian emission is easily explained by the ring
not filling the beam. But because these rings are radially resolved, this
scenario should be rejected. The remaining possibilities are that the disk is
colder than we think or that the rings are optically thin. In the latter case
the spectral slope $\alpha_{\mathrm{spec}}$ at the location of these rings, once
measured, will determine if the grains that are trapped in this ring are
smaller, similar or larger than about $\lambda/2\pi$, where $\lambda$ is the
wavelength around which the slope is measured. If we use
$\lambda=0.13\,\mathrm{cm}$ and $\lambda=0.31\,\mathrm{cm}$ (representing ALMA
band 6 and 3, respectively) to determine $\alpha_{\mathrm{spec}}$ this would
mean we can distinguish between grains smaller, similar or larger than about 300
$\mu$m. A low $\alpha_{\mathrm{spec}}$ would imply grains that are substantially
larger than 300 $\mu$m, i.e.~of the order of millimeter or larger.

In several recent observations of the spectral index across ringed disks
\citep{2015ApJ...808L...3A,2018ApJ...852..122H} one clearly sees that
$\alpha_{\mathrm{spec}}$ varies across these rings, being closer to $2$ at the
ring center and substantially larger between the rings. This makes sense in
terms of the dust trapping scenario in which we expect larger grains to be
trapped more efficiently (and thus dominate the peak of the ring) than smaller
grains, because the smaller grains will be more subject to turbulent
mixing. This scenario requires a grain size distribution, so that the width of
the dust ring is smaller for the bigger grains and bigger for the smaller ones.

While this scenario is plausible and appealing, one may worry that it requires
some fine-tuning. Why would the rings be dominated by a population of grains
that are larger than 300 $\mu$m, but not too large, for else they would be so
effectively trapped that they would produce an even narrower ring? For one or
two sources that could be coincidence, but for a multitude of sources this is
less plausible. One possible explanation is that millimeter grains are indeed
the largest grains in the outer regions protoplanetary disks, because larger
grains would easily get destroyed and/or simply bounce instead of stick when
they collide, and thus not grow further \citep{2010A&A...513A..56G,
  2010A&A...513A..57Z}. These models also predict larger grains to drift quickly
inward, but that is only valid for a model without dust traps. At any rate:
perhaps millimeter grains are a natural consequence of dust growth. If so, it
then remains to be understood why the combination of turbulent strength and gas
surface density conspire to trap these millimeter grains, but not too much, for
else we might expect at least some rings to be narrower than a pressure scale
height. We are aware that, even with a sample of 20 sources, we are still in the
small-number-statistics regime, so these considerations are limited in their
strength.

In this section we will study this question using a very simple model of a dust
trap: a parameterized gaussian pressure profile. The goal is to find out if the
variations of $\alpha_{\mathrm{spec}}$ across the ring are naturally produced,
while still keeping the dust ring radially resolved, without having to fine-tune
the size distribution of the grains, the turbulence of the gas or other
parameters. We will do this in several steps, starting with analytic solutions
in Subsection \ref{sec-analytic-trap}, and proceeding with DISKLAB models in
Subsection \ref{sec-disklab-trap}.

\subsection{Analytic approximate solution of dust trapping}
\label{sec-analytic-model-of-trapping}
%
Let us consider a narrow gas ring around the star at radius $r_0$ with a
midplane pressure given by
\begin{equation}\label{eq-gaussian-pressure-bump}
p(r) = p_0 \exp\left(-\frac{(r-r_0)^2}{2w^2}\right)
\end{equation}
where $w\ll r_0$ is the parameter setting the width of this gaussian ring.
We assume that the gas is weakly turbulent with turbulent diffusion coefficient
$D$. Dust grains get trapped in this ring, and the dust will acquire a
radial density profile that is in equilibrium between the radial dust drift
pointing toward the peak of the gas pressure and radial turbulent diffusion
pointing away from that position. The radial dust drift velocity is
\citep[see e.g.][]{}:
\begin{equation}\label{eq-v-radial-drift}
  v_{\mathrm{dr}} = \frac{1}{\mathrm{St}+\mathrm{St}^{-1}}
  \left(\frac{d\ln p}{d\ln r}\right)\frac{c_s^2}{\Omega_Kr}
\end{equation}
where $c_s$ is the isothermal sound speed and the Stokes number $\mathrm{St}$ is
defined as
\begin{equation}
\mathrm{St} = \Omega_Kt_{\mathrm{stop}}
\end{equation}
where $t_{\mathrm{stop}}$ is the stopping time of the grains. The diffusion
coefficient for the dust is \citep{2007Icar..192..588Y}:
\begin{equation}
D_{\mathrm{d}} = \frac{D}{1+\mathrm{St}^2}
\end{equation}
We take $D$ to be equal to the turbulent viscosity $\nu$ divided by the
Schmidt number $\mathrm{Sc}$, which we usually set to $\mathrm{Sc}=1$.
We use the usual $\alpha$-prescription for the turbulence:
\begin{equation}
D=\frac{\nu}{\mathrm{Sc}}= \alpha_{\mathrm{turb}}\frac{c_s^2}{\mathrm{Sc}\,\Omega_K}
\end{equation}
If $D$ is sufficiently small, the dust will get concentrated into a ring with
width $w_d$ that is substantially smaller than the width of the gas ring
$w$. Also the dust will all be settled vertically into a layer that has a
vertical thickness less than the gas pressure scale height. In that case the
solution to the drift-mixing equation for the dust becomes simple, because
we can then assume everything to be constant with $r$ (within this limited
domain), except for the gas pressure gradient. We will also ignore any terms
arising from the curvature of the coordinates. The steady-state radial
dift-mixing equation for the dust then becomes, in its approximate form:
\begin{equation}
  \frac{d}{dr}
  \left(\Sigma_{\mathrm{d}}v_{\mathrm{dr}}-D_{\mathrm{d}}\frac{d\Sigma_{\mathrm{d}}}{dr}\right) = 0
\end{equation}
Integrating this equation once, with integration constant zero (which
amounts to a zero net radial flux), yields
\begin{equation}\label{eq-drift-mix-equil-eq}
  \Sigma_{\mathrm{d}}v_{\mathrm{dr}} = D_{\mathrm{d}}\frac{d\Sigma_{\mathrm{d}}}{dr}
\end{equation}
From Eqs.(\ref{eq-v-radial-drift},\ref{eq-gaussian-pressure-bump}) we can express
$v_{\mathrm{dr}}$ as
\begin{equation}
v_{\mathrm{dr}} = -\left(\frac{c_s^2}{w^2\Omega_K(\mathrm{St}+\mathrm{St}^{-1})}\right)(r-r_0)
\end{equation}
With this expression we can solve Eq.~(\ref{eq-drift-mix-equil-eq}) for
$\Sigma_{\mathrm{d}}$:
\begin{equation}\label{eq-analytic-sol-radial-trapping}
\Sigma_{\mathrm{d}}(r) = \Sigma_{\mathrm{d0}} \exp\left(-\frac{(r-r_0)^2}{2w_{\mathrm{d}}^2}\right)
\end{equation}
with 
\begin{equation}
  w_{\mathrm{d}} = w \sqrt{\frac{\Omega_KD_{\mathrm{d}}(\mathrm{St}+\mathrm{St}^{-1})}{c_s^2}}
  = w\sqrt{\frac{\alpha_{\mathrm{turb}}}{\mathrm{Sc}\,\mathrm{St}}}
\end{equation}
Note that this is only valid as long as $\alpha_{\mathrm{turb}}\ll \mathrm{Sc}\,\mathrm{St}$.
This solution is, in fact, the radial version of the vertical settling-mixing
equilibrium solutions of \citet{1995Icar..114..237D}.

Now let us introduce a grain size distribution. Since the trapping width
$w_{\mathrm{d}}$ for the different grain sizes is different, we can only
impose the grain size distribution for the total (radially integrated)
dust mass. We define the size distribution according to the following powerlaw
\begin{equation}
m(a)\frac{dN}{d\ln a} = \frac{dM}{d\ln a} \propto a^{q}
\end{equation}
where $a$ is the grain size, $m(a)$ the corresponding grain mass, $N$ the
cumulative particle number and $M$ the cumulative dust mass. The parameter $q$
is the size distribution powerlaw coefficient, and it is $q=1/2$ for the usual
MRN distribution (this corresponds to $dN/da\propto a^{-3.5}$). We also need to
define limits $a_{\mathrm{min}}$ and $a_{\mathrm{max}}$. The size distribution
is then normalized such that its integral over $d\ln(a)$ is the total dust mass
$M$. The radial surface density solution,
Eq.~(\ref{eq-analytic-sol-radial-trapping}), then becomes:
\begin{equation}\label{eq-analytic-sol-radial-trapping-sizedistr}
\frac{d\Sigma_{\mathrm{d}}(r)}{d\ln a} = \frac{1}{(2\pi)^{3/2}r_0
    w_{\mathrm{d}}(a)}\frac{dM}{d\ln a}
\exp\left(-\frac{(r-r_0)^2}{2w_{\mathrm{d}}(a)^2}\right)
\end{equation}


\subsection{Resulting profiles for a size distribution}
\label{sec-anmodel-profiles}
%
The analytic model of the dust trapping for a single grain size speaks for
itself. But how does it look for a grain size distribution? In
Fig.~\ref{fig-anmodel-distribution-sigma} we show the radial profiles for an MRN size
distribution between $a_{\mathrm{min}}=10^{-2}\,\mathrm{cm}$ and
$a_{\mathrm{min}}=1\,\mathrm{cm}$ and a total dust mass of $M=1.3\times
10^{-5}\,M_\odot$. The dust material density is taken to be
$\xi_{\mathrm{dust}}=2\,\mathrm{g}/\mathrm{cm}^3$. We sample this size
distribution with 10 grain sizes, evenly distributed in $\ln(a)$. These are the
10 curves in Fig.~\ref{fig-anmodel-distribution-sigma}. These surface densities are
defined as being the local grain size distribution integrated over the width of
the grain size bin:
$\Sigma_{\mathrm{bin}}=\int_{\mathrm{bin}}(d\Sigma_{\mathrm{dust}}/da)da$. The
stellar mass is $M_{*}=0.9\,M_\odot$ and we set $\mathrm{Sc}=1$.

{\bf MUST ALSO SPECIFY GAS DENSITY, TO GET STOKES NUMBERS. MUST SAY
(AND CHECK) THAT EPSTEIN REGIME.}

\begin{figure}
\centerline{\includegraphics[width=0.5\textwidth]{analytic_trapping_sigma_profiles.pdf}}
\caption{\label{fig-anmodel-distribution-sigma}Results of the analytic dust
  trapping model for an MRN powerlaw grain size distribution between
  $a=0.01\,\mathrm{cm}$ and $a=1\,\mathrm{cm}$. The gas pressure bump has a FWHM
  of 70 au and a peak at $r=77\,\mathrm{au}$. Shown here are the surface density
  profiles for 10 logarithmically binned grain sizes. The curve with the
  narrowest peak corresponds to a grain size of $a=1\,\mathrm{cm}$, while that
  with the widest peak (and lowest maximum value) corresponds to a grain size of
  $a=0.01\,\mathrm{cm}$. The intermediate curves correspond to $a=0.0167$,
  0.0278, 0.0464, 0.0774, 0.1291, 0.2154, 0.3593 and 0.5995 $\mathrm{cm}$
  respectively. The dashed curve shows the total surface density of the dust,
  which is, by definition of the bin-integrated surface density, the sum of all
  10 curves. For a description of the model setup and model parameters, see
  Section \ref{sec-anmodel-profiles}.}
\end{figure}

As expected, the largest grains are concentrated the most and the smallest ones
the least. The largest grains, however, have the largest peak value of their
surface density at the exact location of the gas pressure peak.

\begin{figure}
\centerline{\includegraphics[width=0.5\textwidth]{analytic_trapping_tau_profiles.pdf}}
\caption{\label{fig-anmodel-distribution-tau}As Fig.~\ref{fig-anmodel-distribution-sigma},
  but now the vertical optical depth at $\lambda=1.3\,\mathrm{mm}$ (ALMA band 6)
  is shown. The opacity model of \citet{1997MNRAS.291..121I} was used to compute these
from the size distribution.} 
\end{figure}

When it comes to the contribution to the optical depth, however, the situation
looks a bit different. Let us assume a very simple dust opacity law, which
roughly mimicks the real behavior of dust opacities: the model of
\citet{1997MNRAS.291..121I}, where we take the scattering opacity zero for
simplicity. In this simple but illustrative opacity model the cross
section of a grain with radius $a$ is the geometric cross section $\pi a^2$
for $2\pi a>\lambda$ (with $\lambda$ the wavelength), while it is reduced by
a factor $(2\pi a/\lambda)$ for $2\pi a<\lambda$. This means that for
$2\pi a>\lambda$ the opacity (that is, the cross section per unit mass)
is $\kappa=3/4\xi_{\mathrm{dust}} a$ where $\xi_{\mathrm{dust}}$ is the material
density, here taken to be $\xi_{\mathrm{dust}}=2\,\mathrm{g}/\cm^3$. For
$2\pi a<\lambda$ this is reduced by a factor $(2\pi a/\lambda)$ and
thus becomes independent of $a$.

It can be seen from Fig.~\ref{fig-anmodel-distribution-tau} that at the peak of
the trap all grain sizes contribute equally to the optical depth. That is
because, with our opacity model, the larger grains (the ones with $2\pi a >
\lambda$) have an opacity equal to their geometric cross section per unit dust
mass, which goes as $1/a$. The smallest grains (the ones with $a=0.01$ and
$a=0.167$) have $2\pi a < \lambda$, which means that their opacity remains
constant with $a$. Hence, for these two curves in
Fig.~\ref{fig-anmodel-distribution-tau} the contribution to the peak optical
depth drops below those of the other grains. Of course, with a more realistic
opacity model this can change somewhat, but the rough principle remains valid.

More important is to notice that the relative contributions of the grain sizes
to the peak optical depth may change if a different grain size distribution is
chosen. Here we chose the MRN size distribution ($q=1/2$), but that was a rather
arbitrary choice. Given the complexity of the dust growth processes in
protoplanetary disks, it is not at all said that the MRN distribution should be
considered a preferred one. The reader may experiment with other values of
$q$ or entirely different grain size distribution shapes altogether.



\subsection{Effect of the degree of trapping on the appearance of the ring}
\label{sec-anmodel-degree-trapping-alpha-series}
%
With this simple analytic dust trapping model we can investigate how the degree
of trapping affects the appearance of the dust ring. For a model with a single
grain size this is easy to predict: if the turbulence is weak enough that the
gaussian profile of $\Sigma_{\mathrm{d}}(r)$ is narrower than the ALMA beam,
then we observe a gaussian profile with the width of the beam. Even if the dust
in this narrow ring would be optically thick, the observed ring would have
sub-Planckian brightness due to beam-dilution. For stronger turbulence,
$\Sigma_{\mathrm{d}}(r)$ is spread out wider, and we observe a gaussian profile
with the width of the dust profile itself. As in the case of the strongly
trapped dust, also the wide rust ring has a Gaussian profile, except that if it
is optically thick, the observed profile is flattened at the top due to
saturation to the Planck function.

The situation gets more complex if we have a grain size distribution. For a
given turbulence strength $\alpha_{\mathrm{turb}}$ some grains may be
concentrated into a ring that is narrower than the ALMA beam, while others may
be more spread out. The strongly concentrated grain population produces a cusp
in the brightness, which will be partly smeared out by the beam convolution,
while the weakly concentrated grain population produces a broad underlying
brightness bump.

To see how this looks in practice we set up a realization of our analytic
model. The peak of the pressure is put at $r_{0}=77\,\mathrm{au}$ (corresponding
to the inner of the two main rings of AS 209), and the width is chosen to be
$w=30\,\mathrm{au}$ (corresponding to a FWHM of 70 au).  Given that the ALMA
beam in band 6 has a FWHM of about 35 milliarcseconds, corresponding to 4.4 au
for the distance to AS 209, the gas bump is much wider than the beam. The two
main dust rings of AS 209 are about 3 beams wide, so that for our wide gas bump
model the dust is strongly concentrated. This requires a low turbulent
$\alpha_{\mathrm{turb}}$ or large grains. As in Section
\ref{sec-anmodel-profiles} we choose a grain size distribution between
$a_{\mathrm{min}}=10^{-2}\,\mathrm{cm}$ and $a_{\mathrm{min}}=1\,\mathrm{cm}$,
following the usual MRN powerlaw of $q=1/2$ and a total dust mass of
$M=1.3\times 10^{-5}\,M_\odot$. The dust material density is taken to be
$\xi_{\mathrm{dust}}=2\,\mathrm{g}/\mathrm{cm}^3$. We sample this size
distribution with 10 grain sizes, evenly distributed in $\ln(a)$. For the disk
temperature we choose a simple powerlaw $T(r)=200\,(r/\mathrm{au})^{-1/2}$. The
stellar mass is $M_{*}=0.9\,M_\odot$, and the source distance is taken to be
$d=126\,\mathrm{pc}$. All these setting are taken to represent the inner of the
two main high-contrast rings of the source AS 209. We set $\mathrm{Sc}=1$.

\begin{figure}
\centerline{\includegraphics[width=0.5\textwidth]{analytic_trapping_series_alpha.pdf}}
\caption{\label{fig-anmodel-series-alpha}Results of the analytic dust trapping
  model for various turbulent strengths, for an MRN powerlaw grain size
  distribution between $a=0.01\,\mathrm{cm}$ and $a=1\,\mathrm{cm}$. The gas
  pressure bump has a FWHM of 70 au and a peak at $r=77\,\mathrm{au}$. Shown
  here is the predicted brightness temperature at a wavelength
  $\lambda=1.3\,\mathrm{mm}$ before (dashed) and after (solid) convolution with
  the ALMA beam, assuming a source distance of $d=126\,\mathrm{pc}$. For a
  description of the model setup and model parameters, see Section
  \ref{sec-anmodel-degree-trapping-alpha-series}.}
\end{figure}

The results for several values of $\alpha_{\mathrm{turb}}$ are shown in
Fig.~\ref{fig-anmodel-series-alpha}. In the top panel, representing weak
turbulence, most of the dust is trapped into a ring more narrow than the
beam. A clear cusp can be seen in the un-convolved brightness profile,
which is due to the strongly trapped large grains of the size distribution.
However, the convolution washes out this cusp for the most part. This
situation is equivalent to the simple case of a single dust species being
trapped to a narrow unresolved ring. The only difference is that for a single
dust size the ring is either optically thick or optically thin, while in the
multi-size case this can be a mix. That does not make a difference for the
observed brightness profile, which is a Gaussian either way.

For the case of the strongest turbulence, the bottom panel in
Fig.~\ref{fig-anmodel-series-alpha}, the cusp is still seen in the un-convolved
profile, but now the convolution does not wash it out completely. The resulting
convolved brightness profile is clearly non-Gaussian as a result. It has a
quasi-Gaussian cusp with broad wings. When seen with ALMA such a cuspy profile
(narrow at the peak, broad in the wings) should be recognizable.

These findings show that if the turbulent strength is weak, the predicted dust
ring width equals the ALMA beam width, and the brightness shape should have the
shape of the beam profile (here assumed to be Gaussian). For stronger
turbulence, the dust ring becomes indeed broader than the ALMA beam, but then
the shape of the observed brightness profile may teach us something about the
grain size distribution. Roughly speaking, if the largest grains of the size
distribution still get trapped in a narrow ring, this shows up as a cusp in the
brightness profile. If the smallest grains of the grain size distribution are
too small, the predicted brightness profile would be too broad compared to the
observed ring width. In other words: if the observed brightness profile of a
observed ring in given source is clearly radially resolved, and if it has a
Gaussian shape, then the most likely fit to our analytic model is that of
a narrow size distribution (quasi a single grain size) with the right turbulent
strength to fit the observed ring width. A broader size distribution, with the
appropriately chosen turbulent strength, would yield a cuspy profile, not a
Gaussian one.

The two high-contrast rings of AS 209 both appear to fit a Gaussian profile with
2.5 times the width of the ALMA beam (Fig.~\ref{fig-as209-rings-gauss-fit}). In
both cases it is therefore likely, according to this analysis, that the dust
that is trapped in these rings has a narrow size distribution instead of a broad
MRN powerlaw distribution.

\subsection{Spectral slope variations across the ring}
\label{sec-anmodel-slope-variations}
%





\subsection{Model degeneracies}
\label{sec-anmodel-degeneracies}
%
When trying to infer the effectiveness of dust trapping within the rings with
the analytic model of Section \ref{sec-analytic-model-of-trapping}, one notices
that there are degeneracies in the possible combinations of parameters. Most
obvious is the degeneracy between grain size and turbulence: smaller grains
and stronger turbulence may yield the same dust trapping width. But also the
width of the gas pressure bump can be compensated: a narrower bump but stronger
turbulence (or smaller grains) will yield the same ring width.

{\bf [CONTINUE THIS; SOME DEGENERACIES FOLLOW DIRECTLY FROM THE EQUATIONS.]}


\subsection{Model fits to the ALMA data}
\label{sec-anmodel-fits-to-data}
%
We now apply this model to the isolated rings in our ALMA sample (Section
\ref{sec-data}). Given the degeneracies of the model in many of the parameters,
we perform the fitting procedure in three parameters, while fixing the others
to a particular choice of values. 



{\bf [CONTINUE THIS; SHOW THE FIT FIGURES; SHOW THE MCMC CORNER PLOTS]}







\subsubsection{Could the rings be optically thick at the center?}
\label{sec-anmodel-fits-optically-thick}
%
All rings in our sample show sub-Planckian brightness, at least for the disk
temperature assumed in our model. We have calculated the disk midplane temperature
using a simple ``flaring angle recipe''. If the disks are, however, substantially
colder (by up to a factor of 2.5 to 3) then the peak flux of the rings might be
consistent with saturated emission, i.e.\ emission expected from an optically thick
disk. So the question is: how sure are we that the rings are sub-Planckian? 

To reduce the midplane temperature by a factor of $2$ the irradiation flux must
be reduced roughly by a factor of $2^4=16$. Within the flaring angle recipe this
would lead to unrealistically low flaring angle. Presently we assumed $\varphi=0.02$,
so that it would have to reduce to $\varphi=0.0013$, which is so low that we can no
longer trust the flaring angle recipe. However, maybe the outer disk lies in the
shadow of the inner disk. To see whether this can reduce the temperature to low
enough values, we perform a proper radiative transfer calculation. The results are
shown in Appendix ({\bf XXXX}). It shows that {\bf XXXXX}.

There is another issue, however. If the emission saturates to the optically thick
brightness, i.e.\ to the Planck function, then the radial profile of the ring should
be flat-topped. This is clearly not what we see in the rings in our sample, with
the possible exception of the ring of GW Lup which shows a very slight hint of that.
In general, however, it seems that the simple picture of collecting so much dust into
the ring that the ring becomes optically thick does not seem to explain the observed
radial brightness profiles. On the other hand, optical thickness can also occur
on smaller scales, as we will discuss in Section \ref{sec-si-clump}.

{\bf [TO BE DONE: So far the CMB or cloud background is not included in the model.
    This has to be improved, because we are working at temperatures close to the
    CMB. It might reduce the sub-Planckianness, I guess.]}

{\bf [Question to the experts: The CMB and/or cloud background is automatically
filtered out of the interferometric observations, I presume?]}


\section{Streaming instability, clumping, and the spectral index}
\label{sec-si-clump}
%
Whenever the dust trapping becomes so effective that the volume density of the
dust at the midplane of the disk exceeds the gas volume density, the streaming
instability may set in \citep{2005ApJ...620..459Y}. This produces turbulence and
leads to clumping \citep{2007ApJ...662..627J, 2008A&A...479..883L,
  2010ApJ...722.1437B}. If conditions are right, some of these clumps may
gravitationally collapse and form planetesimals \citep{2007Natur.448.1022J,
  2017ApJ...847L..12S, 2017A&A...597A..69S}. It is believed that this mechanism
is the trigger that starts the process of the formation of planets
\citep[see][for a review]{2014prpl.conf..547J}. If one could find observational
evidence of this dust clumping occurring in protoplanetary disks, this would
be a major step forward for our understanding of planet formation.

The clumps that are formed in simulations of the streaming instability are,
however, typically substantially smaller than a pressure scale height of the
disk \citep{2007ApJ...662..627J, 2013MNRAS.434.1460K}. This makes it very hard
to spatially resolve these structures, even with ALMA at its highest resolution.

It might, however, be possible to find indications of particle clumping in an
indirect way, if it leads to the formation of optically thick clouds of
particles surrounded by optically thin regions. Even if we cannot spatially
resolve these clouds, the spatially averaged thermal emission from such a clumpy
medium has a different spectral slope than that of a homogeneous distribution of
dust. If all the dust is concentrated in a multitude of small and very optically
thick clumps, then the brightness of this unresolved clumpy medium would be the
Planck function times the covering fraction $C$ of the clumps:
\begin{equation}\label{eq-bright-clumpy}
I_\nu^{\mathrm{obs}} = B_\nu(T_{\mathrm{disk}})\,C
\end{equation}
where the limit $C\rightarrow 0$ means concentrating all the dust in infinitely
dense clumps so that the covering fraction is zero, and $C\rightarrow 1$ means
having so many clumps that they multiply overlap, when seen in projection from
above. In contrast, if all this dust is homogeneously distributed, then the
brightness is
\begin{equation}\label{eq-bright-homogeneous}
I_\nu^{\mathrm{obs}} = B_\nu(T_{\mathrm{disk}})\,(1-e^{-\tau_\nu})
\end{equation}
with $\tau_\nu$ being the vertical optical depth of the dust in the disk.
Reality will likely lie in between these two extreme scenarios, with
optically thin dust interdispersed between optically thick clouds.

There are at least two ways by which one could use Eqs.~(\ref{eq-bright-clumpy},
\ref{eq-bright-homogeneous}) to search for hints of clumpiness. The most direct
way is to make use of the different dependencies of these two equations on
wavelength $\lambda=c/\nu$. In the case of the clumpy medium with perfectly
optically thick clumps (Eq.~\ref{eq-bright-clumpy}) the brightness follows the
Planck curve, whereas for the homogeneous medium
(Eq.~\ref{eq-bright-homogeneous}), if it is optically thin
(i.e.~$1-e^{-\tau_\nu}\simeq \tau_\nu$), it follows the Planck curve times the
opacity law. If we are in the Rayleigh-Jeans domain of the Planck curve, then
these two limits are usually referred to as spectral slope
$\alpha_{\mathrm{spec}}=2$ and $\alpha_{\mathrm{spec}}=2+\beta$, respectively,
where $\beta$ is defined by $\kappa_\nu\propto \nu^\beta$. At ALMA band 6, for
the low temperatures we expect in these disks at large radii, we are no longer
strictly in the Rayleigh-Jeans domain, so the spectral slope analysis becomes
temperature-dependent, which complicates matters a bit, but the principle
remains the same. If the streaming instability is operating in the rings of our
sources, and if the numbers work out (see below), then we expect to see
$\alpha_{\mathrm{spec}}\simeq 2$ within the rings, and
$\alpha_{\mathrm{spec}}>2$ between the rings.

There is an important caveat to this idea: the grain size and the corresponding
opacity slope may mimick the same spectral signatures as the clumping. This is,
in fact, the same degeneracy of the interpretation of the millimeter spectral
slope as usual \citep[e.g.][]{2003A&A...403..323T}, just on a much smaller
spatial scale: a scale that remains unresolved even by ALMA.  Nevertheless it is
worthwhile to investigate if we can find clues that point to clumpiness.

In the scope of this paper we will only do a very simple analysis: we will
verify if, under the most benign conditions, the above mentioned optically
thick/thin clumpy medium is at all possible. This is not granted, because
these rings are several tens of au large, and therefore it requires a lot of
mass to make dust clumps optically thick at 1.3 millimeter wavelength.
Furthermore, these dust clouds cannot have a density in excess of the
Roche density, because then they would immediately gravitationally collapse.
Therefore, we cannot make the clumps arbitrary dense and compact.

So let us construct the most optimistic model of a dust cloud. The highest
possible density of a cloud is the Roche density, defined by
\begin{equation}
\rho_{\mathrm{Roche}}(r) = \frac{9}{4\pi}\frac{M_{*}}{r^3}
\end{equation}
where $M_{*}$ is the stellar mass and $r$ is the distance to the star. We take
$\rho_{\mathrm{cloud}}=\rho_{\mathrm{Roche}}$. The simplest opacity model for a
dust grain is that of \cite{1997MNRAS.291..121I}, where we omit the scattering
part:
\begin{equation}
\kappa(a) = \frac{3}{4}\frac{1}{\xi\,a}\left\{\begin{matrix}
1 & \hbox{for} & \lambda<2\pi a \\
2\pi a/\lambda & \hbox{for} & \lambda>2\pi a \\
\end{matrix}\right.
\end{equation}
where $a$ is the grain radius and $\xi$ is the material density of the dust
grain ($\xi=3.6$ for amorphous olivine, $\xi=2.0$ for silica {\bf CHECK}). For a
given wavelength $\lambda$ (in our case $\lambda=0.13\,\mathrm{cm}$ for ALMA
band 6), the most optimistic opacity is therefore $\kappa=(3\pi/2)/\xi\lambda$,
for all $a\le \lambda/2\pi$. For ALMA band 6 and $\xi=2\,\mathrm{g/cm}^3$ this
is $\kappa=18.1\,\mathrm{cm}^2/\mathrm{g}$. The largest grain size to still
have this opacity is therefore $a=\lambda/2\pi$, which for ALMA band 6 is
$a=0.02\,\mathrm{cm}$.  We make a spherical cloud of this dust with radius
$R$. The optical depth through the cloud center is then
\begin{equation}
\tau = 2R\rho_{\mathrm{cloud}}\kappa
\end{equation}
To obtain the effect of an optically thick/thin clumpy medium, these clouds
have to be sufficiently optically thick, not just marginally. So let us demand
a minimal optical depth of $\tau=4$. This gives a minimal cloud radius of
\begin{equation}
  R\gtrsim \frac{2}{\rho_{\mathrm{Roche}}\kappa} = 0.11\frac{1}{\rho_{\mathrm{Roche}}}
  = 0.15\frac{r^3}{M_{*}}
\end{equation}
in CGS units (for ALMA band 6). Note that this scales linearly with
$\rho_{\mathrm{cloud}}/\rho_{\mathrm{Roche}}$, so that if we choose a much less
``critical'' cloud density of, say, $\rho_{\mathrm{cloud}}=0.1\,\rho_{\mathrm{Roche}}$,
then the cloud should become 10$\times$ larger and 100$\times$ more massive.

If we want these clouds to be part of an unresolved clumpy medium, they should
be substantially smaller than the pressure scale height of the disk, and
substantially smaller than the ALMA band 6 resolution. Let us take the example
of the source AS 209, with $M_{*}=0.9\,M_{\odot}$, and its two high-contrast
rings at 74 au and 120 au, respectively. This gives minimal cloud
radii of $R\gtrsim 4\times 10^{-3}\,\mathrm{au}$ and $R\gtrsim 22\times
10^{-3}\,\mathrm{au}$, respectively. Clouds of size similar to these minimal
values are indeed much smaller than the beam width, and are therefore
unresolved.  This does not mean that there cannot be any smaller
clumps/clouds. It only means that any smaller clouds would be optically thin,
and we would thus not see any effect of these clumps on the unresolved
brightness: these clouds would have a spectral slope equal to that of an
optically thin non-clumpy model. At any rate, this estimate shows that, at least
in principle, it is possible that the rings we see in our data consist, in fact,
of numerous unresolved optically thick clouds interdispersed with optically thin
dust (or without any dust in between). Given that such an unresolved, but
optically thick, set of clumps has a lower spectral slope
$\alpha_{\mathrm{spec}}$ than an optically thin dusty medium, an analysis of the
measured spectral slope variations over the radial profile of the rings {\em
  may} give us some clues as to whether clumpiness occurs or not in these
rings, even if we cannot resolve the clumps themselves. We would expect that
within the ring the spectral slope is closer to $\alpha_{\mathrm{spec}}\simeq 2$
(modulo deviations from Rayleigh-Jeans) than in the inter-ring region.

However, the grain size segregation mechanism, that larger grains are more
concentrated in the ring than smaller grains, also tends to produce this
signature. This is particularly the case if the big grains (which are closest to
the ring peak) are larger than $\lambda/2\pi$ while the small grains are smaller
than $\lambda/2\pi$. To distinguish between the roles of size seggregation and
clumping, we need an independent way to probe the grain size. Perhaps the effects
of self-scattering induced polarization \citep{2015ApJ...809...78K} may provide
such an avenue. 

Even if we can distinguish between size-sorting and clumping, we still have to
clarify whether the required total dust mass for the clumping scenario is
reasonable, and whether the required 'filling factor' is consistent with
simulations of the streaming instability. Both questions are related. Let us go
back to the two prominent isolated rings of AS 209. The peak linear brightness
temperature is 3.5 $\mathrm{K}$ for the inner one and 2.7 $\mathrm{K}$ for the
outer one. In our model the disk midplane temperature is 13 $\mathrm{K}$ for the
inner and 10 $\mathrm{K}$ for the outer ring, which translates (for
$\lambda=0.13\,\mathrm{cm}$) into a linear brightness temperature of
8 $\mathrm{K}$ and 5.6 $\mathrm{K}$, respectively. If the disk were optically
thick, we would thus expect to observe a linear brightness temperature of
8 $\mathrm{K}$ and 5.6 $\mathrm{K}$, which are only about 2 times higher. 

{\bf [TODO: Handle the CMB and cloud background, which might reduce the
sub-Planckianness we observe.]}

This means that the filling factor of these clouds, as projected vertically
toward the observer, must be quite high, of the order of 50\%. {\bf [At this
    point we can calculate the minimum mass in dust needed for this]}

In principle the disk could be warmer than we assume, leading to a larger
saturation brightness, and thus a lower required filling factor. But given that
a 2$\times$ higher disk temperature requires a 16$\times$ higher irradiation
flux, we expect at most a factor of 2 uncertainty in the required filling
factor.

Checking whether this filling factor is consistent with the expectation of
simulations of the streaming instability is not easy. Such data are not
published for the models describing the streaming instability so far.  We can
only do a ``by eye'' check of the figures in those papers ({\bf give a few
  example papers here}). Such an exercise gives us the impression that the
typical fraction of the simulation box that is covered by the dense cluster
regions, as projected vertically, is typically substantially less than 50\%. If
this impression is true, then the ring emission seen in our sample cannot be
explained by a set of spatially unresolved optically thick clouds. That does not
imply, however, that the streaming instability does not operate in these
rings. It only implies that if it does, it will be hidden below an optically
thick shroud of more extended dust.





\subsection{Under which conditions should the streaming instability set in?}
While in Section \ref{sec-si-clump} we investigated if we can recognize the
ongoing streaming instability in our data if it is taking place in the disk, in
this Section we will compute, with our analytic dust trapping model, under which
condition we would {\em expect} that the streaming instability will operate in
the first place. This depends on the total dust mass locked in the trap, the
grain sizes and the turbulent strength. To make analytic estimates it is easiest
to assume a single grain size, instead of a grain size distribution.

{\bf FINISH THIS SECTION}






\section{Discussion}


\subsection{Gap depth estimates from 2-D hydrodynamic simulations}

{\bf Dong \& Fung (2017)} presented a systematic study of the gap
opening profiles for given planetary masses, and analyzed the
rings seen in scattered light of 5 sources (HD 97048, TW Hydra,
HD 169142, LkCa15 and RX J1615).


\subsection{Condition for the streaming instability}
In the literature it is often mentioned that the streaming instability requires
a dust-to-gas surface density ratio of
$\Sigma_{\mathrm{d}}/\Sigma_{\mathrm{g}}\gtrsim 0.02$ or higher to operate {\bf
  XXXXXXXX}. This can, however, not be directly compared to our models, because
this value of $0.02$ was found for models without any pre-determined
turbulence. The turbulence in those models was induced by the streaming
instability itself or, if the streaming instability does not operate, by the
Kelvin-Helmholtz instability ({\bf XXXX}). In the analytic model of this
Section, on the other hand, we set the turbulence strength by hand, by setting
$\alpha_{\mathrm{turb}}$ to some value. In essence, we assume that there is
another source of turbulence, such as the magnetorotational instability ({\bf
  XXXX}) or the vertical shear instability ({\bf XXXX}), that determines the
mixing of the dust in the disk.

According to \citet{2005ApJ...620..459Y} the true criterion for the onset of the
streaming instability is the ratio of dust and gas {\em volume} densities
$\rho_{\mathrm{d}}/\rho_{\mathrm{g}}\gtrsim 1$. For a given surface density
ratio $\Sigma_{\mathrm{d}}/\Sigma_{\mathrm{g}}$, the midplane volume density
ratio, for a single grain species with midplane Stokes number $\mathrm{St}\ll
1$, depends on the turbulent strength as
\begin{equation}\label{eq-dtg-sig-vs-rho}
  \frac{\rho_{\mathrm{d}}}{\rho_{\mathrm{g}}}\simeq
  \left(1+\frac{\mathrm{St}}{\alpha_{\mathrm{turb}}}\right)
  \frac{\Sigma_{\mathrm{d}}}{\Sigma_{\mathrm{g}}}
\end{equation}
The criterion of $\Sigma_{\mathrm{d}}/\Sigma_{\mathrm{g}}\gtrsim 0.02$ mentioned
in the literature thus relates to the criterion
$\rho_{\mathrm{d}}/\rho_{\mathrm{g}}\gtrsim 1$ via the turbulent strength and
the Stokes number. Given that we do not compute the turbulent strength
but prescribe it, we should rely on the more fundamental volume density
criterion of \citet{2005ApJ...620..459Y} to assess whether the dust in our
model triggers the streaming instability or not.

In our analytic setup of Section \ref{sec-anmodel-degree-trapping-alpha-series},
the dust ring is about 5 times narrower than the gas bump. Assuming that
initially the dust-to-gas ratio was $0.01$, this implies a dust-to-gas surface
density ratio at the peak of the ring of $0.05$, which is in excess of the usual
$0.02$ value. But as long as $\mathrm{St}\lesssim 19\alpha_{\mathrm{turb}}$,
Eq.~(\ref{eq-dtg-sig-vs-rho}) shows that the midplane volume density ratio still
stays below unity and no streaming instability sets in.


\subsection{Dust trapping in the horseshoe region and in resonances}
The 1-D approach of the {\sf DISKLAB} models make it impossible to treat the
trapping of particles in the horseshoe region. The low-$\alpha$ models of {\bf
  Liu et al.~2018} show that this may be a significant effect. However, this
tends to produce non-axisymmetric rings: there will clearly be a dearth of dust
near the location of the planet, and a slight enhancement of dust near the L4
and L5 Lagrange points. Given that the rings we see so far in the multi-ringed
protoplanetary disks are angularly symmetric, this appears to be ruled out.
That means that the 1-D approach is justified, but we do have to ensure that we
are not in the regime where such non-axisymmetric horseshoe-trapping is
important. Generally the horseshoe region tends to gradually deplete over time,
but the time scale depends on the turbulent viscosity, with low-$\alpha$ disks
depleting slower. 

Another issue is the potential for large particles to get trapped in a
resonance of a planet instead of in a classical dust trap {\bf REFERENCE;
the Dong double-gap is different, though}.
This requires high Stokes number, generally $\mathrm{St}\gg 1$ ({\bf is
  this true?}). Given that grain growth models tend to keep grains below
this limit ({\bf see e.g.~Birnstiel et al.}), this is less likely to play
a role.

\subsection{Multiple gaps with a single planet}
{\bf Dong, Li \& Li (2017)} and 
{\bf Bae, Zhu \& Hartmann (2017)} show that for very low-$\alpha$ disks
a planet can open multiple gaps in the disk rather than just a single one.
The contrast of these rings and gaps in the gas can be large (factor of 2; see
Bae et al.),
and thus clearly constitute dust traps. Dong et al.\ show that these gaps
can even be observable for very low mass planets.

{\bf [TODO: We can do a simple model of dust trapping for their case.]}

{\bf Fung \& Chiang (2017)} have shown that low mass planets in nearly inviscid
disks can avoid migration and can be responsible for disk accretion in spite
of the low viscosity.

\subsection{Interpretation of the millimeter flux in terms of grain size}
A common rule of thumb for interpreting millimeter and sub-millimeter
fluxes and intensities from thermal dust emission is that you ``observe
the grain size equal to the wavelength you observe at''. This rule is
based on the tendency of the dust opacity derived from a Mie calculation
to peak at wavelength around $\lambda\sim 2\pi a$, with $a$ the grain
radius. For $\lambda \ll 2\pi a$ the opacity becomes nearly constant
while $\lambda \gg 2\pi a$ it drops. This effect is even stronger for
the scattering opacity. The validity of this rule of thumb relies,
however, on the grain size distribution itself. For the simple
{\bf Ivezic et al.~(1997)} opacity law, it is only valid for
grain size distribution powerlaws obeying {\bf XXXXXXXXXXX}.
{\bf [TODO: Maybe here do an experiment with full opacity laws.]}




\begin{acknowledgements}
  The authors acknowledge support
  by the High Performance and Cloud Computing Group at the Zentrum f\"ur
  Datenverarbeitung of the University of T\"ubingen, the state of
  Baden-W\"urttemberg through bwHPC and the German Research Foundation (DFG)
  through grant no INST 37/935-1 FUGG. This research was initiated at the
  ``Stars, Planets and Galaxies'' meeting at the Harnackhaus in Berlin,
  April 2018. Part of this work was also funded by the DFG Forschergruppe
  FOR 2634 ``Planet Formation Witnesses and Probes: Transition Disks''.
\end{acknowledgements}


\begingroup
\bibliographystyle{aa}
\bibliography{ms}
\endgroup

\end{document}
