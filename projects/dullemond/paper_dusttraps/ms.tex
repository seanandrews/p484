%\documentclass[referee]{aa}
\documentclass{aa}
%\documentclass{mn2e}
\usepackage[usenames,dvipsnames]{xcolor}
\usepackage{natbib}
\usepackage{xspace}
\usepackage{amssymb}
\usepackage{amsmath}
\usepackage{graphicx}
\usepackage{hyperref}
\def\remark#1{{{\bf remark:} \bf #1}}
\def\action#1{{\bf #1}}
\def\putin#1{{\it #1}}
\def\revised#1{{\bf #1}}
\newcommand{\todo}[1]{\textcolor{red}{\bf #1}}
%
\def\aap{A\&A}
\def\apss{A\&SS}
\def\pasp{Proc.Astr.Soc.Pacific}
\def\pasj{Publ.Astr.Soc.Japan}
\def\araa{AnnRevA\& A}
\def\mnras{MNRAS}
\def\icarus{Icarus}
\def\nat{Nature}
\def\apj{ApJ}
\def\aj{AJ}
\def\apjl{ApJL}
\def\apjs{ApJS}
\def\jqsrt{JQSRT}
\def\max{\mathrm{max}}
\def\erg{\hbox{erg}}
\def\sec{\hbox{s}}
\def\Hz{\hbox{Hz}}
\def\gram{\hbox{g}}
\def\cm{\hbox{cm}}
\def\km{\hbox{km}}
\def\kel{\hbox{K}}
\def\KeV{\hbox{KeV}}
\def\MeV{\hbox{MeV}}
\def\eV{\hbox{eV}}
\def\yr{\hbox{yr}}
\def\AU{\hbox{AU}}
\def\ster{\hbox{ster}}
%\def\in{\mathrm{in}}
\def\out{\mathrm{out}}
\def\eff{\mathrm{eff}}
\def\rad{\mathrm{rad}}
\def\vrt{\mathrm{vert}}
\def\vset{\ensuremath{v_\mathrm{sett}}\xspace}
\def\fact{\ensuremath{\xi}\xspace}
\def\cs{\ensuremath{c_\mathrm{s}}\xspace}
\def\siggas{\ensuremath{\sigma_\mathrm{g}}\xspace}
\def\sigcoag{\ensuremath{\sigma_\mathrm{c}}\xspace}
\def\Tgas{\ensuremath{T_\mathrm{gas}}\xspace}
%\def\rhogas{\ensuremath{\rho_\mathrm{gas}}\xspace}
\def\rhogas{\ensuremath{\rho}\xspace}
\def\rhodust{\ensuremath{\rho_\mathrm{d}}\xspace}
\def\mugas{\ensuremath{\mu_\mathrm{gas}}\xspace}
\def\Hp{\ensuremath{H_\mathrm{p}}\xspace}
\def\Hs{\ensuremath{H_\mathrm{s}}\xspace}
\def\Omegak{\ensuremath{\Omega_\mathrm{K}}\xspace}
\def\Mstar{\ensuremath{M_\star}\xspace}
\def\Sc{\ensuremath{\mathrm{Sc}}\xspace}
\def\St{\ensuremath{\mathrm{St}}\xspace}
\def\Pt{\ensuremath{\mathrm{Pt}}\xspace}
\def\comma{\,,}
\def\fullstop{\,.}
%
\def\submitted{submitted\xspace}
\def\inprep{in prep.\xspace}
\def\paperdsharpandrews{Andrews et al.\ ({\bf 2018})}
\def\paperdsharphuangrings{Huang et al.\ ({\bf 2018})}
\def\paperdsharpisella{Isella et al.\ ({\bf 2018})}
\def\paperdsharpguzmann{Guzmann et al.\ ({\bf 2018})}
\def\paperdsharpbirnstiel{Birnstiel et al.\ ({\bf 2018})}
%
\begin{document}
%\thesaurus{02.01.2,08.03.4,08.06.2,08.16.5,13.09.6}
\title{Dust trapping in thin-ringed protoplanetary disks}  \titlerunning{Thin-ringed disks}
\authorrunning{Dullemond et al.}
\author{C.P.~Dullemond$^{1}$, T.~Birnstiel$^{2}$, N.~Troncoso$^{3}$, L.~Perez$^{3}$, S.~Andrews$^{4}$, A.~Isella$^{5}$, J.~Huang$^{4}$, Z.~Zhu$^{6}$, V.~Guzm\'an$^{7}$, M.~Benisty$^{8}$, J.~Carpenter$^{9}$}
\institute{
  (1) Zentrum f\"ur Astronomie, Heidelberg University, Albert Ueberle Str.~2, 69120 Heidelberg, Germany\\
  (2) {\bf XXXX}
  (3) {\bf XXXX}
  (4) {\bf XXXX}
  (5) {\bf XXXX}
  (6) {\bf XXXX}
} \date{\today}

\abstract{A large fraction of the protoplanetary disks observed with the ALMA
  Large Programme DSHARP display multiple well-defined and nearly perfectly
  circular rings in the continuum, in many cases with substantial peak-to-valley
  contrast. Several of these rings are very narrow in radial extent.
  In this paper we analyze these results using the assumption that
  these dust rings are caused by dust trapping in radial pressure bumps. We
  model this process in a 1-D axisymmetric way, initially with a simple analytic
  model, then with a more detailed numerical model. We find that .....
%  including time-dependent viscous disk evolution coupled to time-dependent
%  radial dust drift. We perform model calculations without and with dust growth,
%  the latter with a simplified dust growth recipe. We test several scenarios
%  for the origin of the pressure bumps, including planetary gaps and dead zones.
{\bf CONCLUSIONS}
}

\maketitle



\begin{keywords}
accretion, accretion disks -- circumstellar matter 
-- stars: formation, pre-main-sequence -- infrared: stars 
\end{keywords}

\section{Introduction}
The concept of dust trapping in local pressure maxima has become a central theme
in studies of planet formation and protoplanetary disk evolution, because it might
provide an elegant solution to several problems in these fields of
study. Theories of planet formation are plagued by the ``radial drift barrier'':
the problem that, as dust aggregates grow by coagulation, they tend to radially
drift toward the star before they reach planetsimal size
\citep[e.g.][]{2010A&A...513A..79B}. A natural solution to this problem could be
the trapping of dust particles in local pressure maxima
\citep{1972fpp..conf..211W, 2007ApJ...664L..55K, 1995A&A...295L...1B,
  1997Icar..128..213K}. Not only does this process prevent excessive radial
drift of dust particles, it also tends to concentrate the dust into small
volumes and high dust-to-gas ratios, which is beneficial to planet formation.
From an observational perspective, the radial drift
problem manifests itself by the presence of large grains in the outer regions of
protoplanetary disks \citep{2003A&A...403..323T, 2009ApJ...700.1502A}, which
appears to be in conflict with theoretical predictions
\citep{2007A&A...469.1169B}. One possible solution to this observational
conundrum could be that the disks are much more massive in the gas than
previously suspected, leading to a higher gas friction for millimeter grains and
thus longer drift time scales \citep{2017ApJ...840...93P}. Another explanation is
to invoke dust traps. The most striking observational evidence for
dust trapping seems to come from large transitional disks, which feature giant
dust rings, sometimes lopsided, in which large quantities of dust appears to be
concentrated \citep{2013Natur.493..191C, 2013Sci...340.1199V}. These observations
appear to be well explained by the dust trapping scenario
\citet{2012A&A...545A..81P}. But these transitional disks seem to be rather
violent environments, possibly with strong warps \citep{2015ApJ...798L..44M,
  2017A&A...597A..42B} and companion-induced spirals
\citep{2016ApJ...816L..12D}. For more ``normal'' protoplanetary disk the dust
traps would have to be more subtle. \cite{2012A&A...538A.114P} explored the
possibility that the disk contains many axisymmetric local pressure maxima, and
calculated how the dust drift and growth would behave under such conditions. It
was found that, if the pressure bumps are strong enough, the dust trapping can
keep a sufficient fraction of the dust mass at large distances from the star to
explain the observed dust millimeter flux. It would leave, however, a detectable
pattern of rings that should be discernable with ALMA observations.  Since the
multi-ringed disk observation of HL Tau \citep{2015ApJ...808L...3A} a number of
such multi-ringed disks have been detected \citep{2016ApJ...820L..40A,
  2016PhRvL.117y1101I, 2017ApJ...851L..23C, 2017A&A...600A..72F,
  2018A&A...610A..24F}. It is therefore very tempting to see also these
multi-ringed disks as evidence for dust trapping, and as an explanation for
the retention of dust in the outer regions of protoplanetary disks.

The data from the ALMA Large Programme DSHARP ({\bf REFERENCES}) offers an exciting new
opportunity to put this concept to the test, and to put constraints on the
physics of dust trapping in axisymmetric pressure maxima. This is an opportunity
which we explore in this paper.

If we assume that the rings seen in our data are caused by dust trapping, the
question arises: what constraints do the data impose on the physics of dust
trapping? To explore this we start by studying the rings individually, assuming
that the dust does not escape from the ring. This makes it possible to look for
a steady-state dust trapping solution in which the radial drift forces (that push
the dust to the pressure peak) are balanced by turbulent mixing (that tends to
smear out the dust away from the pressure peak). In Section
\ref{sec-steady-state-analytic-trap-model} we will construct a very simplified
analytic dust trapping model, and confront this with the most well-isolated
rings from our sample. {\bf Describe the next sections.} But before we do the
modeling, we review, in Section \ref{sec-data}, the key aspects of the data from
our ALMA Large Programme that we compare to our models.



\section{The high-contrast rings of AS 209, Elias 24, HD 163296, GW Lup and HD 143006}
\label{sec-data}
%
In this paper we focus on those sources of the DSHARP Programme
that show high-contrast and radially thin rings that are separated by deep
valleys, and that are sufficiently face-on to not have to worry much about 3-D
line-of-sight issues. These are AS 209, Elias 24, HD 163296, GW Lup and
HD 143006. Their stellar parameters are given in Table \ref{tab-stellar-params}.

\begin{table}
\begin{center}
\begin{tabular}{|c|ccc|}
\hline
\hline
Source     & $d\,\mathrm{[pc]}$ & $M_{*}/M_{\odot}$ & $L_{*}/L_{\odot}$ \\
\hline
AS 209     & 121              & 0.832        &  1.413 \\ % 2018.07.02
Elias 24   & 136              & 0.776        &  6.03  \\ % 2018.07.02
HD 163296  & 101              & 2.04         &  17.0  \\ % 2018.07.02
GW Lup     & 155              & 0.457        &  0.331 \\ % 2018.07.02
HD 143006  & 165              & 1.78         &  3.80 \\  % 2018.07.02 ** Big change **
\hline
\hline
\end{tabular}
\end{center}
\caption{\label{tab-stellar-params}The stellar parameters assumed for the stars
  studied in this paper. Distance is in parsec and mass and luminosity are in
  units of the solar values. More details, as well as references and uncertainty
  estimates, can be found in \paperdsharpandrews{}.}
\end{table}

A gallery of these sources is shown in Fig.~\ref{fig-obs-images}.  For an
overview of the ALMA Large Programme we refer to \paperdsharpandrews{}, and for
an in-depth discussion on the data of the individual sources we refer to
\paperdsharphuangrings{}, \paperdsharpisella{}, \paperdsharpguzmann{}.

\begin{figure*}
\centerline{\includegraphics[width=0.95\textwidth]{gallery_rings.pdf}}
\caption{\label{fig-obs-images}The continuum maps in
  band 6 of the five disks in our sample which have the most pronounced rings.
  The eight highest contrast rings, which are the topic of this paper,
  are marked in the images. For a detailed description of these
  data, see \paperdsharpguzmann{} for AS 209, \paperdsharpisella{}
  for HD 163296, and \paperdsharphuangrings{} for the rest.}
\end{figure*}

The high-contrast rings of these sources provide ``clean laboratories'' for
testing the theory of dust trapping in a ring-by-ring manner.
Fig.\ \ref{fig-obs-profiles} shows the radial profile (deprojected for
inclination) of the thermal emission of the dust of the five disks. The
procedure used to extract these radial profiles from the continuum maps is
described by \paperdsharphuangrings{}, though we removed the blob seen in HD
163296 and HD 143006 for this work.

The brightness profiles are expressed as linear brightness temperature
$I_\nu^{\mathrm{lb}}$, which we define in Appendix \ref{sec-conventions}. The
choice of linear brightness temperature over full brightness temperature is
important, because the full brightness temperature tends to vastly underestimate
the contrast between the rings and the inter-ring regions. The linear brightness
temperature is instead linearly proportional to the intensity in Jy/beam, and
thus represents the contrast appropriately.

\begin{figure*}
\centerline{\includegraphics[width=0.85\textwidth]{gallery_profiles.pdf}}
\caption{\label{fig-obs-profiles}The linear brightness temperature in
  band 6 of the five disks in our sample which have the most pronounced rings.
  The vertical axis is logarithmic to better show the contrast.
  The eight highest contrast rings are fitted by a Gaussian profile, shown
  as the solid inverse parabolas. The dotted inverse parabolas are Gaussians
  with the width of the ALMA beam. For a detailed description of these
  data, \paperdsharpguzmann{} for AS 209, \paperdsharpisella{}
  for HD 163296, and \paperdsharphuangrings{} for the rest.
{\bf TODO: y-labels}}
\end{figure*}


\section{Fitting a Gaussian profile to the ring emission}
\label{sec-gauss-fits}
%
As we will discuss later (Section \ref{sec-rings-as-dust-traps} and Appendix
\ref{sec-steady-state-analytic-trap-model}), for a
Gaussian pressure bump the solution to the radial dust mixing and drift problem
is, to first approximation, also a Gaussian, with a width smaller than, or equal
to, that of the gas pressure bump. Our analysis of the eight rings of this paper
therefore naturally starts with the fitting of the observed radial intensity
profiles with a Gaussian function. For some {\bf [CHECK]} of the rings of this
paper such Gaussian fits were done in the context of fitting the entire radial
intensity profile in the uv-plane to the ALMA data ({\bf Guzman et al.XXXX;
  More??}). Our fits may deviate slightly from those {\bf [CHECK]}, because here
we put emphasis on the shape of the peak of the rings, and deemphasize the
wings. {\bf CHECK}.

The aim is now to find, for each ring, a Gaussian linear brightness temperature
profile
\begin{equation}\label{eq-gauss-lin-br-temp}
I_{\nu}^{\mathrm{lb,gauss}}(r) = A\,\exp\left(-\frac{(r-r_0)^2}{2\sigma^2}\right)
\end{equation}
that best describes the ring. To be more precise: We determine the values of
$A$, $r_0$ and $\sigma$ for which Eq.~(\ref{eq-gauss-lin-br-temp}) best fits the
observed linear brightness temperature profile $I_{\nu}^{\mathrm{lb,obs}}(r)$
shown in Fig.~\ref{fig-obs-profiles} within a prescribed radial domain as given
in Table \ref{tab-gauss-params}. Details of the fitting procedure are described
in Appendix \ref{sec-gauss-fitting-procedure}. The Gaussian fits are shown as
inverse parabolas in Fig.\ \ref{fig-obs-profiles}. In
Fig.~\ref{fig-obs-gaussfits} these Gaussian fits are shown in close-up. The
parameters of the best fits are listed in Table \ref{tab-gauss-params}.  For the
error estimates on these values, see Appendix \ref{sec-gauss-fitting-procedure}.

\begin{figure*}
\centerline{\includegraphics[width=0.85\textwidth]{gaussfits_allrings.pdf}}
\caption{\label{fig-obs-gaussfits}Gaussian fits to the eight rings of this
  paper. The blue curves are the observations, the orange curves are the best fit
  Gaussian profiles. The ``fit range'' bar shows the radial range within which
  the Gauss was fitted to the data. The fit range was chosen to fit the part of
  the curve that, by eye, most resembles a Gaussian. The ``beam'' bar shows the
  FWHM beam size of the observations. The grey band around the blue curve shows
  the estimated uncertainty of the data. {\bf TODO: y-labels}}
\end{figure*}

\begin{figure*}
  \centerline{
    %\includegraphics[width=0.3\textwidth]{AS_209_doublegauss.pdf}
    \includegraphics[width=0.42\textwidth]{HD_163296_doublegauss.pdf}
    \includegraphics[width=0.42\textwidth]{HD_143006_doublegauss.pdf}
  }
  \caption{\label{fig-obs-doublegauss}The sum of the two Gaussian fits
    for the two sources with two partly overlapping rings: HD 163296 and
    HD 143006. {\bf TODO: y-labels}}
\end{figure*}

\begin{table*}
\begin{center}
\begin{tabular}{|cc|cc|cccccc|ccccc|}
\hline
\hline
%Source & Ring & $r_i\, [\mathrm{au}]$ & $r_o\, [\mathrm{au}]$ & $A\, [\mathrm{K}]$ & $r\, [\mathrm{au}]$ & $\mathrm{fwhm}\, [\mathrm{mas}]$ & $\sigma\, [\mathrm{au}]$ & $w_d\, [\mathrm{au}]$ & $T_{\mathrm{d}}\,\mathrm{[K]}$ & $w_d/h_p$ & $M_{\mathrm{d,ri}} [M_{\oplus}]$ & $M_{\mathrm{d,iv}} [M_{\oplus}]$\\
%Source & Ring & $r_{\mathrm{in}}$ & $r_{\mathrm{out}}$ & $A$ & $r_0$ & $\mathrm{fwhm}$ & $\sigma$ & $w_d$ & $T_{\mathrm{d}}$ & $T_{\mathrm{d,br,lin}}$ & $w_d/h_p$ & $M_{\mathrm{d,ri}}$ & $M_{\mathrm{d,iv}}$\\
%            & & $[\mathrm{au}]$ & $[\mathrm{au}]$ & $[\mathrm{K}]$ & $[\mathrm{au}]$ & $[\mathrm{mas}]$ & $[\mathrm{au}]$ & $[\mathrm{au}]$ & $\mathrm{[K]}$ & $\mathrm{[K]}$ &  & $[M_{\oplus}]$ & $[M_{\oplus}]$\\
%Source & Ring & $r_{\mathrm{in}}$ & $r_{\mathrm{out}}$ & $A$ & $r_0$ & $\mathrm{fwhm}$ & $\sigma$ & $w_d$ & $T_{\mathrm{d}}$ & $T_{\mathrm{d,br,lin}}$ & $w_d/h_p$ & $\tau$ & $M_{\mathrm{d,ri}}$ & $M_{\mathrm{d,iv}}$\\
%            & & $[\mathrm{au}]$ & $[\mathrm{au}]$ & $[\mathrm{K}]$ & $[\mathrm{au}]$ & $[\mathrm{mas}]$ & $[\mathrm{au}]$ & $[\mathrm{au}]$ & $\mathrm{[K]}$ & $\mathrm{[K]}$ &  &  & $[M_{\oplus}]$ & $[M_{\oplus}]$\\
%Source & Ring & beam & domain & $A$ & $r_0$ & $\mathrm{fwhm}$ & $\sigma$ & $w_d$ & $T_{\mathrm{d}}$ & $T_{\mathrm{d,br,lin}}$ & $w_d/h_p$ & $\tau$ & $M_{\mathrm{d,ri}}$ & $M_{\mathrm{d,iv}}$\\
%            & & $[\mathrm{mas}]$ & $[\mathrm{au}]$ & $[\mathrm{K}]$ & $[\mathrm{au}]$ & $[\mathrm{mas}]$ & $[\mathrm{au}]$ & $[\mathrm{au}]$ & $\mathrm{[K]}$ & $\mathrm{[K]}$ &  &  & $[M_{\oplus}]$ & $[M_{\oplus}]$\\
Source & Ring & beam & domain & $A$ & $A_{\mathrm{dec}}$ & $r_0$ & $\mathrm{FWHM}$ & $\sigma$ & $w_d$ & $T_{\mathrm{d}}$ & $B_{\mathrm{d}\nu}^{\mathrm{lb}}$ & $w_d/h_p$ & $\tau_\nu^{\mathrm{peak}}$ & $M_{\mathrm{d}}$\\
            & & $[\mathrm{mas}]$ & $[\mathrm{au}]$ & $[\mathrm{K}]$ & $[\mathrm{K}]$ & $[\mathrm{au}]$ & $[\mathrm{mas}]$ & $[\mathrm{au}]$ & $[\mathrm{au}]$ & $\mathrm{[K]}$ & $\mathrm{[K]}$ &  &  & $[M_{\oplus}]$\\
\hline
% NOTE: This table is made automatically with python_make_tables/make_latex_table.py
% Note: New table made with new stellar parameters 2018.07.02
% Note: New table with new beam size estimates 2018.08.06
% Note: Added beam size to table, changed a bit the layout, and used new opacities for mass estimates (though still needs improving) 2018.08.07
% Note: Removed the Ivezic mass column
% Note: Added A_dec
AS 209     & 1 & 37 &  69 --  79 &   3.64 &   4.18 &  74.1 &  75.4 &   3.87 &   3.38 &  15.8 &  10.9 &   0.6 &   0.48 &  28.5\\
AS 209     & 2 & 37 & 115 -- 125 &   2.95 &   3.23 & 120.3 &  90.3 &   4.64 &   4.23 &  12.4 &   7.7 &   0.4 &   0.55 &  60.7\\
Elias 24   & 1 & 35 &  72 --  82 &   5.22 &   5.69 &  76.9 &  88.0 &   5.08 &   4.66 &  22.3 &  17.2 &   0.6 &   0.40 &  33.4\\
HD 163296  & 1 & 43 &  60 --  76 &   8.48 &   8.76 &  67.4 & 171.7 &   7.36 &   7.13 &  30.8 &  25.6 &   1.6 &   0.42 &  43.8\\
HD 163296  & 2 & 43 &  94 -- 104 &   4.92 &   5.20 &  99.9 & 133.6 &   5.73 &   5.42 &  25.3 &  20.2 &   0.8 &   0.30 &  38.0\\
GW Lup     & 1 & 44 &  79 --  89 &   1.48 &   1.87 &  85.1 &  72.1 &   4.74 &   3.76 &  10.3 &   5.7 &   0.5 &   0.40 &  34.1\\
HD 143006  & 1 & 45 &  35 --  45 &   3.21 &   4.11 &  41.0 &  72.2 &   5.06 &   3.95 &  27.2 &  22.0 &   1.9 &   0.21 &  10.0\\
HD 143006  & 2 & 45 &  59 --  72 &   2.48 &   2.71 &  64.8 & 112.5 &   7.88 &   7.22 &  21.6 &  16.6 &   2.0 &   0.18 &  21.6\\
\hline
\hline
\end{tabular}
\end{center}
\caption{\label{tab-gauss-params}The model parameters for the Gaussian ring fits
  in Figs.~\ref{fig-obs-profiles} and \ref{fig-obs-gaussfits}. The ``beam''
  column gives the beam size (full-width-at-half-max). Given that in reality the
  beam is elliptic, and the disk has an inclination, this is merely an estimate.
  The ``domain'' column gives the radial domain within which the Gaussian fit
  was made. $A$ is the peak linear brightness temperature of the best-fit
  Gaussian ring model, $r_0$ is its radius in $\mathrm{au}$, $\mathrm{FWHM}$ is
  the width expressed as full-width-at-half-maximum in units of milliarcsecond,
  $\sigma$ is its width expressed as standard deviation in units of
  $\mathrm{au}$ and $w_d$ is the width of the underlying (deconvolved) dust
  emission profile, also expressed as standard deviation in units of
  $\mathrm{au}$. The $A_{\mathrm{dec}}$ column is like $A$, but corrected for
  the deconvolution (see Eq.~\ref{eq-a-deconv}). The last five columns are
  derived from a simple disk model assumption: $T_{\mathrm{d}}$ is the midplane
  temperature of the disk (we assume gas and dust temperature to be equal)
  computed from Eq.~(\ref{eq-disk-temperature-model}), assuming a flaring angle
  of $\varphi=0.02$. From that one can derive the corresponding linear
  brightness temperature $B_{\mathrm{d}\nu}^{\mathrm{lb}}$ using
  Eq.~(\ref{eq-app-bnu-bright}) for $\nu=c/(0.13\,\mathrm{cm})$ (ALMA band 6).
  The $w_d/h_p$ column expresses the deconvolved dust ring width in units of the
  disk pressure scale height $h_p$ computed from $T_{\mathrm{d}}$. The
  $\tau_\nu^{\mathrm{peak}}$ column gives the estimated optical depth at the peak of the ring,
  calculated from Eq.~(\ref{eq-tau-estimate}). The last column is the computed dust mass in the
  ring in units of Earth masses. See Appendix \ref{sec-computing-dust-mass} for
  details. In making these mass estimates we use the DSHARP dust opacity model
  (\paperdsharpbirnstiel{}) for a grain radius of $a=0.1\,\mathrm{cm}$, which
  yields an absorption opacity
  $\kappa_\nu^{\mathrm{abs}}(\lambda=0.13\,\mathrm{cm})=1.93\,\mathrm{cm}^2/\mathrm{g}$.
  The uncertainty of the opacity value is, however, very large, since we do not
  know apriori the grain size distribution.}
\end{table*}

The observed rings are the result of the thermal emission of a dust ring convolved
with the ALMA beam. To obtain the width of the
underlying dust ring we have to deconvolve. Assuming a Gaussian beam and a
Gaussian dust ring, we can use the rule of the convolution of two Gaussians, and
obtain the width $w_d$ of the dust ring
\begin{equation}\label{eq-simple-deconvolve-gauss}
w_d=\sqrt{\sigma^2-b^2}
\end{equation}
where $b$ is the beam width expressed as standard deviation in units of
$\mathrm{au}$. It is computed from $b_{\mathrm{fwhm,as}}$ (the beam full-width
at half-maximum (FWHM) in units of arcseconds) through the equation
$b=d_{\mathrm{pc}}b_{\mathrm{fwhm,as}}/2.355$, where $d_{\mathrm{pc}}$ is the
distance to the source in units of parsec. In using
Eq.~(\ref{eq-simple-deconvolve-gauss}) we have ignored the elliptical shape of
the beam as well as the inclination of the disk.

The slightly narrower deconvolved ring should also have a correspondingly
higher amplitude $A_{\mathrm{dec}}$ given by
\begin{equation}\label{eq-a-deconv}
A_{\mathrm{dec}} = \frac{\sigma}{w_d} \, A
\end{equation}
to conserve luminosity, where we ignore the geometric effects due to the
circular coordinates. The values of $A_{\mathrm{dec}}$ are listed in Table
\ref{tab-gauss-params} as well.

For completeness, let us note that the deconvolved Gaussian model then becomes
\begin{equation}\label{eq-gauss-lin-br-temp-deconv}
I_{\nu}^{\mathrm{lb,gauss,dec}}(r) = A_{\mathrm{dec}}\,\exp\left(-\frac{(r-r_0)^2}{2w_d^2}\right)
\end{equation}

The width of the dust rings in units of the local pressure scale height $h_p$
(see Table \ref{tab-gauss-params}) can only be roughly estimated, because we do
not know $h_p$ well enough due to our lack of information about the disk midplane
temperature. Our estimate of the midplane disk temperature is made using the
following simple irradiated flaring disk recipe:
\begin{equation}\label{eq-disk-temperature-model}
T_{\mathrm{d}}(r) = \left(\frac{\tfrac{1}{2}\varphi L_{*}}{4\pi r^2\sigma_{\mathrm{SB}}}\right)^{1/4}
\end{equation}
where $\sigma_{\mathrm{SB}}$ is the Stefan-Boltzmann constant and $\varphi$ is
the so-called flaring angle \citep[e.g.][]{1997ApJ...490..368C,
  1998ApJ...500..411D, 2001ApJ...560..957D}. The factor of $1/2$ in front of the
flaring angle $\varphi$ originates from the consideration that the stellar
radiation heats the surface layer of the disk, which then radiates half of that
energy away and half into the deeper regions of the disk. Only the latter half
goes into the energy balance equation between irradiation and radiative cooling
of the disk interior. We take the flaring angle to be $\varphi=0.02$ which is an
estimate based on typical values from models. In using this simple recipe we
assume that the optical and infrared dust opacity is dominated by a population
of small grains that is vertically more extended than the larger dust grains
seen in the ALMA images. The two dust populations may
be largely decoupled (see e.g.\ the BL-series of models of
\citet{2004A&A...417..159D} as an illustration). The value of $\varphi=0.02$ is
nothing more than an educated guess, so its value can easily be wrong by a
factor of four or so. But the fact that the pressure scale height goes as
$T_{\mathrm{d}}^{1/2}\propto \varphi^{1/8}$ means that such an error only
affects $h_p$ mildly.

Assuming that the gas temperature is equal to the dust temperature, the
pressure scale height of the disk now
follows from $h_p=\sqrt{k_BT_{\mathrm{d}}r^3/\mu m_pGM_{*}}$ with $k_B$ the
Boltzmann constant, $m_p$ the proton mass, $G$ the gravitational constant and
$\mu=2.3$ the mean molecular weight in atomic units. The stellar parameters are
taken from Table \ref{tab-stellar-params}.

We see from Table \ref{tab-gauss-params} that some rings are narrower than the
(estimated) pressure scale height $h_p$, while others are broader. This
comparison is important because long-lived pressure bumps in the gas cannot be
radially narrower than about one pressure scale height. If a thermal emission
ring produced by the dust is substantially narrower than $h_p$, then some kind
of dust trapping must have taken place. We can thus conclude that we have strong
evidence of dust trapping operating in the rings of AS 209, Elias 24 and GW Lup.
For the other rings dust trapping is certainly not ruled out either, but would
require further evidence. 

For these observations the FWHM beam size for each ring is listed in Table
\ref{tab-gauss-params}. These are azimuthal average beam size values according
to Eq.~(\ref{eq-def-average-beam}), based on the values listed in
\paperdsharphuangrings{}. This number can be compared to the FWHM ring width in
Table \ref{tab-gauss-params}, and it shows that the widths of all the rings are
spatially resolved. Some are only marginally resolved (AS 209 rings 1 and 2, GW
Lup and the ring 1 of HD 143006), but the data are inconsistent with radially
unresolved rings. For AS 209 this is confirmed by more detailed analysis in the
uv-plane by \paperdsharpguzmann{}. {\bf CHECK IF HUANG DOES THIS, TOO.}

These Gaussian profiles fit the observed data remarkably well, as can be best
seen in Fig.~\ref{fig-obs-gaussfits}.  For the broad rings of HD 163296 the
closeness to a Gaussian profile is particularly striking, given that these rings
are resolved by many beam widths.  For Elias 24 the ring fits a Gaussian near
the peak, but deviates from Gaussian already less half a width from the
center. Likewise for the rings of HD 143006. The ring around GW Lup is an
intermediate case, which fits a Gauss well on the inner flank, but rather badly
on the outer flank.

In all ring profiles shown in Fig.~\ref{fig-obs-gaussfits} the observed profile
rises above the Gaussian fit at some point in the wings. For the inner flanks of
ring 1 of Elias 24 and ring 1 of HD 143006, as well as for the outer flanks of
ring 2 of AS 209, ring 2 of HD 163296, the ring of GW Lup and ring 2 of HD
143006 the excess above the Gaussian gradually increases away from the peak of
the Gaussian. Put differently, the profiles appear to deviate from Gaussian
shape to pyramid-shape (or Lorenzian shape) in the flanks. This cannot be simply
the addition of a constant background, because that would preserve the Gaussian
shape, and merely lift it higher. Instead, while the core of the profile can clearly
be well-described by a Gaussian, in the flanks there is a clear transition
to a different, wider, shape.

A similar effect occurs between the two rings in HD 163296 and HD 143006, where
in particular for HD 143006 it is clear from Fig.~\ref{fig-obs-doublegauss} that
it cannot be just the effect of the sum of the two Gaussians.

It is not immediately clear whether this is relevant or just chance. It lies in
the nature of fitting a Gaussian profile to something non-Gaussian that there
will be a region close to the peak where the curve fits the Gaussian reasonably
well, while the deviation will increase the farther away from the peak one
looks. This danger is particularly large in the present case, since the fitting
range was chosen to maximize the similarity to the Gauss near the
peak. Nevertheless, it clearly signals a deviation from Gaussian.



\section{Optically thin analysis}
\label{sec-opt-thin-analysis}
%
As a first guess, we will assume that we can ignore optical depth effects,
i.e.\ that the thermal emission of the dust is optically thin. The linear
brightness temperature profiles shown in Section \ref{sec-data}, after
deconvolution with the beam, are then linear maps of the spatial distribution of
dust, if we ignore any temperature gradients across these rings.  In other
words: the deconvolved versions of the Gaussian fits of Section
\ref{sec-gauss-fits} (Eq.~\ref{eq-gauss-lin-br-temp-deconv}) will then directly
translate in Gaussian models of the radial dust distribution.

The conversion between the deconvolved observed linear brightness temperature
profile $I_\nu^{\mathrm{lb,dec}}(r)$ and the dust surface density profile
$\Sigma_d(r)$ is then
\begin{equation}\label{eq-optthin-conversion-linbright-sigmad}
  %I_\nu^{\mathrm{lb}} = \Sigma_d\,\kappa_\nu^{\mathrm{abs}}\, B_\nu^{\mathrm{lb}}(T_d)
  \Sigma_d(r) =  \frac{I_\nu^{\mathrm{lb,dec}}(r)}{\kappa_\nu^{\mathrm{abs}}\, B_\nu^{\mathrm{lb}}(T_d)} 
\end{equation}
where $T_d$ is the temperature of the dust, $\kappa_\nu^{\mathrm{abs}}$ is the
absorption opacity, and $B_\nu^{\mathrm{lb}}(T_d)$ is the Planck function
expressed as linear brightness temperature (Eq.~\ref{eq-app-bnu-bright}).

By replacing $I_\nu^{\mathrm{lb,dec}}(r)$ with the Gaussian fit
$I_\nu^{\mathrm{lb,gauss,dec}}(r)$ given by
Eq.~(\ref{eq-gauss-lin-br-temp-deconv}) we obtain the corresponding
$\Sigma_d^{\mathrm{gauss}}(r)$ from
Eq.~(\ref{eq-optthin-conversion-linbright-sigmad}). From this Gaussian
model we can derive the total dust mass trapped in the ring:
\begin{equation}\label{eq-dust-mass-estimate}
  M_d = \int_0^\infty 2\pi r \Sigma_d(r) dr \simeq
  \frac{(2\pi)^{3/2} r_0\, A\,\sigma}{\kappa_\nu^{\mathrm{abs}}\, B_\nu^{\mathrm{lb}}(T_d)}
\end{equation}
where we used the identity $A\,\sigma=A_{\mathrm{dec}}\,w_d$.

We use the DSHARP opacity model (\paperdsharpbirnstiel{}) which, for a grain
radius of $a=0.1\,\mathrm{cm}$ yields a dust opacity of
$\kappa_\nu^{\mathrm{abs}}(\lambda=0.13\,\mathrm{cm})=1.93\,\mathrm{cm}^2/\mathrm{g}$.
The resulting dust mass estimates are listed in Table \ref{tab-gauss-params}.

The main uncertainty lies in the uncertainty in the opacity value
$\kappa_\nu^{\mathrm{abs}}$. This value depends on the grain size (or grain size
distribution) as well as many other factors including composition, grain shape
and uncertainties in the method of computation of the opacity. We refer to
\paperdsharpbirnstiel{} for a discussion of the opacity model used for the
DSHARP campaign.

{\bf [Discuss the minimum and maximum mass values]}

The other uncertainty, though much less severe, is the dust temperature
$T_d$. From the continuum images we have no information about $T_d(r)$. From the
CO line emission one can estimate the temperature in the disk surface layers,
but not in the midplane \citep[see e.g.][]{2018ApJ...853..113W}. Instead, we
will use the flaring angle recipe given by Eq.~(\ref{eq-disk-temperature-model})
with $\varphi=0.02$. The resulting values of $T_d$ at the peak of the rings are
given in Table \ref{tab-gauss-params}. For
Eq.~(\ref{eq-optthin-conversion-linbright-sigmad}) we need the corresponding
linear brightness temperature $B_{\mathrm{d}\nu}^{\mathrm{lb}}\equiv B_\nu^{\mathrm{lb}}(T_d)$, which is listed in the
next column in Table \ref{tab-gauss-params}.

Given the amplitude of the deconvolved Gauss fit $A_{\mathrm{dec}}$ (see
Eq.~\ref{eq-gauss-lin-br-temp-deconv}), we can derive the optical depth
$\tau_\nu^{\mathrm{peak}}$ of the ring at its peak at $r=r_0$:
\begin{equation}\label{eq-tau-estimate}
\tau_\nu^{\mathrm{peak}} = -\ln\left(1-\frac{A_{\mathrm{dec}}}{B_{\nu}^{\mathrm{lb}}(T_d)}\right)
\end{equation}
This allows us to verify if our assumption of optical thinness is justified. The
results are listed in Table \ref{tab-gauss-params}.  We conclude that the
optically thin assumption is not entirely wrong, but not quite right either. The
estimated dust surface density $\Sigma_d$ and dust mass $M_d$ derived from the
optically thin assumption (Eqs.~\ref{eq-optthin-conversion-linbright-sigmad} and
\ref{eq-dust-mass-estimate}, respectively) are only mildly affected by the
optical depth effects.

A much stronger effect on the conversion of the observations to dust mass
remains the uncertainty of the opacity. As shown in \paperdsharpbirnstiel{}
the value of $\kappa_\nu^{\mathrm{abs}}=1.93\,\mathrm{cm}^2/\mathrm{g}$ that
we use here can easily be wrong by a factor of $10$ upward or a factor of
$0.1$ downward, with correspondingly large changes in the derived dust mass.




\section{Optical depth effects}
\label{sec-optical-depth-effects}
%
As we have seen in Section \ref{sec-opt-thin-analysis}, the optical depth of
these dust rings is not very low: about 0.2 and upward, usually around 0.4. For
the case of HD 163296 there is evidence from the absorption of CO line emission
from the back side of the disk that the optical depth in the two prominent rings
is even close to unity, as shown by \paperdsharpisella{}. For simplicity we will
perform most of our analysis of the later Sections using the optically thin
approximation. But before doing that, it is important to investigate how strong
the effect of optical depth is, and how it would affect our results. This is the
topic of this Section.

\subsection{Simple optical depth correction}
Let us assume that the dust has zero albedo. Then the inclusion of the effect of
optical depth is simple. We replace
Eq.~(\ref{eq-optthin-conversion-linbright-sigmad}) with the formal transfer
equation:
\begin{equation}\label{eq-simple-formal-rt}
  I_\nu^{\mathrm{lb,dec}}(r) = e^{-\tau_\nu(r)}I_\nu^{\mathrm{lb,bg}} +
  \left(1-e^{-\tau_\nu(r)}\right)B_\nu^{\mathrm{lb}}(T_d)
\end{equation}
where $\tau_\nu(r)$ is the optical depth profile across the ring, and
$I_\nu^{\mathrm{lb,bg}}$ is the background brightness, either from background
clouds or from the cosmic microwave background. In interferometric observations
a flat background is Fourier-filtered out, but in the radiative transfer it
plays a role in the non-linear optical depth regime ($\tau_\nu\gtrsim 0.3$),
even if we are only interested in the excess emission above the background
$I_\nu-I_\nu^{\mathrm{bg}}$. We assume, however, that the cloud background
emission is negligible. The linear brightness temperature of the cosmic
background at $\lambda=0.13\,\mathrm{cm}$ is
$I_\nu^{\mathrm{lb,cmb}}=0.195\,\mathrm{K}$. This is not negligible compared to
the brightness temperatures in the inter-ring regions, but those regions are low
optical depth regions, where the radiative transfer equation for for
$I_\nu-I_\nu^{\mathrm{bg}}$ is linear and the background drops out. Only near
the peak of the rings the optical depth becomes non-negligible, but the value of
$B_{\mathrm{d}\nu}^{\mathrm{lb}}\equiv B_\nu^{\mathrm{lb}}(T_d)$ is always much
larger than the cosmic background value of $0.195\,\mathrm{K}$ (see Table
\ref{tab-gauss-params}). We will therefore ignore the background and set
$I_\nu^{\mathrm{lb,bg}}=0$ in Eq.~(\ref{eq-simple-formal-rt}) for simplicity.

To obtain the dust distribution we first compute $\tau_\nu(r)$
\begin{equation}\label{eq-tau-profile}
\tau_\nu(r) = -\ln\left(1-\frac{I_\nu^{\mathrm{lb,dec}}(r)}{B_\nu^{\mathrm{lb}}(T_d)}\right)
\end{equation}
which is identical to Eq.~(\ref{eq-tau-estimate}), but now for the entire
profile instead of just for the peak. The profile for $\Sigma_d(r)$ now follows
from
\begin{equation}
\Sigma_d(r) = \frac{\tau_\nu(r)}{\kappa_\nu^{\mathrm{abs}}}
\end{equation}

\subsection{Saturation and flat-topped ring profiles}
\label{sec-saturation-flat-top-profiles}
%
Given the non-linearity of Eq.~(\ref{eq-tau-profile}), in particular near
the peak of the ring profile, the resulting $\Sigma_d(r)$ profile becomes
non-Gaussian even if the measured $I_\nu^{\mathrm{lb,dec}}(r)$ has a Gaussian
shape. Or put the other way: if $\Sigma_d(r)$ is a Gaussian profile, the
resulting $I_\nu^{\mathrm{lb,dec}}(r)$ will not: it will lead to a flattened
top, where $I_\nu^{\mathrm{lb,dec}}(r)$ saturates to $B_\nu^{\mathrm{lb}}(T_d)$.
See Fig.~\ref{fig-flat-topped} for a quantitative example.

\begin{figure}
  \centerline{\includegraphics[width=0.47\textwidth]{saturation_example.pdf}}
  \caption{\label{fig-flat-topped}The saturation effect of a gaussian emissivity
    profile at various peak optical depths. Solid curves: the result of the
    formal radiative transfer equation (Eq.~\ref{eq-simple-formal-rt} with
    background set to zero). Dotted curves: the unsaturated case, normalized to
    have the same total integral as the solid curves.}
\end{figure}

The Gaussian fits of Section \ref{sec-gauss-fits} suggest, however, that the
$I_\nu^{\mathrm{lb,dec}}(r)$ profiles do not show clear signs of being
flat-topped. A simple comparison to Fig.~\ref{fig-flat-topped} shows that peak
optical depths of $\gg 1$ can be clearly excluded.

However, the deviation from true Gaussian becomes small for $\tau_\nu\lesssim
1$. It is therefore an important question how strongly we can exclude the
flattening of the top of the peak, or in other words: how strong an upper limit
we can set on $\tau_\nu^{\mathrm{peak}}$. Can the flattening of the underlying
Gaussian emissivity due to self-absorption be mistaken for a different Gaussian
profile that is wider and lower than the underlying dust emissivity, because the
deviations vanish within the errorbars on the measurements? We investigate this
in detail in Appendix \ref{sec-mimicry-flat-topped}. We find that up to
$\tau_\nu\lesssim 1$ the flattened-out profile can still be reasonably well
fitted by a Gaussian function. This Gauss, of course, has a strongly reduced
amplitude compared to the optically thin approximation: by a factor
$(1-e^{-\tau_\nu^{\mathrm{peak}}})/\tau_\nu^{\mathrm{peak}}$, before beam
smearing, in order to mimick the self-absorption. Also, it has a slightly wider
shape, but this is a small effect. 

This means that, just from the analysis of the shape of the radial profiles of
the rings, we can only exclude large optical depth ($\tau_\nu^{\mathrm{peak}}\gg
1$), but not mild optical depth $\tau_\nu^{\mathrm{peak}}\simeq 0.3\cdots 1$.
The optically thin treatment, as we will adopt in Section
\ref{sec-rings-as-dust-traps} and onward, could therefore lead to ignoring
optical depth effects of the order of
$(1-e^{-\tau_\nu^{\mathrm{peak}}})/\tau_\nu^{\mathrm{peak}}$, which amounts to
an underestimation of dust mass of up to 37\% for $\tau_\nu^{\mathrm{peak}}$ up
to 1.

There is, however, another way to constrain these optical depth effects.  As we
have already seen in Section \ref{sec-gauss-fits}: given the measured linear
brightness temperature $I_\nu^{\mathrm{lb,obs}}$, and given a dust temperature
$T_d$, the optical depth follows immediately. In other words: a larger optical
depth than given in Table \ref{tab-gauss-params} requires a lower dust
temperature than given in that table. These temperature estimates are already on
the low side, by assuming the somewhat low value of the flaring angle
$\varphi=0.02$ in Eq.~(\ref{eq-disk-temperature-model}). It requires a quite
strong reduction of $\varphi$ to reduce the temperature sufficiently to increase
the $\tau_\nu^{\mathrm{peak}}$ from Table \ref{tab-gauss-params} to unity. As an
example let us take ring 1 of HD 163296 with $T_d=30.8\,\mathrm{K}$ and
$\tau_\nu^{\mathrm{peak}}=0.42$ according to Table \ref{tab-gauss-params}.
According to Eq.~(\ref{eq-tau-estimate}), if we set
$\tau_\nu^{\mathrm{peak}}=1$, we obtain from $A_{\mathrm{dec}}=8.76$ the value
$B_\nu^{\mathrm{lb}}(T_d)=13.9\,\mathrm{K}$, corresponding, via
Eq.~(\ref{eq-app-bnu-bright}) to $T_d=18.9\,\mathrm{K}$. This temperature is
almost twice as low as the original estimate, and would require, according to
Eq.~(\ref{eq-disk-temperature-model}), a flaring angle of
$\varphi=0.00082$. Such a flaring angle is unreasonably low and it invalidates
the simple flaring angle recipe in the first place. Instead we would need to
perform detailed 2-D/3-D radiative transfer calculations of the disk to
investigate if such a low temperature is physically attainable. Or equivalently,
such a study may allow us to find a theoretical lower limit to the dust
temperature, which then translates into an upper limit to the
$\tau_\nu^{\mathrm{peak}}$. Alternatively, and possibly more reliably, if we
obtain a high-resolution map in a second band, we can investigate the behavior
of the spectral index over the radial profile of the rings, which also provides
information about the optical depth.


\subsection{The effect of scattering albedo}
If the dust grains have a radius $a$ comparable to the wavelength of our
observations, the scattering albedo can be quite high. This means that the
absorption optical depth can be substantially lower than unity, even if the full
extinction optical depth (absorption plus scattering) is unity or larger.  The
near-unity extinction found in HD 163296 by \paperdsharpisella{} from the CO
maps could thus be compatible with the sub-unity absorption optical depth we
derived in our Gaussian fitting procedure (see Table \ref{tab-gauss-params}).

In fact, staying with the case of HD 163296, if we would assume that the albedo
is zero, i.e.\ that the measured extinction optical depth from the CO maps
equals the absorption optical depth, then we would find rather low dust
temperatures at the location of the rings, which may be hard to explain
theoretically, as shown in Section \ref{sec-saturation-flat-top-profiles}.
If, however, part of the extinction is due to scattering, then it is easier
to remain consistent with the dust temperature estimated from the flaring
angle recipe.

However, when scattering is included, the radiative transfer becomes more
complex than a simple use of a factor $1-e^{-\tau_\nu}$. For completeness we
describe, in Appendix \ref{sec-oned-radtrans-model}, an approximate solution to
this problem for a thin slab model.  In principle one would have to replace, in
the above sections, as well as in Appendix \ref{sec-mimicry-flat-topped}, all
instances of $1-e^{-\tau_\nu}$ with the more detailed radiative transfer model
of Appendix \ref{sec-oned-radtrans-model}. But to avoid too much complexity in
this paper, we leave this to an interested reader.

\subsection{Why are most rings so ``fine-tuned''?}
It is rather striking that in the analysis of the rings up to this point we have
found several rather ``fine-tuned'' properties. For instance, the rings of AS
209, Elias 24, GW Lup and the inner ring of HD 143006 have a width that is only
roughly twice the beam size (between $1.5$ and $2.4$ times, to be precise), but
none are unresolved. Given the small sample it is very well possible that this
is just coincidence. The fact that some rings (in particular those of HD 163296)
are clearly much wider, lends some support to this.

The derived peak optical depths for most sources (except HD 143006), assuming
our model of the dust temperature is correct, hover around 0.4, i.e.\ just in
between the optically thin and optically thick regime. Is this coincidence as
well, or is there a deeper reason behind this? In this paper we have focussed on
the strongest, most isolated rings in the DSHARP sample, which could mean that
this is a selection effect.

Finally, all ring profiles are remarkably similar to a Gaussian shape. On the
one hand, this may be something similar to the effect discussed in Section
\ref{sec-saturation-flat-top-profiles} and Appendix
\ref{sec-mimicry-flat-topped}. And with some rings being only about two beams
wide, any `peaklike' shape will tend to look similar to a Gaussian. On the other
hand, the two rings or HD 163296 are resolved by more than 3 beams, yet they
show a Gaussian shape well into the wings.

Could it be that some observational or data reduction effect may have smeared
out our data more than expected? We have, to the best of our abilities, been
unable to identify such an effect. And, as described in Appendix {\bf XXXX}, we
have performed a full self-test cycle of a mock ring being processed through our
pipeline, and found no excessive smearing-out, nor effects that would skew our
conclusion that the rings indeed have an intrinsically Gaussian profile. It seems
that these ``fine-tuned'' properties of the rings are real. 

\section{The rings as dust traps}
\label{sec-rings-as-dust-traps}
%
The hypothesis we are now going to test is that the
rings are caused by dust trapping in axisymmetric pressure bumps. For simplicity
we will assume that the radial gas pressure profile is fixed in time, and there
is no back reaction of the dust onto the gas. The pressure bump is assumed to be
so strong that the dust trapping in these rings is nearly perfect: no dust
escapes. We then expect that the dust distribution finds an equilibrium between
dust drift and turbulent spreading. Low turbulence will lead to thinner dust
rings than high turbulence. For simplicity we shall assume the gas pressure bump
to have a Gaussian radial profile with its peak at radius $r_0$ and width $w\ll
r_0$. This problem can be solved analytically.

\subsection{Model}
\label{sec-model-dusttrap-rings}
%
Consider the following radial Gaussian profile for the
pressure at the disk midplane:
\begin{equation}\label{eq-gaussian-pressure-bump}
p(r) = p_0 \exp\left(-\frac{(r-r_0)^2}{2w^2}\right)
\end{equation}
We will discuss the hydrodynamic stability of a gas
distribution arranged in this way in Section \ref{sec-gas-ring-stability}. But
for now we will not be concerned with that and we will assume the
pressure-profile of Eq.~(\ref{eq-gaussian-pressure-bump}) to be given and fixed.

In Appendix \ref{sec-steady-state-analytic-trap-model} we derive that, for our
purposes, a good approximation of the stationary dust distribution within the
dust trap is:
\begin{equation}\label{eq-analytic-sol-radial-trapping-summary}
\Sigma_{\mathrm{d}}(r) = \Sigma_{\mathrm{d0}} \exp\left(-\frac{(r-r_0)^2}{2w_{\mathrm{d}}^2}\right)
\end{equation}
where
\begin{equation}\label{eq-wd-afo-w-psi}
  w_{\mathrm{d}} = w\,\left(1+\psi^{-2}\right)^{-1/2}
\end{equation}
with $\psi$ given by
\begin{equation}\label{eq-psi-afo-alpha-sc-st}
\psi = \sqrt{\frac{\alpha_{\mathrm{turb}}}{\mathrm{Sc}\,\mathrm{St}}}
\end{equation}
Here $\mathrm{St}$ is the Stokes number of the dust particles
(Eq.~\ref{eq-definition-stokes-number}), $\mathrm{Sc}$ is the Schmidt number of
the turbulence in the gas, and $\alpha_{\mathrm{turb}}$ is the usual turbulence
parameter. Note that this solution is for a single grain size.

For large grains and/or weak turbulence one finds $\psi\ll 1$, which leads to
$w_{\mathrm{d}}\ll w$. In this case the dust is strongly trapped near the peak
of the pressure bump. The opposite is the case for small grains and/or strong
turbulence, for which one gets $\psi\gg 1$, which leads to
$w_{\mathrm{d}}\rightarrow w$. In this case the trapping is very weak and the
dust-to-gas ratio within the pressure bump stays constant. Only in the very
wings of the Gaussian pressure bump is dust trapping significant, but this
effect can only be studied using the more accurate non-gaussian solution of
Eq.~(\ref{eq-analytic-sol-radial-trapping-better}). We will return to the dust
trapping effect in the wings in Section {\bf XXXXX}.

Eq.~(\ref{eq-analytic-sol-radial-trapping-summary}) has only two parameters:
$\Sigma_{\mathrm{d0}}$ and $w_{\mathrm{d}}$. Both can be fairly directly
measured from the observations, though $\Sigma_{\mathrm{d0}}$ will depend on the
opacity, and hence on the unknown grain size (see discussion in Appendix
\ref{sec-computing-dust-mass}). The values of $w_{\mathrm{d}}$ for
the rings in our sample can be directly taken from Table \ref{tab-gauss-params}.

The width of the dust ring $w_{\mathrm{d}}$ is physically set by
$\alpha_{\mathrm{turb}}$, $\mathrm{Sc}$, $\mathrm{St}$ and $w$ through the above
equations. We therefore
have one observational value for four unknown parameters. This is heavily
degenerate. All we can do is to test if the measured value of
$w_{\mathrm{d}}$ is consistent with expected values of $\alpha_{\mathrm{turb}}$,
$\mathrm{Sc}$, $\mathrm{St}$ and $w$.

\subsection{Ranges of $\alpha_{\mathrm{turb}}$, $\mathrm{Sc}$, $\mathrm{St}$ and $w$}
\label{sec-ranges-of-params}
%
Reasonable values of $\alpha_{\mathrm{turb}}$, $\mathrm{Sc}$, $\mathrm{St}$ and
$w$ obey certain restrictions. First of all, the Schmidt number $\mathrm{Sc}$
is merely a way to relate the turbulent viscosity with the turbulent mixing. If
we do not strive to learn about the turbulent viscosity, and instead are
satisfied with learning only about the turbulent mixing, then we are only
interested in the combination $\alpha_{\mathrm{turb}}/\mathrm{Sc}$. For
simplicity we set $\mathrm{Sc}=1$.

The value of the turbulence parameter $\alpha_{\mathrm{turb}}$ is usually
considered to be between $10^{-6}\lesssim \alpha_{\mathrm{turb}}\lesssim
10^{-2}$.

The width of the pressure bump cannot be smaller than about a pressure
scale height, but also not smaller than the width of the dust ring.
Therefore $w_{\mathrm{min}}=\mathrm{max}(h_p,w_{\mathrm{d}})$.
In the case of the double rings (AS 209, HD 163296 and HD 143006),
the full-width-at-half-maximum $2.355\,w$ should not exceed the radial
separation of the rings. For the two single ring sources we take the deepest
point of the gap to the inside of the ring as the upper limit on the
half-width-at-half-maximum $1.178\,w$. These lower and upper limits on $w$
are listed in Table \ref{tab-ring-model-limits}.

The Stokes number $\mathrm{St}$ can be any value. But it is directly related to
the grain size $a$ and the gas density $\rho_{\mathrm{g}}$, where the gas
density is directly related to the gas surface density $\Sigma_{\mathrm{g}}$ via
$\Sigma_{\mathrm{g}}=\sqrt{2\pi}h_p\rho_{\mathrm{g}}$. If we have observational
constraints on $a$ and a good estimate of the gas surface density
$\Sigma_{\mathrm{g}}$, then we can eliminate this uncertainty, and we are left
with two unknown parameters ($\alpha_{\mathrm{turb}}$ and $w$) for one
measurement ($w_{\mathrm{d}}$).  Unfortunately, while estimating $a$ from
observations may be doable, it is far more difficult to estimate
$\Sigma_{\mathrm{g}}$. Standard disk gas mass estimates are of limited use, as
they are based on measuring the dust mass and multiplying it by the estimated
gas-to-dust ratio. Since we are testing the hypothesis of dust trapping, we
cannot assume a standard gas-to-dust ratio.

One can, however, set an upper bound on $\Sigma_{\mathrm{g}}$ by demanding that
the disk is gravitationally stable, i.e.\ that the Toomre parameter obeys
\begin{equation}
Q_{\mathrm{Toomre}}\equiv \frac{c_s\Omega_K}{\pi G \Sigma_{\mathrm{g}}} >2
\end{equation}
Here $c_s=\sqrt{k_BT/\mu m_p}$ is the isothermal sound speed, with $k_B$ the
Boltzmann constant, $m_p$ the proton mass, and $mu=2.3$ the mean molecular
weight in atomic units. $\Omega_K=\sqrt{GM_{*}/r^3}$ is the Kepler frequency,
$G$ is the gravitational constant and, finally, $\Sigma_{\mathrm{g}}$ the gas
surface density. Taking the disk midplane
temperature from Table \ref{tab-gauss-params}, which was calculated using
Eq.~(\ref{eq-disk-temperature-model}), we can compute the upper limits on
$\Sigma_{\mathrm{g}}$ for all of the rings. The results are given in Table
\ref{tab-ring-model-limits}.

\begin{table*}
\begin{center}
\begin{tabular}{|cc|ccccccc|ccc|}
\hline
\hline
Source     & Ring & $w_{\mathrm{min}}$ & $w_{\mathrm{max}}$ & $\Sigma_{\mathrm{g,min,iv}}$ & $\Sigma_{\mathrm{g,min,ri}}$ & $\Sigma_{\mathrm{g,max}}$ & $a_{\mathrm{max}}$ & $\mathrm{St}_{(a=0.02\,\mathrm{cm})}$ & $\alpha/\mathrm{St}$ & $\alpha/\mathrm{St}$ & $\alpha_{\mathrm{exmp}}$\\
           &      & $[\mathrm{au}]$  & $[\mathrm{au}]$  & $[\mathrm{g}/\mathrm{cm}^3]$ & $[\mathrm{g}/\mathrm{cm}^3]$ & $[\mathrm{g}/\mathrm{cm}^3]$ & $[\mathrm{cm}]$ & (for $\Sigma_{\mathrm{g,max}}$)& (for $w_\mathrm{max}$) & (for $w_\mathrm{min}$) &   \\
\hline
% Note: New table made with new stellar parameters 2018.07.02
% Note: New table with new beam size estimates 2018.08.06
AS 209     & 1 &  5.59 & 19.6 & 1.1e-02 & 2.0e-01 & 1.6e+01 & 15.7 & 3.9e-03 & 3.1e-02 & 5.7e-01 & 1.2e-04\\
AS 209     & 2 & 10.25 & 19.6 & 1.2e-02 & 2.3e-01 & 6.9e+00 &  5.8 & 9.1e-03 & 4.9e-02 & 2.1e-01 & 4.4e-04\\
Elias 24   & 1 &  7.26 & 17.1 & 9.7e-03 & 1.8e-01 & 1.8e+01 & 18.8 & 3.6e-03 & 8.0e-02 & 7.0e-01 & 2.9e-04\\
HD 163296  & 1 &  7.13$^{*}$ & 13.8 & 1.1e-02 & 1.9e-01 & 4.1e+01 & 40.0 & 1.5e-03 & 3.6e-01 & -- & 5.6e-04\\
HD 163296  & 2 &  7.07 & 13.8 & 7.8e-03 & 1.4e-01 & 2.0e+01 & 27.3 & 3.1e-03 & 1.8e-01 & 1.4e+00 & 5.6e-04\\
GW Lup     & 1 &  7.49 &  9.9 & 8.3e-03 & 1.5e-01 & 7.8e+00 &  9.8 & 8.0e-03 & 1.7e-01 & 3.4e-01 & 1.3e-03\\
HD 143006  & 1 &  3.95$^{*}$ & 10.1 & 4.7e-03 & 8.6e-02 & 7.5e+01 & 167.1 & 8.4e-04 & 1.8e-01 & -- & 1.5e-04\\
HD 143006  & 2 &  7.22$^{*}$ & 10.1 & 4.8e-03 & 8.8e-02 & 3.4e+01 & 73.1 & 1.9e-03 & 1.0e+00 & -- & 1.9e-03\\
\hline
\hline
\end{tabular}
\end{center}
\caption{\label{tab-ring-model-limits}Limits on the free parameters of the dust
  trapping model. The lower limit to the pressure bump width $w_{\mathrm{min}}$
  is the pressure scale height $h_p$. If, however, the width of the dust ring
  $w_d>h_p$, then the lower limit is $w_d$ (marked with the symbol $^{*}$).  The
  upper limit $w_{\mathrm{max}}$ is derived from the separation between the
  rings (for AS 209, HD 163296 and HD 143006) or from the separation of the ring
  to the nearest minimum (for Elias 24 and GW Lup). The two lower limits on the
  gas surface density $\Sigma_{\mathrm{g}}$ were derived by demanding
  $\Sigma_{\mathrm{g}}\gtrsim \Sigma_{\mathrm{d}}$, where $\Sigma_{\mathrm{d}}$
  was computed in Appendix \ref{sec-computing-dust-mass} using two different
  opacities: that of Ivezic et al. (leading to $\Sigma_{\mathrm{g,min,iv}}$) and
  that of Ricci et al.\ (leading to $\Sigma_{\mathrm{g,min,ri}}$). The column
  marked with $\Sigma_{\mathrm{g,max}}$ is the upper limit on the gas surface
  density derived from demanding that the gas disk is gravitationally
  stable. The $a_{\mathrm{max}}$ column is the maximum grain size for which the
  derived dust surface density together with the gas surface density remain
  gravitationally stable. The $\mathrm{St}_{(a=0.02\,\mathrm{cm})}$ column gives
  an example value of the Stokes number: it is the value of $\mathrm{St}$ if the
  grains have a radius of $0.02\,\mathrm{cm}$, for the case of
  $\Sigma_{\mathrm{g}}=\Sigma_{\mathrm{g,max}}$. The two columns for
  $\alpha/\mathrm{St}$ are derived for the widest and narrowest
  gas bump, respectively. The column $\alpha_{\mathrm{exmp}}$ is the value of
  $\alpha_{\mathrm{turb}}$ for an example choice of parameters: $w=w_{\mathrm{max}}$,
  $\Sigma_{\mathrm{g}}=\Sigma_{\mathrm{g,max}}$ and $a=0.02\,\mathrm{cm}$.}
\end{table*}

One can estimate a lower limit to the gas density by demanding that the gas
surface density must be at least as large as the dust surface density, since
dust trapping is unlikely to achieve a larger concentration of dust than
that. Depending on which of the two opacities we use (see Appendix
\ref{sec-computing-dust-mass}) we arrive at different estimates of
$\Sigma_{\mathrm{d}}(r=r_0)$. By demanding $\Sigma_{\mathrm{g}}\gtrsim
\Sigma_{\mathrm{d}}$ we arrive at two lower limits of the gas surface density, given in
Table \ref{tab-ring-model-limits}. It is likely that even for much
larger values of the gas surface density the dust-gas mixture becomes
unstable to the streaming instability and other types of instabilities.
If we can convince ourselves that such instabilities are not operating
in these rings, then this puts a substantially stronger (i.e.~larger)
lower limit on the gas surface density.

Also if the grains are much larger than $\lambda/(2\pi)\simeq 0.02\,\mathrm{cm}$,
the opacity drops and the resulting dust surface density estimate increases,
yielding larger lower limits to the gas density.

Along this line of thinking one can compute the largest grain radius for which
the Toomre parameter of the disk stays above 2. We use the Ivezic
opacity for that, which, for large grains, has similar values as the Ricci
opacity. This leads to values of several centimeters to half a meter (Table
\ref{tab-ring-model-limits}). Again these are conservative limits, with real
values likely to be substantially smaller. {\bf UPDATE THESE VALUES AND RICCI
NAME SINCE WE NOW USE DIFFERENT OPACITY MODEL.}


\subsection{Application to the observed rings}
\label{sec-application-to-rings}
%
We now wish to apply the model of Subsection \ref{sec-model-dusttrap-rings}
with the limits on the parameter ranges derived in Subsection
\ref{sec-ranges-of-params} to the observed ring widths $w_d$ listed in
Table \ref{tab-gauss-params}. 

From an assumed value of $w$ and the measured value $w_{\mathrm{d}}$ we can
directly compute the ratio $\alpha_{\mathrm{turb}}/\mathrm{St}$
\begin{equation}\label{eq-alphaSt-equation}
\frac{\alpha_{\mathrm{turb}}}{\mathrm{St}} \equiv \psi^{-2} = \left[\left(\frac{w}{w_{\mathrm{d}}}\right)^2-1\right]^{-1}
\end{equation}
where we used Eqs.~(\ref{eq-wd-afo-w-psi}, \ref{eq-psi-afo-alpha-sc-st}), and
set $\mathrm{Sc}=1$. We will consider two choices of $w$: the $w_{\mathrm{min}}$ and
$w_{\mathrm{max}}$ from Table \ref{tab-ring-model-limits}. 

For the choice $w=w_{\mathrm{max}}$ (the widest possible pressure bump) the dust
rings are all narrower that the gas rings: $w_{\mathrm{d}}<w$, which implies
that the dust trapping is operational, and Eq.~(\ref{eq-alphaSt-equation}) gives
information about the turbulent strength. For the choice $w=w_{\mathrm{min}}$
(the narrowest possible pressure bump) we can only use
Eq.~(\ref{eq-alphaSt-equation}) for the rings for which
$w_{\mathrm{d}}<h_p$. The reason is that for those rings with
$w_{\mathrm{d}}>h_p$ (marked with a $^{*}$ in Table \ref{tab-ring-model-limits})
the minimal pressure bump width is $w_{\mathrm{min}}=w_{\mathrm{d}}$, and the
dust ring is as wide as the pressure bump, implying that dust trapping is weak
or non-operational. Any increase of $\alpha_{\mathrm{turb}}/\mathrm{St}$ will
keep $w_{\mathrm{d}}=w_{\mathrm{min}}$, so one cannot derive any value for
$\alpha_{\mathrm{turb}}/\mathrm{St}$. But for other rings we can compute
$\alpha_{\mathrm{turb}}/\mathrm{St}$. The resulting values for both choices of
pressure bump width are given in Table \ref{tab-ring-model-limits}. They can be
understood as the lower and upper limit on $\alpha_{\mathrm{turb}}/\mathrm{St}$.

% If we will be able to measure the width $w$ of the gas pressure bump in an
% independent way, for instance with molecular line observations, then instead of
% the lower- and upper limits on $\alpha_{\mathrm{turb}}/\mathrm{St}$ we will be
% able to directly measure $\alpha_{\mathrm{turb}}/\mathrm{St}$. However, reliable
% measurements of the gas density are notoriously difficult. CO molecular lines
% are usually emitted in the warm sub-surface layers of the disk, not near the
% midplane (see e.g.~{\bf Weaver \& Isella companion paper as well as earlier
%   literature}), and CO appears to be depleted compared to chemistry models
% ({\bf XXXX}). Other gas tracers are harder to measure and also not reliable
% indicators of the gas density. Therefore we will, at least for now, have to
% make do with the lower and upper limits on $\alpha_{\mathrm{turb}}/\mathrm{St}$
% from Table \ref{tab-ring-model-limits}.

The next task is to convert from Stokes number $\mathrm{St}$ to grain size
$a$. The Epstein regime is valid for the typical grains we are interested in
(less than a meter in size), in which case $a$ and $\mathrm{St}$ are related
by 
\begin{equation}
\mathrm{St} = \frac{\pi}{2}\frac{\xi_{\mathrm{dust}}a}{\Sigma_{\mathrm{g}}}
\end{equation}
where $\Sigma_{\mathrm{g}}$ is the gas surface density and $\xi_{\mathrm{dust}}$
is the material density of the dust grains. Given that grains are expected
to be a mixture of silicate, amorphous carbon and water ice, a value of
$\xi_{\mathrm{dust}}\simeq 2\,\mathrm{g}/\mathrm{cm}^3$ is reasonable.

To get a feeling for the results, let us choose the grain size to be
$a=0.02\,\mathrm{cm}$, which corresponds to $\lambda/2\pi$ for
$\lambda=0.13\,\mathrm{cm}$ (the wavelength of ALMA band 6).  The corresponding
Stokes numbers, for the most massive possible gas disk
($\Sigma_{\mathrm{g}}=\Sigma_{\mathrm{g,max}}$), are listed in Table
\ref{tab-ring-model-limits}. This then allows us to convert the value of
$\alpha_{\mathrm{turb}}/\mathrm{St}$ into a value of
$\alpha_{\mathrm{turb}}$. For the case $w=w_{\mathrm{max}}$ this leads to values
$\alpha_{\mathrm{turb}}= 10^{-4}\cdots\mathrm{few}\,\times 10^{-3}$, listed in
Table \ref{tab-ring-model-limits} (column $\alpha_{\mathrm{exmp}}$).

These low values of $\alpha_{\mathrm{turb}}$ are consistent with the low values
or upper limits reported recently \citep{2016ApJ...816...25P,
  2018ApJ...856..117F} {\bf [more?]}. However, it has to be kept in mind that
the values of $\alpha_{\mathrm{turb}}=\alpha_{\mathrm{exmp}}$ were derived
for an example choice of parameters: : $w=w_{\mathrm{max}}$,
$\Sigma_{\mathrm{g}}=\Sigma_{\mathrm{g,max}}$ and $a=0.02\,\mathrm{cm}$. For a
smaller value of $w$, a lower value of $\Sigma_{\mathrm{g}}$, or larger grains,
the computed value of $\alpha_{\mathrm{turb}}$ will increase. So that it is hard
to set a true upper limit on $\alpha_{\mathrm{turb}}$ from these observations.

This seems to be in conflict with the results found by
\citet{2016ApJ...816...25P} who, from measuring an upper limit on the vertical
extent of the dust, imply an upper limit on $\alpha_{\mathrm{turb}}$ of $3\times
10^{-3}$. {\bf [IMPORTANT: We have to find out why Pinte et al. find this
    limit. Do they have additional constraints? Why can't they increase the grain
    size indefinitely? What are the constraints Eric and Andrea find for the
similar analysis of HD 163296?]}

Can we derive a {\em lower} limit to $\alpha_{\mathrm{turb}}$? This depends on
whether we have information about the grain size. At present we have only the
high resolution data for band 6, so we do not yet have information about the
radial profile of the spectral slope. But in several recent observations of the
spectral index across ringed disks
\citep{2015ApJ...808L...3A,2018ApJ...852..122H} one clearly sees that
$\alpha_{\mathrm{spec}}$ varies across these rings, being closer to $2$ at the
ring center and substantially larger between the rings. This makes sense in
terms of the dust trapping scenario in which we expect larger grains to be
trapped more efficiently (and thus dominate the peak of the ring) than smaller
grains, because the smaller grains will be more subject to turbulent
mixing. This scenario requires a grain size distribution, so that the width of
the dust ring is smaller for the bigger grains and bigger for the smaller ones.
We will discuss our model in the context of grain size distributions in Section
\ref{sec-model-with-grain-size-distribution}. But assuming that future data will
also show similar spectral slope changes for our sources, we can already
conclude that the grains trapped in the rings must have sizes larger than
$\lambda/2\pi=0.02\,\mathrm{cm}$. The measured (resolved) width of the dust ring
$w_d$ then implies a lower limit to $\alpha_{\mathrm{turb}}$. The value
$\alpha_{\mathrm{turb}}=\alpha_{\mathrm{exmp}}$ is then, in fact, this lower
limit. We will, however, have to wait until high resolution ALMA data in another
band becomes available to verify the minimal grain size of $0.02\,\mathrm{cm}$.

Should the low spectral slope at the peak of the rings be confirmed, then the
lower limits on $\alpha_{\mathrm{turb}}$ from the $\alpha_{\mathrm{exmp}}$
column in Table \ref{tab-ring-model-limits} paint a different picture from the
low-turbulence picture that has emerged in recent times. The values of a few
times $10^{-4}$ to a few times $10^{-3}$ are low, but they are only lower
bounds. It would be somewhat unlikely if all rings have parameters at this
extreme size of parameter space, and it is therefore likely that
$\alpha_{\mathrm{turb}}$ is easily a factor of ten higher or so. This would
imply that protoplanetary disks are substantially turbulent after all. But to
confirm this, we need to dig a bit deeper by including size distributions
(Section \ref{sec-model-with-grain-size-distribution}) and non-perfect
(``leaky'') dust traps (Section {\bf XXXX}).



{\bf We can make predictions for pressure slopes in CO}


{\bf CONSIDER PRESSURE OFFSET!}


\subsection{Including a grain size distribution}
%
\label{sec-model-with-grain-size-distribution}

{\bf [Here we will move away from a single grain size and see if we can understand
the deviations from Gaussian by a grain size distribution]}



\section{Streaming instability, clumping, and the spectral index}
\label{sec-si-clump}
%

{\bf [This section is temporarily hidden, because a lot still has to be
done on this. Maybe it should be removed and put into a future paper.]}

% Whenever the dust trapping becomes so effective that the volume density of the
% dust at the midplane of the disk exceeds the gas volume density, the streaming
% instability may set in \citep{2005ApJ...620..459Y}. This produces turbulence and
% leads to clumping \citep{2007ApJ...662..627J, 2008A&A...479..883L,
%   2010ApJ...722.1437B}. If conditions are right, some of these clumps may
% gravitationally collapse and form planetesimals \citep{2007Natur.448.1022J,
%   2017ApJ...847L..12S, 2017A&A...597A..69S}. It is believed that this mechanism
% is the trigger that starts the process of the formation of planets
% \citep[see][for a review]{2014prpl.conf..547J}. If one could find observational
% evidence of this dust clumping occurring in protoplanetary disks, this would
% be a major step forward for our understanding of planet formation.
% 
% The clumps that are formed in simulations of the streaming instability are,
% however, typically substantially smaller than a pressure scale height of the
% disk \citep{2007ApJ...662..627J, 2013MNRAS.434.1460K}. This makes it very hard
% to spatially resolve these structures, even with ALMA at its highest resolution.
% 
% It might, however, be possible to find indications of particle clumping in an
% indirect way, if it leads to the formation of optically thick clouds of
% particles surrounded by optically thin regions. Even if we cannot spatially
% resolve these clouds, the spatially averaged thermal emission from such a clumpy
% medium has a different spectral slope than that of a homogeneous distribution of
% dust. If all the dust is concentrated in a multitude of small and very optically
% thick clumps, then the brightness of this unresolved clumpy medium would be the
% Planck function times the covering fraction $C$ of the clumps:
% \begin{equation}\label{eq-bright-clumpy}
% I_\nu^{\mathrm{obs}} = B_\nu(T_{\mathrm{disk}})\,C
% \end{equation}
% where the limit $C\rightarrow 0$ means concentrating all the dust in infinitely
% dense clumps so that the covering fraction is zero, and $C\rightarrow 1$ means
% having so many clumps that they multiply overlap, when seen in projection from
% above. In contrast, if all this dust is homogeneously distributed, then the
% brightness is
% \begin{equation}\label{eq-bright-homogeneous}
% I_\nu^{\mathrm{obs}} = B_\nu(T_{\mathrm{disk}})\,(1-e^{-\tau_\nu})
% \end{equation}
% with $\tau_\nu$ being the vertical optical depth of the dust in the disk.
% Reality will likely lie in between these two extreme scenarios, with
% optically thin dust interdispersed between optically thick clouds.
% 
% There are at least two ways by which one could use Eqs.~(\ref{eq-bright-clumpy},
% \ref{eq-bright-homogeneous}) to search for hints of clumpiness. The most direct
% way is to make use of the different dependencies of these two equations on
% wavelength $\lambda=c/\nu$. In the case of the clumpy medium with perfectly
% optically thick clumps (Eq.~\ref{eq-bright-clumpy}) the brightness follows the
% Planck curve, whereas for the homogeneous medium
% (Eq.~\ref{eq-bright-homogeneous}), if it is optically thin
% (i.e.~$1-e^{-\tau_\nu}\simeq \tau_\nu$), it follows the Planck curve times the
% opacity law. If we are in the Rayleigh-Jeans domain of the Planck curve, then
% these two limits are usually referred to as spectral slope
% $\alpha_{\mathrm{spec}}=2$ and $\alpha_{\mathrm{spec}}=2+\beta$, respectively,
% where $\beta$ is defined by $\kappa_\nu\propto \nu^\beta$. At ALMA band 6, for
% the low temperatures we expect in these disks at large radii, we are no longer
% strictly in the Rayleigh-Jeans domain, so the spectral slope analysis becomes
% temperature-dependent, which complicates matters a bit, but the principle
% remains the same. If the streaming instability is operating in the rings of our
% sources, and if the numbers work out (see below), then we expect to see
% $\alpha_{\mathrm{spec}}\simeq 2$ within the rings, and
% $\alpha_{\mathrm{spec}}>2$ between the rings.
% 
% There is an important caveat to this idea: the grain size and the corresponding
% opacity slope may mimick the same spectral signatures as the clumping. This is,
% in fact, the same degeneracy of the interpretation of the millimeter spectral
% slope as usual \citep[e.g.][]{2003A&A...403..323T}, just on a much smaller
% spatial scale: a scale that remains unresolved even by ALMA.  Nevertheless it is
% worthwhile to investigate if we can find clues that point to clumpiness.
% 
% In the scope of this paper we will only do a very simple analysis: we will
% verify if, under the most benign conditions, the above mentioned optically
% thick/thin clumpy medium is at all possible. This is not granted, because
% these rings are several tens of au large, and therefore it requires a lot of
% mass to make dust clumps optically thick at 1.3 millimeter wavelength.
% Furthermore, these dust clouds cannot have a density in excess of the
% Roche density, because then they would immediately gravitationally collapse.
% Therefore, we cannot make the clumps arbitrary dense and compact.
% 
% So let us construct the most optimistic model of a dust cloud. The highest
% possible density of a cloud is the Roche density, defined by
% \begin{equation}
% \rho_{\mathrm{Roche}}(r) = \frac{9}{4\pi}\frac{M_{*}}{r^3}
% \end{equation}
% where $M_{*}$ is the stellar mass and $r$ is the distance to the star. We take
% $\rho_{\mathrm{cloud}}=\rho_{\mathrm{Roche}}$. The simplest opacity model for a
% dust grain is that of \cite{1997MNRAS.291..121I}, where we omit the scattering
% part:
% \begin{equation}
% \kappa(a) = \frac{3}{4}\frac{1}{\xi\,a}\left\{\begin{matrix}
% 1 & \hbox{for} & \lambda<2\pi a \\
% 2\pi a/\lambda & \hbox{for} & \lambda>2\pi a \\
% \end{matrix}\right.
% \end{equation}
% where $a$ is the grain radius and $\xi$ is the material density of the dust
% grain ($\xi=3.6$ for amorphous olivine, $\xi=2.0$ for silica {\bf CHECK}). For a
% given wavelength $\lambda$ (in our case $\lambda=0.13\,\mathrm{cm}$ for ALMA
% band 6), the most optimistic opacity is therefore $\kappa=(3\pi/2)/\xi\lambda$,
% for all $a\le \lambda/2\pi$. For ALMA band 6 and $\xi=2\,\mathrm{g/cm}^3$ this
% is $\kappa=18.1\,\mathrm{cm}^2/\mathrm{g}$. The largest grain size to still
% have this opacity is therefore $a=\lambda/2\pi$, which for ALMA band 6 is
% $a=0.02\,\mathrm{cm}$.  We make a spherical cloud of this dust with radius
% $R$. The optical depth through the cloud center is then
% \begin{equation}
% \tau = 2R\rho_{\mathrm{cloud}}\kappa
% \end{equation}
% To obtain the effect of an optically thick/thin clumpy medium, these clouds
% have to be sufficiently optically thick, not just marginally. So let us demand
% a minimal optical depth of $\tau=4$. This gives a minimal cloud radius of
% \begin{equation}
%   R\gtrsim \frac{2}{\rho_{\mathrm{Roche}}\kappa} = 0.11\frac{1}{\rho_{\mathrm{Roche}}}
%   = 0.15\frac{r^3}{M_{*}}
% \end{equation}
% in CGS units (for ALMA band 6). Note that this scales linearly with
% $\rho_{\mathrm{cloud}}/\rho_{\mathrm{Roche}}$, so that if we choose a much less
% ``critical'' cloud density of, say, $\rho_{\mathrm{cloud}}=0.1\,\rho_{\mathrm{Roche}}$,
% then the cloud should become 10$\times$ larger and 100$\times$ more massive.
% 
% If we want these clouds to be part of an unresolved clumpy medium, they should
% be substantially smaller than the pressure scale height of the disk, and
% substantially smaller than the ALMA band 6 resolution. Let us take the example
% of the source AS 209, with $M_{*}=0.9\,M_{\odot}$, and its two high-contrast
% rings at 74 au and 120 au, respectively. This gives minimal cloud
% radii of $R\gtrsim 4\times 10^{-3}\,\mathrm{au}$ and $R\gtrsim 22\times
% 10^{-3}\,\mathrm{au}$, respectively. Clouds of size similar to these minimal
% values are indeed much smaller than the beam width, and are therefore
% unresolved.  This does not mean that there cannot be any smaller
% clumps/clouds. It only means that any smaller clouds would be optically thin,
% and we would thus not see any effect of these clumps on the unresolved
% brightness: these clouds would have a spectral slope equal to that of an
% optically thin non-clumpy model. At any rate, this estimate shows that, at least
% in principle, it is possible that the rings we see in our data consist, in fact,
% of numerous unresolved optically thick clouds interdispersed with optically thin
% dust (or without any dust in between). Given that such an unresolved, but
% optically thick, set of clumps has a lower spectral slope
% $\alpha_{\mathrm{spec}}$ than an optically thin dusty medium, an analysis of the
% measured spectral slope variations over the radial profile of the rings {\em
%   may} give us some clues as to whether clumpiness occurs or not in these
% rings, even if we cannot resolve the clumps themselves. We would expect that
% within the ring the spectral slope is closer to $\alpha_{\mathrm{spec}}\simeq 2$
% (modulo deviations from Rayleigh-Jeans) than in the inter-ring region.
% 
% However, the grain size segregation mechanism, that larger grains are more
% concentrated in the ring than smaller grains, also tends to produce this
% signature. This is particularly the case if the big grains (which are closest to
% the ring peak) are larger than $\lambda/2\pi$ while the small grains are smaller
% than $\lambda/2\pi$. To distinguish between the roles of size seggregation and
% clumping, we need an independent way to probe the grain size. Perhaps the effects
% of self-scattering induced polarization \citep{2015ApJ...809...78K} may provide
% such an avenue. 
% 
% Even if we can distinguish between size-sorting and clumping, we still have to
% clarify whether the required total dust mass for the clumping scenario is
% reasonable, and whether the required 'filling factor' is consistent with
% simulations of the streaming instability. Both questions are related. Let us go
% back to the two prominent isolated rings of AS 209. The peak linear brightness
% temperature is 3.5 $\mathrm{K}$ for the inner one and 2.7 $\mathrm{K}$ for the
% outer one. In our model the disk midplane temperature is 13 $\mathrm{K}$ for the
% inner and 10 $\mathrm{K}$ for the outer ring, which translates (for
% $\lambda=0.13\,\mathrm{cm}$) into a linear brightness temperature of
% 8 $\mathrm{K}$ and 5.6 $\mathrm{K}$, respectively. If the disk were optically
% thick, we would thus expect to observe a linear brightness temperature of
% 8 $\mathrm{K}$ and 5.6 $\mathrm{K}$, which are only about 2 times higher. 
% 
% 
% This means that the filling factor of these clouds, as projected vertically
% toward the observer, must be quite high, of the order of 50\%. {\bf [At this
%     point we can calculate the minimum mass in dust needed for this]}
% 
% In principle the disk could be warmer than we assume, leading to a larger
% saturation brightness, and thus a lower required filling factor. But given that
% a 2$\times$ higher disk temperature requires a 16$\times$ higher irradiation
% flux, we expect at most a factor of 2 uncertainty in the required filling
% factor.
% 
% Checking whether this filling factor is consistent with the expectation of
% simulations of the streaming instability is not easy. Such data are not
% published for the models describing the streaming instability so far.  We can
% only do a ``by eye'' check of the figures in those papers ({\bf give a few
%   example papers here}). Such an exercise gives us the impression that the
% typical fraction of the simulation box that is covered by the dense cluster
% regions, as projected vertically, is typically substantially less than 50\%. If
% this impression is true, then the ring emission seen in our sample cannot be
% explained by a set of spatially unresolved optically thick clouds. That does not
% imply, however, that the streaming instability does not operate in these
% rings. It only implies that if it does, it will be hidden below an optically
% thick shroud of more extended dust.
% 
% 
% 
% 
% 
% \subsection{Under which conditions should the streaming instability set in?}
% While in Section \ref{sec-si-clump} we investigated if we can recognize the
% ongoing streaming instability in our data if it is taking place in the disk, in
% this Section we will compute, with our analytic dust trapping model, under which
% condition we would {\em expect} that the streaming instability will operate in
% the first place. This depends on the total dust mass locked in the trap, the
% grain sizes and the turbulent strength. To make analytic estimates it is easiest
% to assume a single grain size, instead of a grain size distribution.
% 
% {\bf FINISH THIS SECTION}
% 
% 




\section{Discussion}


\subsection{Condition for the streaming instability}
In the literature it is often mentioned that the streaming instability requires
a dust-to-gas surface density ratio of
$\Sigma_{\mathrm{d}}/\Sigma_{\mathrm{g}}\gtrsim 0.02$ or higher to operate {\bf
  XXXXXXXX}. This can, however, not be directly compared to our models, because
this value of $0.02$ was found for models without any pre-determined
turbulence. The turbulence in those models was induced by the streaming
instability itself or, if the streaming instability does not operate, by the
Kelvin-Helmholtz instability ({\bf XXXX}). In the analytic model of this
Section, on the other hand, we set the turbulence strength by hand, by setting
$\alpha_{\mathrm{turb}}$ to some value. In essence, we assume that there is
another source of turbulence, such as the magnetorotational instability ({\bf
  XXXX}) or the vertical shear instability ({\bf XXXX}), that determines the
mixing of the dust in the disk.

According to \citet{2005ApJ...620..459Y} the true criterion for the onset of the
streaming instability is the ratio of dust and gas {\em volume} densities
$\rho_{\mathrm{d}}/\rho_{\mathrm{g}}\gtrsim 1$. For a given surface density
ratio $\Sigma_{\mathrm{d}}/\Sigma_{\mathrm{g}}$, the midplane volume density
ratio, for a single grain species with midplane Stokes number $\mathrm{St}\ll
1$, depends on the turbulent strength as
\begin{equation}\label{eq-dtg-sig-vs-rho}
  \frac{\rho_{\mathrm{d}}}{\rho_{\mathrm{g}}}\simeq
  \left(1+\frac{\mathrm{St}}{\alpha_{\mathrm{turb}}}\right)
  \frac{\Sigma_{\mathrm{d}}}{\Sigma_{\mathrm{g}}}
\end{equation}
The criterion of $\Sigma_{\mathrm{d}}/\Sigma_{\mathrm{g}}\gtrsim 0.02$ mentioned
in the literature thus relates to the criterion
$\rho_{\mathrm{d}}/\rho_{\mathrm{g}}\gtrsim 1$ via the turbulent strength and
the Stokes number. Given that we do not compute the turbulent strength
but prescribe it, we should rely on the more fundamental volume density
criterion of \citet{2005ApJ...620..459Y} to assess whether the dust in our
model triggers the streaming instability or not.

In our analytic setup of Section {\bf CHECK} %\ref{sec-anmodel-degree-trapping-alpha-series},
the dust ring is about 5 times narrower than the gas bump. Assuming that
initially the dust-to-gas ratio was $0.01$, this implies a dust-to-gas surface
density ratio at the peak of the ring of $0.05$, which is in excess of the usual
$0.02$ value. But as long as $\mathrm{St}\lesssim 19\alpha_{\mathrm{turb}}$,
Eq.~(\ref{eq-dtg-sig-vs-rho}) shows that the midplane volume density ratio still
stays below unity and no streaming instability sets in.



\subsection{Can a resolved ring be in fact a blend of several unresolved rings?}
Measuring the width of rings with finite-resolution observations is, of course,
limited to ring widths that are larger than the ``beam''. However, even for
cases where the ring width clearly exceeds the beam size by a factor of a few or
more, one may ask the question: is the wide ring we see truly a single wide
ring, or could it also be made up of a series of spatially unresolved rings that
are blended into a single wide ring due to the beam convolution? This question
is particularly relevant for the very wide rings of HD 163296. It is, of course,
hard to answer in general, since we have no observational means to test this.

But from the perspective of particle trapping by a pressure bump this question
can be rigorously answered. A long-lived radial pressure perturbation in a
protoplanetary disk cannot be much narrower than about a pressure scale height
$h_p(r)$. {\bf [Perhaps we should add an appendix where we use the stability
    criteria to rigorously put limits on the width and depth of the rings (even
    a wide ring will, deep in the gaussian wings, become unstable), and
    link this to a leakage of dust out of the trap.]}

A dust ring produced by dust trapping in this pressure bump may become
rather narrow, dependent on a variety of parameters, as discussed in
Section \ref{sec-steady-state-analytic-trap-model}. But there can not be
more than a single such dust ring in each pressure bump.

For the wide rings of HD 163296 the pressure scale height, even under the most
optimistically low disk temperature (e.g.~10 K) the pressure scale height at
rings 1 and 2 are 2.4 au and 4.3 au, respectively, which correspond to FWHM
widths of 55 and 101 milliarcseconds, respectively (where we have multiplied by
2.355 to obtain the full-width at half-maximum corresponding to a gaussian with
$h_p$ as standard deviation width). Clearly the ALMA observations in band 6,
with FWHM beam size of 35 mas, spatially resolve the pressure scale height.
This means that the ring separation will be spatially resolved by ALMA,
ruling out the possibility that the wide rings are made up of a multitude of
narrow rings, at least in the dust trapping scenario.

{\bf [Add discussion of the plateau of Elias 24]}

\subsection{Interpretation of the millimeter flux in terms of grain size}
A common rule of thumb for interpreting millimeter and sub-millimeter
fluxes and intensities from thermal dust emission is that you ``observe
the grain size equal to the wavelength you observe at''. This rule is
based on the tendency of the dust opacity derived from a Mie calculation
to peak at wavelength around $\lambda\sim 2\pi a$, with $a$ the grain
radius. For $\lambda \ll 2\pi a$ the opacity becomes nearly constant
while $\lambda \gg 2\pi a$ it drops. This effect is even stronger for
the scattering opacity. The validity of this rule of thumb relies,
however, on the grain size distribution itself. For the simple
{\bf Ivezic et al.~(1997)} opacity law, it is only valid for
grain size distribution powerlaws obeying {\bf XXXXXXXXXXX}.
{\bf [TODO: Maybe here do an experiment with full opacity laws.]}




\begin{acknowledgements}
  The authors acknowledge support
  by the High Performance and Cloud Computing Group at the Zentrum f\"ur
  Datenverarbeitung of the University of T\"ubingen, the state of
  Baden-W\"urttemberg through bwHPC and the German Research Foundation (DFG)
  through grant no INST 37/935-1 FUGG. This research was initiated at the
  ``Stars, Planets and Galaxies'' meeting at the Harnackhaus in Berlin,
  April 2018. Part of this work was also funded by the DFG Forschergruppe
  FOR 2634 ``Planet Formation Witnesses and Probes: Transition Disks''.
\end{acknowledgements}


\begingroup
\bibliographystyle{aa}
\bibliography{ms}
\endgroup

\appendix

\section{Conventions on brightness}
\label{sec-conventions}
%
The intensity of an interferometrically produced image at (sub-)millimeter
wavelengths can be expressed in numerous ways. We will define our symbols here
to avoid confusion. Expressed in CGS units, the intensity $I_\nu$ at frequency
$\nu$ has dimension
$\mathrm{erg}\,\mathrm{cm}^{-2}\,\mathrm{s}^{-1}\,\mathrm{Hz}^{-1}\,\mathrm{ster}^{-1}$.
From an observational point of view it is convenient to express it instead in
Jansky/beam. We shall write this as $I_\nu^{\mathrm{jpb}}$. The conversion is
\begin{equation}
  I_\nu = 10^{-23}\,\frac{\pi(\mathrm{au}/\mathrm{pc})^2 b_{\mathrm{as}}^2}{4\ln(2)}\frac{1}{b^2_{\mathrm{as}}}I_\nu^{\mathrm{jpb}}
  = 3.755\times 10^{-13}\frac{1}{b^2_{\mathrm{as}}}I_\nu^{\mathrm{jpb}}
\end{equation}
where
\begin{equation}\label{eq-def-average-beam}
b_{\mathrm{as}}=\sqrt{b_{\mathrm{as}}^{\parallel}b_{\mathrm{as}}^{\bot}}
\end{equation}
is the position-angle-averaged beam full-width-at-half-maximum in units of
arcseconds. Alternatively it can be expressed as brightness temperature
$T_\nu^{\mathrm{b}}$ in Kelvin by solving the equation
\begin{equation}
I_\nu = B_\nu(T_\nu^{\mathrm{b}})
\end{equation}
where $B_\nu(T)=2h\nu^3c^{-2}/(\exp(h\nu/k_BT)-1)$ is the Planck function with $h$
Planck's constant, $k_B$ Boltzmann's constant and $c$ the light speed. The
relation between brightness temperature and intensity is non-linear, which is
inconvenient. As an alternative to the brightness temperature (henceforth called
``full brightness tempearture'') we therefore use the {\em linear} brightness
temperature $I_\nu^{\mathrm{lb}}$, also expressed in Kelvin, defined as
\begin{equation}
I_\nu^{\mathrm{lb}} = \frac{c^2}{2\nu^2k_B}I_\nu
  = 3.255\times 10^{36}\, \frac{1}{\nu^2}I_\nu
\end{equation}
The linear brightness temperature is thus a linear representation of the
intensity, which is why we use the symbol $I_\nu^{\mathrm{lb}}$ instead of
$T_\nu^{\mathrm{lb}}$. The full brightness temperature and the linear brightness
temperature become equal in the Rayleigh-Jeans regime. However, at the
wavelength we are working in (ALMA band 6 at $\lambda\simeq 0.13\,\mathrm{cm}$)
and the temperatures we are dealing with (several Kelvins to tens of Kelvins) we
are not in the Rayleigh-Jeans regime anymore, so the distinction between linear
and non-linear brightness temperature becomes important. 

The conversion between the full brightness temperature $T_\nu^{\mathrm{b}}$ and
the linear brightness temperature $I_\nu^{\mathrm{lb}}$ is given by
\begin{equation}\label{eq-convert-lin-to-full-bright}
\frac{h\nu}{k_BT_\nu^{\mathrm{b}}} = \ln\left(1+\frac{h\nu}{k_BI_\nu^{\mathrm{lb}}}\right)
\end{equation}
where in our case, with $\lambda=0.13\,\mathrm{cm}$, we have
$h\nu/k_B=11\,\mathrm{K}$.

The Planck function in units of Kelvin, to be used in conjunction with the
brightness, is then
\begin{equation}\label{eq-app-bnu-bright}
B_\nu^{\mathrm{lb}}(T) = \frac{h\nu}{k_B}\frac{1}{e^{h\nu/k_BT}-1}
\end{equation}
so that at high optical depth the brightness saturates to
$I_\nu^{\mathrm{lb}}=B_\nu^{\mathrm{lb}}(T)$. Throughout this paper we 
express the intensity as $I_\nu^{\mathrm{lb}}$. 



\section{Gauss fitting procedure}
\label{sec-gauss-fitting-procedure}
%
In this paper we study each dust ring individually, and try to understand it in
terms of the trapping of dust in a pressure trap. In Appendix
\ref{sec-steady-state-analytic-trap-model} we find that for a Gaussian pressure
bump the solution to the radial dust mixing and drift problem is, to first
approximation, also a Gaussian, albeit a narrower one. Fitting the dust trapping
model to the data can therefore be done in two stages: first fitting a Gaussian
radial profile to the observed rings, then interpreting these Gaussian fits
through the dust trapping model. In this Appendix we detail the procedure we
used to fit the radial intensity profiles of the rings with Gaussian profiles.

The radial intensity profiles (expressed as linear brightness temperature) were
extracted from the images using a procedure similar to that described by
\paperdsharphuangrings{}. This procedure involves the fitting of an ellipse to
describe the inclined ring shape, the deprojection into a ring, and the
averaging of the intensity along the ring. This averaging procedure enhances the
signal-to-noise ratio considerably, by a factor $\sqrt{N}$, where $N$ is the
number of beams that fit along the ring. We estimate the intrinsic noise simply
by computing the standard deviation along the ring. The resulting averaged
radial intensity profile $I_\nu^{\mathrm{lb}}(r)$ thus obtains also an error
estimate $\varepsilon(r)$, which is typically of the order of $\sim$1\% of
the peak intensity.

The rings display themselves as bumps in $I_\nu^{\mathrm{lb}}(r)$. We choose by eye
a radial domain around the bump where we believe a Gaussian description is justified.
The inner and outer radii of this domain are listed in Table \ref{tab-gauss-params}.
By choosing this domain we can select a specific ring to fit, which is not possible
when doing the fitting procedure in the uv-plane.

We now fit a Gaussian profile to this bump
\begin{equation}
I_\nu^{\mathrm{lb,gauss}}(r) = A\exp\left(-\frac{(r-r_0)^2}{2\sigma^2}\right)
\end{equation}
We use the code {\small\sf emcee} \citep{2013PASP..125..306F} to perform a
Markov Chain Monte Carlo procedure to find the set of parameters
$(A,r_0,\sigma)$ which have the highest likelihood. Given that the radial
sampling of $I_\nu^{\mathrm{lb,gauss}}(r)$ is about $N\simeq 70$ points per beam, we
have to multiply the error estimate of the datapoints by $\sqrt{N}$ before
feeding it into {\small\sf emcee}, otherwise the MCMC procedure would treat each
sampling point as an independent measuring point, which is not the case.

We show the corner plot of the fitting of ring 1 of AS 209 in
Fig.~\ref{fig-corner-gaussfit-as209-1}. The plots for the other rings are
similar, and are provided in the online material. {\bf TODO: PUT OTHER PLOTS
IN ONLINE MATERIAL.}

\begin{figure}
\centerline{\includegraphics[width=0.45\textwidth]{AS_209_ring1_cornerplot_gaussfit.pdf}}
\caption{\label{fig-corner-gaussfit-as209-1}The correletion plot for the Gauss
  fitting of ring 1 of AS 209, from the MCMC procedure described in Appendix
  \ref{sec-gauss-fitting-procedure}.}
\end{figure}

The most likely parameter values and their error estimates are given in
Table \ref{tab-gauss-fit-errors}.

\begin{table}
\begin{center}
\begin{tabular}{|cc|ccc|}
\hline
\hline
Source & Ring & $A$ & $r_0$ & $\sigma$ \\
\hline
AS 209     & 1 & $  3.637_{-0.047}^{+0.048}$ & $ 74.129_{-0.066}^{+0.063}$ & $  3.885_{-0.090}^{+0.098}$  \\
AS 209     & 2 & $  2.949_{-0.031}^{+0.030}$ & $120.309_{-0.068}^{+0.065}$ & $  4.651_{-0.111}^{+0.115}$  \\
Elias 24   & 1 & $  5.214_{-0.068}^{+0.066}$ & $ 76.851_{-0.092}^{+0.088}$ & $  5.104_{-0.172}^{+0.186}$  \\
HD 163296  & 1 & $  8.475_{-0.056}^{+0.056}$ & $ 67.403_{-0.055}^{+0.057}$ & $  7.370_{-0.098}^{+0.098}$  \\
HD 163296  & 2 & $  4.918_{-0.026}^{+0.027}$ & $ 99.862_{-0.061}^{+0.059}$ & $  5.741_{-0.116}^{+0.119}$  \\
GW Lup     & 1 & $  1.467_{-0.052}^{+0.055}$ & $ 85.178_{-0.259}^{+0.316}$ & $  4.882_{-0.402}^{+0.530}$  \\
HD 143006  & 1 & $  3.202_{-0.084}^{+0.082}$ & $ 41.080_{-0.169}^{+0.190}$ & $  5.133_{-0.320}^{+0.373}$  \\
HD 143006  & 2 & $  2.473_{-0.043}^{+0.044}$ & $ 64.831_{-0.248}^{+0.230}$ & $  8.029_{-0.501}^{+0.593}$  \\
\hline
\hline
\end{tabular}
\end{center}
\caption{\label{tab-gauss-fit-errors}The Gaussian fit values with their error
  estimates from the MCMC procedure described in Appendix \ref{sec-gauss-fitting-procedure}.}
\end{table}


\section{Gaussian mimicry: Detectability of mild optical depth effects}
\label{sec-mimicry-flat-topped}
%
As was discussed in Section \ref{sec-saturation-flat-top-profiles}, the
saturation effect of radiative transfer leads to Gaussian radial profiles of the
rings to be flattened at the peak by the $1-e^{-\tau_\nu}$ self-absorption
effect. The resulting flat-topped profile deviates from the optically thin
approximation (which is proportional to $\tau_\nu$) by a factor
$(1-e^{-\tau_\nu})/\tau_\nu$. This leads, for $\tau_\nu^{\mathrm{peak}}\gtrsim
0.6$, to a substantial reduction near the peak. However, for
$\tau_\nu^{\mathrm{peak}}\lesssim 1$, the {\em shape} of the profile stays
remarkably close to a Gaussian profile, albeit one that is wider and weaker than
the underlying $\tau_\nu(r)$. If the differences lie within the errorbars of the
observations, there is no way to tell the difference.

This `mimicry' (a non-Gaussian function posing as a Gaussian one) makes it hard
to exclude, from looking at the shape of the profile (it being not
`flat-topped'), that optical depth effects play a role. Given that much of our
analysis of the rings in terms of dust trapping relies on the optically thin
approximation (``what you see is what is there''), it is important to
quantify this. This is the topic of this appendix.

We do the following experiment: We start from a simple Gaussian function
$g(x)$ given by
\begin{equation}\label{eq-mimicry-underlying-gauss}
g(x) = \exp\left(-\frac{1}{2}x^2\right)
\end{equation}
where $x=(r-r_0)/w_d$. From this we generate the optical depth profile
\begin{equation}
\tau_\nu(x) = \tau_\nu^{\mathrm{peak}}\,g(x)
\end{equation}
for a given value of $\tau_\nu^{\mathrm{peak}}$. We then compute the function
\begin{equation}
  f(x) \equiv \frac{I_\nu^{\mathrm{lb}}(x)}{B_\nu^{\mathrm{lb}}(T_d)}
  = 1-e^{-\tau_\nu(x)}
\end{equation}
The results, for different values of $\tau_\nu^{\mathrm{peak}}$, are shown in
the left panel of Fig.~\ref{fig-mimicry-profiles}. As one can see, the
differences between $\tau_\nu(x)$ and $f(x)$ become quite substantial for
$\tau_\nu^{\mathrm{peak}}\simeq 0.7$.

\begin{figure*}
\centerline{\includegraphics[width=0.95\textwidth]{saturation_example_combined.pdf}}
\caption{\label{fig-mimicry-profiles}Demonstration of the Gaussian 'mimicry'
  effect discussed in Appendix \ref{sec-mimicry-flat-topped}. Left panel:
  Profiles before beam-convolution. Solid curves show the radiative transfer
  result $f(x)=1-e^{-\tau_\nu(x)}$, dotted curves show the underlying
  $\tau_\nu(x)$.  Middle panel is the same, but convolved with a beam that is
  half the width of the Gaussian profile of $\tau_\nu(x)$. Right panel: Solid
  lines are the same as for the middle panel, but the dashed lines are the best
  fit Gaussian curves (also convolved with the beam).}
\end{figure*}

For both $\tau_\nu(x)$ and $f(x)$ we now compute the beam-convolved versions:
\begin{equation}
  \tau_\nu^{\mathrm{conv}}(x) = \frac{1}{\sqrt{2\pi}\sigma_{\mathrm{beam}}}\int_{-\infty}^{+\infty}
  \tau_\nu(x')\,\exp\left(-\frac{(x-x')^2}{2\sigma_{\mathrm{beam}}^2}\right)dx'
\end{equation}
and likewise for $f(x)$, where $\sigma_{\mathrm{beam}}$ is the
standard-deviation beam width in $x$-units. In this experiment the width of the
underlying Gaussian `dust distribution' (Eq.~\ref{eq-mimicry-underlying-gauss})
stays always fixed at $\sigma_{\mathrm{gauss}}=1$, and the beam size is
expressed in relation to that. For the example of a beam that is half the width
of the Gaussian profile of $\tau_\nu(x)$ (corresponding roughly to ring 1 of AS
209, for instance) the results are shown in the middle panel of
Fig.~\ref{fig-mimicry-profiles}. This panel shows that the beam convolution
changes the results only by a little, at least for beams smaller than, or equal
to, the original width of the dust distribution. Given that all the rings of
this paper appear to be spatially resolved, we do not consider larger beams
here.

Now we try to find the best-fit Gaussian profile for $f(x)$. We do that by
constructing an alternative Gaussian function $\tilde g(x)$
\begin{equation}\label{eq-mimicry-mimicked-gauss}
\tilde g(x) = \tilde a\exp\left(-\frac{x^2}{2\tilde\sigma^2}\right)
\end{equation}
where, in contrast to the original $g(x)$ of
Eq.~(\ref{eq-mimicry-underlying-gauss}), the $\tilde a$ and $\tilde\sigma$ are
free parameters to fit. We subsequently obtain $\tilde \tau_\nu(x)$ and $\tilde
\tau_\nu^{\mathrm{conv}}(x)$ by the same procedure as above, then then use a
minimization procedure to find the values of $\tilde a$ and $\tilde \sigma$ for
which the difference to the function $f^{\mathrm{conv}}(x)$ is minimal.  We then
plot the resulting $\tilde \tau_\nu^{\mathrm{conv}}(x)$ as dashed lines in the
right panel of Fig.~\ref{fig-mimicry-profiles}. One can see that even for the
case of $\tau_\nu^{\mathrm{peak}}=3.0$ the Gaussian fit mimicks the real curve
remarkably (or perhaps disturbingly) well. Given the high signal to noise we
have in our azimuthally-averaged radial profiles (see
Fig.~\ref{fig-obs-gaussfits}), we can nevertheless exclude the case
$\tau_\nu^{\mathrm{peak}}=3.0$. But it is much harder to exclude the case
$\tau_\nu^{\mathrm{peak}}=1.2$, and impossible to exclude the case
$\tau_\nu^{\mathrm{peak}}=0.7$.

\begin{figure*}
\centerline{\includegraphics[width=0.95\textwidth]{gausssatur_ratios.pdf}}
\caption{\label{fig-mimicry-bestfitgauss-ratios}A quantitative analysis of
  the mimicry effect of Appendix \ref{sec-mimicry-flat-topped}, scanning
  the two parameters of the model: $(\sigma_{\mathrm{beam}},\tau_\nu^{\mathrm{peak}})$.
  The four panels show the four derived parameters explained in the text.
  Upper left shows the degree of overestimation of the width, upper right
  the degree of underestimation of the dust mass, bottom left the underestimation
  of the temperature, bottom right an estimate of the deviation of the Gauss fit
  to the real curve (the closer to unity, the better the mimicry, and therefore
  the worse the interpretation of the observations in terms of an optically thin
  model).
}
\end{figure*}

To quantify this a bit better, we have run the above models for a 2-D grid of
parameters $(\sigma_{\mathrm{beam}},\tau_\nu^{\mathrm{peak}})$, and computed, for
each pair of parameters, the following quantities:
\begin{eqnarray}
  {\cal W} &=& \tilde\sigma \\
  {\cal M} &=& \tilde a \\
  {\cal T} &=& f^{\mathrm{conv}}(x=0)\\
  {\cal F} &=& \tilde \tau_\nu^{\mathrm{conv}}(0)/f^{\mathrm{conv}}(x=0)
\end{eqnarray}
They are plotted in Fig.~\ref{fig-mimicry-bestfitgauss-ratios}. The ${\cal W}$
parameter tells us how much the mimicry will lead to an overestimation of the
true width of the dust distribution. The ${\cal M}$ parameter tells us how much we
would underestimate the dust mass by this effect. The ${\cal T}$ parameter tells
us the degree to which we underestimate the true linear brightness temperature
$B_\nu^{\mathrm{lb}}(T_d)$ (and by inference, the true dust temperature $T_d$)
if we would interpret the observed linear brightness temperature as if it were
optically thick (i.e.~by setting
$B_\nu^{\mathrm{lb}}(T_d)=I_{\nu}^{\mathrm{lb,obs}}$). For the case of no
convolution this becomes ${\cal T}=1-e^{\tau_\nu^{\mathrm{peak}}}$. Finally the
${\cal F}$ parameter gives the ratio of the mimicking Gauss function to the real
$f^{\mathrm{conv}}$ function, at $x=0$, as a representation of `how bad is the
mimicking'. The closer ${\cal F}$ is to 1, the better is the mimicking, meaning
the more difficult it is to distinguish the real from the mimicked curve.
If ${\cal F}-1$ is less than the relative error of the observations at the
peak of the curve, then they are indistinguishable.

These results show that up to $\tau_\nu^{\mathrm{peak}}\lesssim 1$ the
differences between the best-fit Gauss and the real curve are at most $\sim
1\cdots 2$ \%, depending on the beam size. The beam size has most influence
on ${\cal F}$, but less on the other parameters. That is because many of the
effects of beam smearing are cancelled out because we know the beam size and
can deconvolve. But the beam smearing does make a non-Gaussian profile more
Gaussian-shaped, hence the stronger effect on ${\cal F}$. 

Although the mimicry effect is bad news for the interpretation of the observed
radial ring profiles, the upper-left panel of
Fig.~\ref{fig-mimicry-bestfitgauss-ratios} shows that the effect on the derived
width of the dust ring remains modest. The fact that the widths of the rings in
our sample all seem to be spatially resolved by at least 2 beam widths can
therefore not be due to an unresolved dust ring being broadened by the optical
depth effect. To broaden by a factor of 2 (i.e.~${\cal W}=2$) you would need
an optical depth so large that the mimicry will clearly fail, i.e.\ we would
easily recognize the deviations from a Gaussian shape.



\section{Thermal emission from a thin dust layer with scattering}
\label{sec-oned-radtrans-model}
%
So far we have assumed that only the absorption opacity
$\kappa_\nu^{\mathrm{abs}}$, not the scattering opacity, matters. For optically
thin dust layers this is indeed appropriate. But we have seen in Sections
\ref{sec-opt-thin-analysis} and \ref{sec-optical-depth-effects} that the
optical depth are not that low. Furthermore, the CO line extinction data
of HD 163296 discussed by \paperdsharpisella{} suggest that the dust
layer has an extinction optical depth close to unity. Even if the absorption
optical depth is substantially below 1, the total extinction (absorption +
scattering) can easily exceed unity, if the grains are of similar size to
the wavelength. For $a\simeq \lambda/2\pi=0.13/2\pi = 0.02\,\mathrm{cm}$ the
albedo of the grain can, in fact, be as high as 0.8 {\bf CHECK}.

The inclusion of scattering complicates the radiative transfer equation
enormously. Strictly speaking a full radiative transfer calculation, for
instance with a Monte Carlo code, is necessary. However, in the spirit of
this paper we wish to find a simple approximation to handle this without
resorting to numerical simulation.

We will outline here a simple two-stream radiative transfer approach to the
problem. We will assume that the dust seen in the ALMA observations is located
in a geometrically thin layer at the midplane, so we can use the 1-D slab
geometry approach. We will assume that the scattering is isotropic. This
may be a bad approximation, especially for $2\pi a\gg \lambda$. To reduce
the impact of this approximation we replace the scattering opacity
$\kappa_\nu^{\mathrm{scat}}$
with
\begin{equation}
\kappa_\nu^{\mathrm{scat,eff}} = (1-g_\nu)\, \kappa_\nu^{\mathrm{scat}}
\end{equation}
where $g_\nu$ is the usual forward-scattering parameter (the expectation
value of $\cos\theta$, where $\theta$ is the scattering angle). According
to \citet{1978wpsr.book.....I}, this approximation works well in
optically thick media.

We will now follow the two-steam / moment method approach from
\citet{1986rpa..book.....R} to derive the solution to the
emission/absorption/scattering problem in this slab. The slab is put between
$z=-\tfrac{1}{2}\Delta z$ and $z=+\tfrac{1}{2}\Delta z$ and we assume a constant
density of dust between these two boundaries. The mean intensity $J_\nu(z)$
of the radiation field then obeys the equation
\begin{equation}
\frac{1}{3}\frac{d^2J_\nu(\tau)}{d\tau_\nu^2} = \epsilon_\nu \big(J_\nu-B_\nu(T_d)\big)
\end{equation}
where
\begin{equation}
  \tau_\nu=\rho_d\,(\kappa_\nu^{\mathrm{abs}}+\kappa_\nu^{\mathrm{scat,eff}}) z
  \equiv \rho_d\, \kappa_\nu^{\mathrm{tot}} z
\end{equation}
with $\rho_d$ being the dust density, and
\begin{equation}
\epsilon_\nu = \frac{\kappa_\nu^{\mathrm{abs}}}{\kappa_\nu^{\mathrm{abs}}+\kappa_\nu^{\mathrm{scat,eff}}}
\end{equation}
The boundary conditions at $z=\pm\tfrac{1}{2}\Delta z$ are
\begin{equation}
\frac{dJ_\nu}{d\tau_\nu} = \mp \sqrt{3}J_\nu
\end{equation}
This leads to the following solution:
\begin{equation}
  \frac{J_\nu(\tau_\nu)}{B_\nu(T_d)} =  1-b\,
    \left(e^{-\sqrt{3\epsilon_\nu}\left(\tfrac{1}{2}\Delta\tau-\tau_\nu\right)}+e^{-\sqrt{3\epsilon_\nu}\left(\tfrac{1}{2}\Delta\tau+\tau_\nu\right)}\right)
\end{equation}
where $\Delta\tau = \rho_d\, \kappa_\nu^{\mathrm{tot}} \Delta z$ and $b$ is
\begin{equation}
b = \left[(1-\sqrt{\epsilon_\nu})e^{-\sqrt{3\epsilon_\nu}\Delta\tau} + 1 + \sqrt{\epsilon_\nu}\right]^{-1}
\end{equation}
Given this solution for the mean intensity $J_\nu(\tau_\nu)$ we can now numerically integrate
the formal transfer equation along a single line of sight passing through the slab at an
angle $\theta$:
\begin{equation}\label{eq-fte-slab-with-scat}
  \mu\frac{dI_\nu(\tau_\nu)}{d\tau_\nu} = \epsilon_\nu B_\nu(T_d) + (1-\epsilon_\nu) J_\nu(\tau_\nu)
  -I_\nu(\tau_\nu)
\end{equation}
where $\mu=\cos\theta$. We start at $\tau_\nu=-\tfrac{1}{2}\Delta \tau$ with
$I_\nu=0$ and integrate to $\tau_\nu=+\tfrac{1}{2}\Delta \tau$. The resulting
$I_\nu^{\mathrm{out}}=I_\nu(\tfrac{1}{2}\Delta \tau)$ is the intensity that is
observed by the telescope. An approximation for $I_\nu^{\mathrm{out}}$ which
works well to within a few percent is the following modifed version of the
Eddington-Barbier approximation:
\begin{equation}\label{eq-modif-eddington-barbier}
I_\nu^{\mathrm{out}} \simeq \left(1-e^{-\Delta\tau/\mu}\right)\,S_\nu\left((\tfrac{1}{2}\Delta\tau-\tau_\nu)/\mu=2/3\right)
\end{equation}
where
\begin{equation}
S_\nu(\tau_\nu) = \epsilon_\nu B_\nu(T_d) + (1-\epsilon_\nu) J_\nu(\tau_\nu)
\end{equation}
is the source function. The results are shown in Fig.~\ref{fig-scatter-parscan}.

For small optical depths ($\Delta\tau\ll 1$) the role of scattering vanishes,
and the solution approaches: $I_\nu^{\mathrm{out}} \rightarrow \epsilon_\nu\Delta\tau
B_\nu(T_d)$.  This is the same limiting solution as when
$\kappa_\nu^{\mathrm{scat}}$ is set to zero but $\kappa_\nu^{\mathrm{abs}}$ is
kept the same. For high optical depth the outcoming intensity does not saturate
to the Planck function, but a bit below, if the albedo is non-zero. This is the
well-known effect that scattering makes objects appear cooler than they really
are.


\begin{figure}
\centerline{\includegraphics[width=0.47\textwidth]{scatter_parscan.pdf}}
\caption{\label{fig-scatter-parscan}The intensity $I_\nu$, in units of the
  Planck function, emerging from a slab seen face on, with total optical depth
  $\Delta\tau$, a constant temperature, and an albedo of
  $\eta_\nu=1-\epsilon_\nu$. See Appendix \ref{sec-oned-radtrans-model} for
  details. The solid lines are the results of numerical integration of
  Eq.~(\ref{eq-fte-slab-with-scat}). The dotted lines are the result of the
  modified Eddington-Barbier approximation
  (Eq.~\ref{eq-modif-eddington-barbier}).}
\end{figure}


\section{Steady-state dust distribution in a ringlike trap}
\label{sec-steady-state-analytic-trap-model}
%
\subsection{Analytic approximate solution of dust trapping}
\label{sec-analytic-model-of-trapping}
%
Let us consider a narrow gas ring around the star at radius $r_0$ with a
midplane pressure given by
\begin{equation}\label{eq-gaussian-pressure-bump-repeat}
p(r) = p_0 \exp\left(-\frac{(r-r_0)^2}{2w^2}\right)
\end{equation}
where $w\ll r_0$ is the parameter setting the width of this gaussian gas ring.
We assume that the gas is turbulent with turbulent diffusion coefficient
$D$. Dust grains get trapped in this ring, and the dust will acquire a
radial density profile that is in equilibrium between the radial dust drift
pointing toward the peak of the gas pressure and radial turbulent diffusion
pointing away from that position. The radial dust drift velocity is
\citep[see e.g.][]{2010A&A...513A..79B}:
\begin{equation}\label{eq-v-radial-drift}
  v_{\mathrm{dr}} = \frac{\mathrm{St}}{1+\mathrm{St}^{2}}
  \left(\frac{d\ln p}{d\ln r}\right)\frac{c_s^2}{\Omega_Kr}
\end{equation}
where $c_s$ is the isothermal sound speed and the Stokes number $\mathrm{St}$ is
defined as
\begin{equation}\label{eq-definition-stokes-number}
\mathrm{St} = \Omega_Kt_{\mathrm{stop}}
\end{equation}
where $t_{\mathrm{stop}}$ is the stopping time of the grains. The diffusion
coefficient for the dust is \citep{2007Icar..192..588Y}:
\begin{equation}
D_{\mathrm{d}} = \frac{D}{1+\mathrm{St}^2}
\end{equation}
We take $D$ to be equal to the turbulent viscosity $\nu$ divided by the
Schmidt number $\mathrm{Sc}$, which we usually set to $\mathrm{Sc}=1$.
We use the usual $\alpha$-prescription for the turbulence:
\begin{equation}
D=\frac{\nu}{\mathrm{Sc}}= \alpha_{\mathrm{turb}}\frac{c_s^2}{\mathrm{Sc}\,\Omega_K}
\end{equation}
If $D$ is sufficiently small, the dust will get concentrated into a ring with
width $w_d$ that is substantially smaller than the width of the gas ring
$w$. In the following, we will ignore any terms
arising from the curvature of the coordinates. The steady-state radial
dift-mixing equation for the dust then becomes, in its approximate form:
\begin{equation}
  \frac{d}{dr}
  \left(\Sigma_{\mathrm{d}}v_{\mathrm{dr}}-D_{\mathrm{d}}\frac{d\Sigma_{\mathrm{d}}}{dr}\right) = 0
\end{equation}
Integrating this equation once, with integration constant zero (which
amounts to a zero net radial flux), yields
\begin{equation}\label{eq-drift-mix-equil-eq}
  \Sigma_{\mathrm{d}}v_{\mathrm{dr}} = D_{\mathrm{d}}\frac{d\Sigma_{\mathrm{d}}}{dr}
\end{equation}
From Eqs.(\ref{eq-v-radial-drift},\ref{eq-gaussian-pressure-bump-repeat}) we can express
$v_{\mathrm{dr}}$ as
\begin{equation}
v_{\mathrm{dr}} = -\left(\frac{c_s^2}{w^2\Omega_K(\mathrm{St}+\mathrm{St}^{-1})}\right)(r-r_0)
\end{equation}
With this expression we can solve Eq.~(\ref{eq-drift-mix-equil-eq}) for
$\Sigma_{\mathrm{d}}$, leading to the following simple analytic solution to the dust
trapping problem:
\begin{equation}\label{eq-analytic-sol-radial-trapping}
\Sigma_{\mathrm{d}}(r) = \Sigma_{\mathrm{d0}} \exp\left(-\frac{(r-r_0)^2}{2w_{\mathrm{d}}^2}\right)
\end{equation}
with 
\begin{equation}
  w_{\mathrm{d}} = w\, \sqrt{\frac{\Omega_KD_{\mathrm{d}}(\mathrm{St}+\mathrm{St}^{-1})}{c_s^2}}
  = w\,\sqrt{\frac{\alpha_{\mathrm{turb}}}{\mathrm{Sc}\,\mathrm{St}}}
\end{equation}
The normalization constant $\Sigma_{\mathrm{d0}}$ can be approximately expressed
in terms of the total dust mass trapped in the pressure bump:
\begin{equation}
  M_d = 2\pi \int_0^\infty \Sigma_{\mathrm{d}}(r) rdr \simeq
  2\pi r_0 \int_0^\infty \Sigma_{\mathrm{d}}(r)dr
\end{equation}
which leads to
\begin{equation}
\Sigma_{d0}\simeq \frac{M_d}{(2\pi)^{3/2}\, r_0\,w_{\mathrm{d}}}
\end{equation}
The approximation is best for narrow dust rings.

Note that this analytic solution is only valid as long as
$\alpha_{\mathrm{turb}}\ll \mathrm{Sc}\,\mathrm{St}$, or in other words as long
as $w_{\mathrm{d}}$ is substantially smaller than $w$.  This solution is, in
fact, the radial version of the vertical settling-mixing equilibrium solutions
of \citet{1995Icar..114..237D}.

Unfortunately, the condition that $\alpha_{\mathrm{turb}}\ll
\mathrm{Sc}\,\mathrm{St}$ (and equivalently $w_d\ll w$) is easily broken for
small grains and/or non-weak turbulence. This is similar to the case for
vertical settling and mixing: Small grains tend not to settle below a few
pressure scale heights. They settle down to some critical $z_{\mathrm{settle}}$,
below which they are well mixed with the gas, and above which they are strongly
depleted \citep{2004A&A...421.1075D}. \citet{2009A&A...496..597F} give an
analytic solution to the settling-mixing problem (their Eq.~19) that reproduces
the solutions of \citet{1995Icar..114..237D} for $\alpha_{\mathrm{turb}}\ll
\mathrm{Sc}\,\mathrm{St}$, but at the same time remains valid also for
$\alpha_{\mathrm{turb}}\gtrsim \mathrm{Sc}\,\mathrm{St}$, as long as
$\alpha_{\mathrm{turb}}$, $\mathrm{Sc}$ and $\mathrm{St}$ remain constant with
height above the midplane.

Given the similarity between the radial dust trapping problem and the vertical
settling problem, one can show that the radial version of the solution of
\citet{2009A&A...496..597F} reads:
\begin{equation}\label{eq-analytic-sol-radial-trapping-better}
  \Sigma_{\mathrm{d}}(r) = \Sigma_{\mathrm{d0}} \exp\Bigg[
    -\frac{\mathrm{Sc}\,\mathrm{St}}{\alpha_{\mathrm{turb}}}
     \left(\exp\left(\frac{\Delta r^2}{2w^2}\right)-1\right)
    -\frac{\Delta r^2}{2w^2}\Bigg]
\end{equation}
where we defined $\Delta r$ as
\begin{equation}
\Delta r \equiv (r-r_0)
\end{equation}
The solution Eq.~(\ref{eq-analytic-sol-radial-trapping-better}) is valid for
any value of $\alpha_{\mathrm{turb}}/\mathrm{Sc}\,\mathrm{St}$, as long as
$\alpha_{\mathrm{turb}}$, $\mathrm{Sc}$ and $\mathrm{St}$ remain constant along
the radial width of the dust trap, and as long as $w\ll r_0$, to prevent
geometric terms from the cilindrical coordinates from dominating.  One can
easily verify that Eq.~(\ref{eq-analytic-sol-radial-trapping-better}) reproduces
the simpler Gaussian solution Eq.~(\ref{eq-analytic-sol-radial-trapping}) for
$\Delta r\ll w$. One can also verify that for $\alpha_{\mathrm{turb}}\gg
\mathrm{Sc}\,\mathrm{St}$ the shape of $\Sigma_{\mathrm{d}}(r)$ follows the
shape of the gas pressure profile $p(r)$ (Eq.~\ref{eq-gaussian-pressure-bump-repeat}).

The solution Eq.~(\ref{eq-analytic-sol-radial-trapping-better}), like its
simplified version Eq.~(\ref{eq-analytic-sol-radial-trapping}), assumes that the
dust is always near the midplane of the disk, so that one can ignore the
vertical extent, and thereby ignore any variations in $\mathrm{St}$. In other
words, we assume that the vertical scale height of the dust layer $h_d$ is much
less than the gas pressure scale height $h_p$. As it turns out, this condition
$h_d\ll h_p$ is satisfied if $\alpha_{\mathrm{turb}}\ll
\mathrm{Sc}\,\mathrm{St}$, which is the same condition that leads to $w_d\ll w$
(at least for $\mathrm{St}\ll 1$, which is what we assume).
In fact, the radial dust ring width compared to the gas ring width is the
same as the vertical dust layer thickness compared to the gas disk thickness:
\begin{equation}
\frac{w_d}{w} = \frac{h_d}{h_p}
\end{equation}
In other words: the dust behaves radially the same way as vertically. The
vertical dust settling can be regarded as dust trapping in vertical direction,
or the radial dust trapping can be seen as radial settling of dust toward
the pressure peak.

This also means that whenever the simplified Gaussian analytic solution
(Eq.~\ref{eq-analytic-sol-radial-trapping}) ceases to be valid, so does the
simplified vertical Gaussian solution of \citet{1995Icar..114..237D}. Or in
other words, whenever the more sophisticated analytic solution
Eq.~(\ref{eq-analytic-sol-radial-trapping-better}) should be used instead of the
simplified version, the assumption that the dust is settled near the midplane
ceases to hold. Strictly speaking this invalidates
Eq.~(\ref{eq-analytic-sol-radial-trapping-better}), since this was derived under
the assumption that the dust is all located near the midplane. But given that
most of the dust mass is still close to the midplane, even for
$\alpha_{\mathrm{turb}}\gtrsim \mathrm{Sc}\,\mathrm{St}$, we are confident that
Eq.~(\ref{eq-analytic-sol-radial-trapping-better}) still remains approximately
valid.

\begin{figure}
\centerline{\includegraphics[width=0.47\textwidth]{analytic_trap_example.pdf}}
\caption{\label{fig-analytic-model-example}The analytic solution shown for three
  different grain sizes: small ($0.1\,\mu\mathrm{m}$), middle middle
  ($10\,\mu\mathrm{m}$) and large ($1000\,\mu\mathrm{m}$). The normalizations of
  the three curves are arbitrary, optimized for clarity. The solid (blue) curves
  show the full analytic solution
  (Eq.~\ref{eq-analytic-sol-radial-trapping-better}).  The dotted (green) curves
  show the simplified analytic solution
  (Eq.~\ref{eq-analytic-sol-radial-trapping}). The dashed (red) curves show the
  dust profile {\em if} the dust-to-gas ratio would be constant, which shows the
  shape of the gas pressure bump. The absense of the green dotted curve for the
  case of $1000\,\mu\mathrm{m}$ grains (the narrowest peak) is because for these
  large grains the simple analytic solution nearly perfectly overlaps with the
  full analytic solution. The parameters are chosen for AS 209, ring 1,
  with $M_{*}=0.912\,M_{\odot}$, $T_{\mathrm{gas}}=12.9\,\mathrm{K}$. The gas
  surface density is taken to be $5\,\mathrm{g}/\mathrm{cm}^2$ at the position
  of the ring, the turbulence parameter was set to
  $\alpha_{\mathrm{turb}}=2\times 10^{-4}$, the dust material density was chosen
  to be $2\,\mathrm{g}/\mathrm{cm}^3$ consistent with slightly porous silicate
  dust aggregates.
  {\bf [CHECK IF THIS IS CONSISTENT WITH THE OPACITY ESTIMATES, IN PARTICULAR
  THE RICCI OPACITY.]}
  The standard-deviation width of the gas pressure bump was set
  to $w=9.7\,\mathrm{au}$ which equals two times the gas pressure scale height.}
\end{figure}

The typical shapes of the analytic dust surface density profiles is shown in
Fig.~\ref{fig-analytic-model-example} for three different grain sizes from small
to large. For the largest grains, which are the most strongly trapped near the
pressure peak, and hence have the narrowest peak in
Fig.~\ref{fig-analytic-model-example}, the full analytic solution
(Eq.~\ref{eq-analytic-sol-radial-trapping-better}) nearly perfectly overlaps
with the simplified (Gaussian) analytic solution
(Eq.~\ref{eq-analytic-sol-radial-trapping}). For large grains the simplified
analytic solution is therefore sufficiently accurate. But in this parameter
regime the dust rings are much narrower than the gas pressure bump ($w_d\ll w$).

For the tiny grains, which are well coupled to the gas, and hence have the
widest peak in Fig.~\ref{fig-analytic-model-example}, the dust trapping is very
inefficient. Between 60 and 90 au the dust surface density follows nearly
perfectly the shape of the gas bump, i.e.\ keeping the dust-to-gas ratio nearly
constant. Only for $r\lesssim 60\,\mathrm{au}$ and $r\gtrsim 90\,\mathrm{au}$
the full solution drops below a constant dust-to-gas ratio. To good approximation
one can thus say that a dust ring made up of these tiny grains has a width
that equals the width of the gas pressure bump. Note that in this regime the
simplified analytic model (Eq.~\ref{eq-analytic-sol-radial-trapping}) fails
miserably, as can be seen by the nearly horizontal green dotted curve in
Fig.~\ref{fig-analytic-model-example}.

The intermediate size grains, corresponding to the intermediate curve (marked
with $100\,\mu\mathrm{m}$) in Fig.~\ref{fig-analytic-model-example}, have a
behavior between the above two extremes. The simplified analytic solution is a
bit wider than the more accurate solution. The shape of the more accurate
solution deviates from Gaussian. However, the deviation from Gaussian shape of
the model is opposite to the deviation from Gaussian shape in the observations
(Fig.~\ref{fig-obs-gaussfits}). The observed data exceed the best-fit Gaussian
in the flanks, while the model drops below the best-fit Gaussian in the flanks.

Whether this deviation from Gaussian is detectable in the ALMA data is,
however, not so clear. For intermediate size grains the non-Gaussian profile
can be easily mistaken for a Gaussian profile with a narrower width than
both the gas bump and the simple analytic solution. 







\subsection{Resulting profiles for a size distribution}
\label{sec-anmodel-profiles}
%

{\bf [THIS SECTION NEEDS HEAVY EDITING]}

The analytic model of the dust trapping for a single grain size speaks for
itself. But how does it look for a grain size distribution? Since the trapping
width $w_{\mathrm{d}}$ for the different grain sizes is different, we can only
impose the grain size distribution for the total (radially integrated) dust
mass. We define the size distribution according to the following powerlaw
\begin{equation}
m(a)\frac{dN}{d\ln a} = \frac{dM}{d\ln a} \propto a^{q}
\end{equation}
where $a$ is the grain size, $m(a)$ the corresponding grain mass, $N$ the
cumulative particle number and $M$ the cumulative dust mass. The parameter $q$
is the size distribution powerlaw coefficient, and it is $q=1/2$ for the usual
MRN distribution (this corresponds to $dN/da\propto a^{-3.5}$). We also need to
define limits $a_{\mathrm{min}}$ and $a_{\mathrm{max}}$. The size distribution
is then normalized such that its integral over $d\ln(a)$ is the total dust mass
$M$. The radial surface density solution,
Eq.~(\ref{eq-analytic-sol-radial-trapping}), then becomes:
\begin{equation}\label{eq-analytic-sol-radial-trapping-sizedistr}
\frac{d\Sigma_{\mathrm{d}}(r)}{d\ln a} = \frac{1}{(2\pi)^{3/2}r_0
    w_{\mathrm{d}}(a)}\frac{dM}{d\ln a}
\exp\left(-\frac{(r-r_0)^2}{2w_{\mathrm{d}}(a)^2}\right)
\end{equation}

In Fig.~\ref{fig-anmodel-distribution-sigma} we show the radial profiles for an
MRN size distribution between $a_{\mathrm{min}}=10^{-2}\,\mathrm{cm}$ and
$a_{\mathrm{min}}=1\,\mathrm{cm}$ and a total dust mass of $M=1.3\times
10^{-5}\,M_\odot$. The dust material density is taken to be
$\xi_{\mathrm{dust}}=2\,\mathrm{g}/\mathrm{cm}^3$. We sample this size
distribution with 10 grain sizes, evenly distributed in $\ln(a)$. These are the
10 curves in Fig.~\ref{fig-anmodel-distribution-sigma}. These surface densities
are defined as being the local grain size distribution integrated over the width
of the grain size bin:
$\Sigma_{\mathrm{bin}}=\int_{\mathrm{bin}}(d\Sigma_{\mathrm{dust}}/da)da$. The
stellar mass is $M_{*}=0.9\,M_\odot$ and we set $\mathrm{Sc}=1$.

{\bf MUST ALSO SPECIFY GAS DENSITY, TO GET STOKES NUMBERS. MUST SAY
(AND CHECK) THAT EPSTEIN REGIME.}

\begin{figure}
\centerline{\includegraphics[width=0.5\textwidth]{analytic_trapping_sigma_profiles.pdf}}
\caption{\label{fig-anmodel-distribution-sigma}Results of the analytic dust
  trapping model for an MRN powerlaw grain size distribution between
  $a=0.01\,\mathrm{cm}$ and $a=1\,\mathrm{cm}$. The gas pressure bump has a FWHM
  of 70 au and a peak at $r=77\,\mathrm{au}$. Shown here are the surface density
  profiles for 10 logarithmically binned grain sizes. The curve with the
  narrowest peak corresponds to a grain size of $a=1\,\mathrm{cm}$, while that
  with the widest peak (and lowest maximum value) corresponds to a grain size of
  $a=0.01\,\mathrm{cm}$. The intermediate curves correspond to $a=0.0167$,
  0.0278, 0.0464, 0.0774, 0.1291, 0.2154, 0.3593 and 0.5995 $\mathrm{cm}$
  respectively. The dashed curve shows the total surface density of the dust,
  which is, by definition of the bin-integrated surface density, the sum of all
  10 curves. For a description of the model setup and model parameters, see
  Section \ref{sec-anmodel-profiles}.}
\end{figure}

As expected, the largest grains are concentrated the most and the smallest ones
the least. The largest grains, however, have the largest peak value of their
surface density at the exact location of the gas pressure peak.

\begin{figure}
\centerline{\includegraphics[width=0.5\textwidth]{analytic_trapping_tau_profiles.pdf}}
\caption{\label{fig-anmodel-distribution-tau}As Fig.~\ref{fig-anmodel-distribution-sigma},
  but now the vertical optical depth at $\lambda=1.3\,\mathrm{mm}$ (ALMA band 6)
  is shown. The opacity model of \citet{1997MNRAS.291..121I} was used to compute these
from the size distribution.} 
\end{figure}

When it comes to the contribution to the optical depth, however, the situation
looks a bit different. Let us assume a very simple dust opacity law, which
roughly mimicks the real behavior of dust opacities: the model of
\citet{1997MNRAS.291..121I}, where we take the scattering opacity zero for
simplicity. In this simple but illustrative opacity model the cross
section of a grain with radius $a$ is the geometric cross section $\pi a^2$
for $2\pi a>\lambda$ (with $\lambda$ the wavelength), while it is reduced by
a factor $(2\pi a/\lambda)$ for $2\pi a<\lambda$. This means that for
$2\pi a>\lambda$ the opacity (that is, the cross section per unit mass)
is $\kappa=3/4\xi_{\mathrm{dust}} a$ where $\xi_{\mathrm{dust}}$ is the material
density, here taken to be $\xi_{\mathrm{dust}}=2\,\mathrm{g}/\cm^3$. For
$2\pi a<\lambda$ this is reduced by a factor $(2\pi a/\lambda)$ and
thus becomes independent of $a$.

It can be seen from Fig.~\ref{fig-anmodel-distribution-tau} that at the peak of
the trap all grain sizes contribute equally to the optical depth. That is
because, with our opacity model, the larger grains (the ones with $2\pi a >
\lambda$) have an opacity equal to their geometric cross section per unit dust
mass, which goes as $1/a$. The smallest grains (the ones with $a=0.01$ and
$a=0.167$) have $2\pi a < \lambda$, which means that their opacity remains
constant with $a$. Hence, for these two curves in
Fig.~\ref{fig-anmodel-distribution-tau} the contribution to the peak optical
depth drops below those of the other grains. Of course, with a more realistic
opacity model this can change somewhat, but the rough principle remains valid.

More important is to notice that the relative contributions of the grain sizes
to the peak optical depth may change if a different grain size distribution is
chosen. Here we chose the MRN size distribution ($q=1/2$), but that was a rather
arbitrary choice. Given the complexity of the dust growth processes in
protoplanetary disks, it is not at all said that the MRN distribution should be
considered a preferred one. The reader may experiment with other values of
$q$ or entirely different grain size distribution shapes altogether.



% \subsection{Effect of the degree of trapping on the appearance of the ring}
% \label{sec-anmodel-degree-trapping-alpha-series}
% %
% With this simple analytic dust trapping model we can investigate how the degree
% of trapping affects the appearance of the dust ring. For a model with a single
% grain size this is easy to predict: if the turbulence is weak enough that the
% gaussian profile of $\Sigma_{\mathrm{d}}(r)$ is narrower than the ALMA beam,
% then we observe a gaussian profile with the width of the beam. Even if the dust
% in this narrow ring would be optically thick, the observed ring would have
% sub-Planckian brightness due to beam-dilution. For stronger turbulence,
% $\Sigma_{\mathrm{d}}(r)$ is spread out wider, and we observe a gaussian profile
% with the width of the dust profile itself. As in the case of the strongly
% trapped dust, also the wide rust ring has a Gaussian profile, except that if it
% is optically thick, the observed profile is flattened at the top due to
% saturation to the Planck function.
% 
% The situation gets more complex if we have a grain size distribution. For a
% given turbulence strength $\alpha_{\mathrm{turb}}$ some grains may be
% concentrated into a ring that is narrower than the ALMA beam, while others may
% be more spread out. The strongly concentrated grain population produces a cusp
% in the brightness, which will be partly smeared out by the beam convolution,
% while the weakly concentrated grain population produces a broad underlying
% brightness bump.
% 
% To see how this looks in practice we set up a realization of our analytic
% model. The peak of the pressure is put at $r_{0}=77\,\mathrm{au}$ (corresponding
% to the inner of the two main rings of AS 209), and the width is chosen to be
% $w=30\,\mathrm{au}$ (corresponding to a FWHM of 70 au).  Given that the ALMA
% beam in band 6 has a FWHM of about 35 milliarcseconds, corresponding to 4.4 au
% for the distance to AS 209, the gas bump is much wider than the beam. The two
% main dust rings of AS 209 are about 3 beams wide, so that for our wide gas bump
% model the dust is strongly concentrated. This requires a low turbulent
% $\alpha_{\mathrm{turb}}$ or large grains. As in Section
% \ref{sec-anmodel-profiles} we choose a grain size distribution between
% $a_{\mathrm{min}}=10^{-2}\,\mathrm{cm}$ and $a_{\mathrm{min}}=1\,\mathrm{cm}$,
% following the usual MRN powerlaw of $q=1/2$ and a total dust mass of
% $M=1.3\times 10^{-5}\,M_\odot$. The dust material density is taken to be
% $\xi_{\mathrm{dust}}=2\,\mathrm{g}/\mathrm{cm}^3$. We sample this size
% distribution with 10 grain sizes, evenly distributed in $\ln(a)$. For the disk
% temperature we choose a simple powerlaw $T(r)=200\,(r/\mathrm{au})^{-1/2}$. The
% stellar mass is $M_{*}=0.9\,M_\odot$, and the source distance is taken to be
% $d=126\,\mathrm{pc}$. All these setting are taken to represent the inner of the
% two main high-contrast rings of the source AS 209. We set $\mathrm{Sc}=1$.

\begin{figure}
\centerline{\includegraphics[width=0.5\textwidth]{analytic_trapping_series_alpha.pdf}}
\caption{\label{fig-anmodel-series-alpha}Results of the analytic dust trapping
  model for various turbulent strengths, for an MRN powerlaw grain size
  distribution between $a=0.01\,\mathrm{cm}$ and $a=1\,\mathrm{cm}$. The gas
  pressure bump has a FWHM of 70 au and a peak at $r=77\,\mathrm{au}$. Shown
  here is the predicted brightness temperature at a wavelength
  $\lambda=1.3\,\mathrm{mm}$ before (dashed) and after (solid) convolution with
  the ALMA beam, assuming a source distance of $d=126\,\mathrm{pc}$. For a
  description of the model setup and model parameters, see Section
  \ref{sec-anmodel-degree-trapping-alpha-series}.}
\end{figure}
 
% The results for several values of $\alpha_{\mathrm{turb}}$ are shown in
% Fig.~\ref{fig-anmodel-series-alpha}. In the top panel, representing weak
% turbulence, most of the dust is trapped into a ring more narrow than the
% beam. A clear cusp can be seen in the un-convolved brightness profile,
% which is due to the strongly trapped large grains of the size distribution.
% However, the convolution washes out this cusp for the most part. This
% situation is equivalent to the simple case of a single dust species being
% trapped to a narrow unresolved ring. The only difference is that for a single
% dust size the ring is either optically thick or optically thin, while in the
% multi-size case this can be a mix. That does not make a difference for the
% observed brightness profile, which is a Gaussian either way.
% 
% For the case of the strongest turbulence, the bottom panel in
% Fig.~\ref{fig-anmodel-series-alpha}, the cusp is still seen in the un-convolved
% profile, but now the convolution does not wash it out completely. The resulting
% convolved brightness profile is clearly non-Gaussian as a result. It has a
% quasi-Gaussian cusp with broad wings. When seen with ALMA such a cuspy profile
% (narrow at the peak, broad in the wings) should be recognizable.
% 
% These findings show that if the turbulent strength is weak, the predicted dust
% ring width equals the ALMA beam width, and the brightness shape should have the
% shape of the beam profile (here assumed to be Gaussian). For stronger
% turbulence, the dust ring becomes indeed broader than the ALMA beam, but then
% the shape of the observed brightness profile may teach us something about the
% grain size distribution. Roughly speaking, if the largest grains of the size
% distribution still get trapped in a narrow ring, this shows up as a cusp in the
% brightness profile. If the smallest grains of the grain size distribution are
% too small, the predicted brightness profile would be too broad compared to the
% observed ring width. In other words: if the observed brightness profile of a
% observed ring in given source is clearly radially resolved, and if it has a
% Gaussian shape, then the most likely fit to our analytic model is that of
% a narrow size distribution (quasi a single grain size) with the right turbulent
% strength to fit the observed ring width. A broader size distribution, with the
% appropriately chosen turbulent strength, would yield a cuspy profile, not a
% Gaussian one.
% 
% The two high-contrast rings of AS 209 both appear to fit a Gaussian profile with
% 2.5 times the width of the ALMA beam (Fig.~\ref{fig-as209-rings-gauss-fit}). In
% both cases it is therefore likely, according to this analysis, that the dust
% that is trapped in these rings has a narrow size distribution instead of a broad
% MRN powerlaw distribution.

\subsection{Spectral slope variations across the ring}
\label{sec-anmodel-slope-variations}
%

{\bf [TO DO]}

\subsection{Leakage and retention time of the dust in the traps}
\label{sec-leakage}
%
In our analytic model we have assumed that the pressure bump has a perfect
Gaussian shape, meaning that the further away one goes from the peak of the
trap, the steeper the pressure gradient (and thereby the faster the dust drift)
becomes. This leads to perfect trapping: no dust can leak out of the trap.
In reality the pressure bump has a finite radial width, and smoothly
connects to the global disk structure. There will be a maximum pressure
gradient at some distance away from the peak, and beyond that the pressure
gradient declines again. If the width of the trapped dust is much narrower
than this distance, virtually no dust will get lost out of the trap. But
for small grains and/or strong turbulence, the width of the dust profile may
be large enough that at the point of largest pressure gradient there is
still a substantial amount of dust. Then the dust will leak out of the
trap. 

{\bf First make estimates based on current model. Then add the gas radial
velocity by accretion.}


\subsection{Stability of gas ring}
\label{sec-gas-ring-stability}

{\bf Here we must discuss how far into the wings of the gaussian is
  stable against RWI and other instabilities. Solberg-Hoiland stability
  criterion. A Gauss may not be very stabile in its flanks. This may limit
the applicability of the full analytic solution.}


\subsection{Model degeneracies}
\label{sec-anmodel-degeneracies}
%
When trying to infer the effectiveness of dust trapping within the rings with
the analytic model of Section \ref{sec-analytic-model-of-trapping}, one notices
that there are degeneracies in the possible combinations of parameters. Most
obvious is the degeneracy between grain size and turbulence: smaller grains
and stronger turbulence may yield the same dust trapping width. But also the
width of the gas pressure bump can be compensated: a narrower bump but stronger
turbulence (or smaller grains) will yield the same ring width.

{\bf [CONTINUE THIS; SOME DEGENERACIES FOLLOW DIRECTLY FROM THE EQUATIONS.]}


%\subsection{Model fits to the ALMA data}
%\label{sec-anmodel-fits-to-data}
%%
%We now apply this model to the isolated rings in our ALMA sample (Section
%\ref{sec-data}). Given the degeneracies of the model in many of the parameters,
%we perform the fitting procedure in three parameters, while fixing the others
%to a particular choice of values. 
%
%
%
%{\bf [CONTINUE THIS; SHOW THE FIT FIGURES; SHOW THE MCMC CORNER PLOTS]}
%
%
%
%
%{\bf For HD 163296 we have constraints from the sub/superkepler data of
%Teague. This limits the width of the gas bump.}





\subsubsection{Could the rings be optically thick at the center?}
\label{sec-anmodel-fits-optically-thick}
%
All rings in our sample show sub-Planckian brightness, at least for the disk
temperature assumed in our model. We have calculated the disk midplane temperature
using a simple ``flaring angle recipe''. If the disks are, however, substantially
colder (by up to a factor of 2.5 to 3) then the peak flux of the rings might be
consistent with saturated emission, i.e.\ emission expected from an optically thick
disk. So the question is: how sure are we that the rings are sub-Planckian? 

To reduce the midplane temperature by a factor of $2$ the irradiation flux must
be reduced roughly by a factor of $2^4=16$. Within the flaring angle recipe this
would lead to unrealistically low flaring angle. Presently we assumed $\varphi=0.02$,
so that it would have to reduce to $\varphi=0.0013$, which is so low that we can no
longer trust the flaring angle recipe. However, maybe the outer disk lies in the
shadow of the inner disk. To see whether this can reduce the temperature to low
enough values, we perform a proper radiative transfer calculation. The results are
shown in Appendix ({\bf XXXX}). It shows that {\bf XXXXX}.

There is another issue, however. If the emission saturates to the optically thick
brightness, i.e.\ to the Planck function, then the radial profile of the ring should
be flat-topped. This is clearly not what we see in the rings in our sample, with
the possible exception of the ring of GW Lup which shows a very slight hint of that.
In general, however, it seems that the simple picture of collecting so much dust into
the ring that the ring becomes optically thick does not seem to explain the observed
radial brightness profiles. On the other hand, optical thickness can also occur
on smaller scales, as we will discuss in Section \ref{sec-si-clump}.


\section{Computing the dust mass trapped in the rings}
\label{sec-computing-dust-mass}
%

{\bf THIS APPENDIX HAS TO BE COMPLETELY REWRITTEN: FOCUS ON OPACITY
  UNCERTAINTY INSTEAD OF COMPUTATION OF DUST MASS, BECAUSE THE LATTER IS
NOW IN THE MAIN TEXT.}
% 
% We can compute the dust mass trapped in each of the rings by integrating over
% the measured brightness temperature profiles. However, the profiles do not go
% down to zero far way from the peak, and so it is not well defined what belongs
% to the ring and what not. Instead, we will compute the dust mass from the
% Gaussian profile that is fitted to the ring emission. This will slightly
% underestimate the dust mass, because the excess emission in the flanks of the
% profiles (see Fig.~\ref{fig-obs-gaussfits}) will not be included. But there are
% far more serious issues to worry about, such as our limited knowledge of the
% disk temperature, the uncertainties of the dust opacities, as well as optical
% depth effects. So we will estimate the mass of the best fitting Gaussian model
% only.
% 
% We will assume that the intrinsic dust temperature is as given in Table
% \ref{tab-gauss-params} (column $T_{\mathrm{d}}$), which was computed based on
% a simple flaring angle irradiation model. We consider the Gaussian profile with
% the deconvolved width, i.e.\ $w_d$ (not $\sigma$). The peak linear brightness
% temperature is taken to be
% \begin{equation}
% A_{\mathrm{deconv}} = \frac{\sigma}{w_d}\,A
% \end{equation}
% where $A$, $\sigma$ and $w_d$ are from Table \ref{tab-gauss-params}. We do not
% account for geometric terms in the deconvolution, which is justified as long as
% $\sigma\ll r$. The deconvolved Gaussian brightness profile is then
% \begin{equation}\label{eq-tbrlin-deconvolved}
% T_{\mathrm{br,lin,deconv}}(r) = A_{\mathrm{deconv}}\,\exp\left(-\frac{(r-r_0)^2}{2w_d^2}\right)
% \end{equation}
% 
% Next we have to worry about optical depth effects. From Table
% \ref{tab-gauss-params} one can see that for most rings 
% $A_{\mathrm{deconv}}$ is significantly smaller than $T_{\mathrm{d,br,lin}}$.
% However, for the rings of AS 209 the ratios
% $A_{\mathrm{deconv}}/T_{\mathrm{d,br,lin}}$ are around 0.5, so optical depth
% effects start to play a role. Given a dust surface density profile
% $\Sigma_{\mathrm{d}}(r)$ as a function of radial coordinate $r$, and
% given an absorption opacity $\kappa_{\mathrm{a}}$ at the wavelength we
% are measuring, the absorption optical depth $\tau_{\mathrm{a}}(r)$ is
% \begin{equation}\label{eq-tau-afo-sig-kap}
% \tau_{\mathrm{a}}(r) = \Sigma_{\mathrm{d}}(r)\kappa_{\mathrm{a}}
% \end{equation}
% We assume that scattering does not play a role. The emission from this
% dusty ring is then, expressed as linear brightness temperature:
% \begin{equation}
% T_{\mathrm{br,lin,deconv}}(r) = \left(1-e^{-\tau_{\mathrm{a}}(r)}\right)\,T_{\mathrm{d,br,lin}}
% \end{equation}
% If $\tau_{\mathrm{a}}(r)\ll 1$ for all $r$ this reduces to
% \begin{equation}\label{eq-int-afo-tau-bnu}
% T_{\mathrm{br,lin,deconv}}(r) = \tau_{\mathrm{a}}(r)\,T_{\mathrm{d,br,lin}}
% \end{equation}
% But given that for some rings $A_{\mathrm{deconv}}/T_{\mathrm{d,br,lin}}
% =\mathrm{max}(I_{\mathrm{obs}}(r)/B_\nu(T_{\mathrm{d}}))$ can be as large as
% $\sim 0.5$, this condition is not always fulfilled.  On the other hand, these
% optical depth effects cannot be too strong, because then the Gaussian shape
% would flatten off at the top, which is not what we see in the observations. We
% will therefore continue our computation without accounting for these (mild)
% optical depth effects. This will lead to a slight underestimation of the
% mass. The reader can estimate the magnitude of this underestimation by comparing
% $(1-\exp(-\tau_{\mathrm{a}}))$ to $\tau_{\mathrm{a}}$.
% 
% Using Eq.~(\ref{eq-int-afo-tau-bnu}) we can, from the deconvolved
% $T_{\mathrm{br,lin,deconv}}(r)$ (Eq.~\ref{eq-tbrlin-deconvolved}), solve for
% $\tau_{\mathrm{a}}(r)$ and obtain
% \begin{equation}
%   \tau_{\mathrm{a}}(r)=\frac{A_{\mathrm{deconv}}}{T_{\mathrm{d,br,lin}}}\,
%   \exp\left(-\frac{(r-r_0)^2}{2w_d^2}\right)
% \end{equation}
% leading with Eq.~(\ref{eq-tau-afo-sig-kap}) to
% \begin{equation}
%   \Sigma_{\mathrm{d}}(r)=\frac{A_{\mathrm{deconv}}}{\kappa_{\mathrm{a}}T_{\mathrm{d,br,lin}}}\,
%   \exp\left(-\frac{(r-r_0)^2}{2w_d^2}\right)
% \end{equation}
% Integrating over $r$, ignoring geometric terms leads to
% \begin{equation}\label{eq-ringmass-from-obs}
%   \begin{split}
%     M_{\mathrm{d}} &\simeq 2\pi r_0\int_{r_{\mathrm{in}}}^{r_{\mathrm{out}}}\Sigma_{\mathrm{d}}(r)dr\\
%     &= (2\pi)^{3/2}\frac{r_0w_dA_{\mathrm{deconv}}}{\kappa_{\mathrm{a}}T_{\mathrm{d,br,lin}}}\\
%     &= (2\pi)^{3/2}\frac{r_0\sigma A}{\kappa_{\mathrm{a}}T_{\mathrm{d,br,lin}}}
%   \end{split}
% \end{equation}
% 
% \begin{figure}
% \centerline{\includegraphics[width=0.47\textwidth]{kappas_alma_band_6.pdf}}
% \caption{\label{fig-opacities}The dust opacity at $\lambda=0.13\,\mathrm{cm}$
%   as a function of the grain radius $a$. Plotted are the absorption opacity of
%   \citet{2010A&A...512A..15R}, the corresponding scattering opacity, as well as
%   the analytic absorption opacity of \citet{1997MNRAS.291..121I}.}
% \end{figure}
% 
% As the final step we have to make an estimate of the absorption opacity
% $\kappa_{\mathrm{a}}$. This is the biggest uncertainty because it is not well
% understood what the composition of the dust is, what the grain sizes are and
% what the effect of the geometry is on the opacity. We will use two estimates: on
% the one hand we will employ the opacity used by \citet{2010A&A...512A..15R},
% which is a mixture of astronomical silicate and amorphous carbon (see
% Fig.~\ref{fig-opacities}). On the other hand we will use the dummy opacity model
% of \citet{1997MNRAS.291..121I}. These two opacities serve as two extremes that
% bracket the likely true value. It is known that at long wavelengths the
% opacities computed with the Mie procedure from laboratory measured optical
% constants are likely to be underestimates by up to a factor of 10
% \citep{2017A&A...600A.123D}. The more realistic opacities of
% \citep{2017A&A...600A.123D} are in between our two bracketing opacities.  For
% the mass estimate we use the opacity at $\lambda=0.13\,\mathrm{cm}$ for small
% grains ($a=1\mu\mathrm{m}$). For the Ricci opacity this is
% $\kappa_{\mathrm{a,ri}}=1.70\,\mathrm{cm}^2/\mathrm{g}$, while for the Ivezic
% opacity this is $\kappa_{\mathrm{a,ri}}=31.3\,\mathrm{cm}^2/\mathrm{g}$. The
% masses estimated from Eq.~(\ref{eq-ringmass-from-obs}) using these two
% bracketing opacities are listed in Table \ref{fig-obs-gaussfits}. Note that the
% opacity of \citet{2010A&A...512A..15R} has a peak around a grain size of 2
% $\mathrm{mm}$. That peak value happens to be very close to the Ivezic value of
% $31.3\,\mathrm{cm}^2/\mathrm{g}$. We can thus also regard the Ivezic mass
% estimate to be the mass if all grains have a radius of 2 mm.
% 
\end{document}
